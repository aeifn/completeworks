\date{1908-04-12\footnote{в 1908 году было наводнение в Москве}}
\from{Виктор Георгиевич}

Письмо от сына Виктора, папаши\footnote{папаша — Павел, брат Е.К. Кузьмичев} и мамаши\footnote{мамаша — Прасковья Степановна, жена Егора Кузьмича Кузьмичева} Егору Кузьмичу Кузьмичеву

%Дата (предположительно 1908 г)* 12 апреля
%Конверт
%В Село Ярополец Волоколамского Уезда, передать в деревню Васильевское
%Господину Егору Кузьмичу Кузьмичеву
%Иконостащику

%Письмо

Посылаю Вам, родные, как от себя, так и от крестнаго с маманей все благие пожелания и крепко целую Вас.

Уведомляю Вас, что я нахожусь у крестнаго на старой квартире: (Тихвинская ул. д. Карташева, кв. № 6). На страстной неделе говел. В четверг или пятницу, а может и раньше на  Пасху (уеду в семинарию. 

А теперь (свободное время) брожу по городу, в котором меня многое интересует. В Москве много интересующего: люди, их дела, природа, порабощенная людьми. А теперь Москва-река бушует! (Вы небось из газет знаете) сколько улиц залито водой, подвалов, в которых  досталось-таки жильцам их и товарам. Многие помещения размыты: по воде плывут заборы, ящики, бочки, одежа, матрацы, шляпы. Из деревень — крыши, на льду — животные. Не обошлось и без человеческих жертв. 

Многие трамваи не ходят. Извозчики выручаются. 

Беды, беды принесла стихия воды. Бедняки остались без крова, богачи без прибыли. 
«Ох, наказал Господь» — вздыхают богобоязненные.

Крестный и маманя живут по старому: крестный только хворает — простудился, но выздоравливает. Готовимся к Пасхе. 

Остаемся любящие Вас Витя, папаня и маманя.

Москва 12 апреля 


\date{1909-02-20}
\from{Виктор Георгиевич}

%Письмо от сына Виктора (из семинарии)

Подольск% 1909 г. 

%Конверт
%Ярополец  Моск. губ.
%Николаю Гавриловичу Юрьеву
%для передачи Егору Кузьмичу Кузьмичеву
%штамп Подольск Моск.Г. 21.2.09


Милый тятя, спасибо тебе  за письмо. Я очень рад тому, что ты при полном недостатке времени пишешь мне такие хорошие письма. Из твоего ответа вижу, что дела по стройке идут слава Богу. Вот уж думаю теперь у вас такое горячее время: работа, забота, хлопоты. Хорошо, когда это интересно самому: самое обширное поле деятельности и физической и умственной, и умом нужно раскинуть, чтобы тут хорошо было, и тут не упустить, и чтоб красиво и удобно было и по возможности дешевле. Для энергичного человека просто омут делов! Знаю как вы с папашей хлопочете, но очень интересно, как мама теперь себя чувствует? Небось прямо от стройки не отходит, сразу чувствуется самый хозяйственный человек. И мне хотелось бы теперь у вас быть, а только во сне и живешь у вас, да из писем только радуешься на дело.  И то теперь не утерпишь, чтобы с Макаровым не поделиться впечатлениями по получении от тяти письма. Нас это еще больше сближает. Да, кстати, тятя, ты в Москве все возмущался, что мы с Карулиным при встрече ничего путного не поговорили. Зато как шли мы с Макаровым в семинарию от самого Подольска все о самых то высочайших материях толковали: Как нужно понимать Бога, о свободном браке, о святой дружбе и т. п. Мнения наши сошлись как нельзя ближе, только вот одно  и то же  понятие о Боге мы назвали разными именами: он называл Его Совершенством или Силой вселенной, а прямо Богом. Но хотя это все похоже на пустяки, но все же таки интереснее это, чем мелочность. 
Тятя, не напишешь ли мне о Бреневских, что-то от них ни слуху ни духу. Да пиши о Колпских. Если удача заставить встретиться  то передавай мое сердечное приветствие. 
Хорошо бы все время весь год была такая погода: не очень холодно не очень жарко. Действует подбадривающе.  Как близко чувствуется весна и веселит душу! В не замерзшие окна виднеется ослепительно яркий снег, доживающий последние дни своего холодного царствования. На солнце смотреть нельзя: так сильно горит оно в ясной глубокой лазури; лес синеет тонкою дымкою, кажется, ожидая скорой зелени и песен пташек. На дверцах не скрипят уже  пружины; почти все распахнуто и снаружи врывается легкий холодок. За уроком чаще поглядишь в окно на воскресающую природу. Думается весело и легко о милых домашних. Крепко целую всех любящий вас Витя К. 
20 февраля 1909.

Сын широких полей и раздольных лесов ты родился рабом и возрос средь рабов. И окреп среди лишений, нужды и трудов.
От зари до зари ты на пашне с сохой, на росистых лугах с трудовою косой. 
Незаметной тропой твоя жизнь повелась как в степи ручеек тихой лентой виясь
И впотьмах без желаний в груди не искал ты пытливо огней впереди
И казалось бы жизнь твоя скучно бесстыдно прошла…
Но нет!..  О счастливец! Судьба тебе песни дала…
В сердце гудная сила таилась Дар заветный в нем тихо горел… Ото сна твоя жизнь пробудилась -  песен дивный аккорд зазвенел
Благородные сердца порывы тебя вознесли над толпой Лучезарным прекрасным маяком
Ты во тьме загорелся победной главой. 


Рассказ Виктора (сына Егора Кузьмича Кузьмичева)

Зимний вечер на родине

О, как я люблю вспоминать те прошлые дорогие минуты…
Сидишь, бывало, в простой небольшой комнатке, освещенной только одной лампадкой. Вокруг полумрак. Ночь. В камине весело пылает огонь. За окном воет вьюга. Неистово гудит ветер. Но мне тепло в родном гнездышке. Рядом со мной сидит мой отец, и больше никого нет в комнате  кроме нас. Отец и я заняты разговором. Говорим мы о загробной жизни, о учении, какое получает человек счастье от образования и многом другом. Я слушаю речь отца с большим вниманием, и сам задаю ему на обсуждение вопросы. Отец говорит с увлечением, и слова его глубоко врезаются в мою душу. Теплота в этот миг разливается в моем сердце и я чувствую в себе сильную любовь к отцу и жажду к учению. Отец кажется мне чем то великим, с высокими мыслями человеком, который сначала пропадал в грязи и земной прихоти, а потом вознесся над тьмою народной…
О, как бы я хотел почаще видеть такие вечера. 
В. Кузьмичев.
------------------------------------------------------------------------


%* [1912-02-27] < от "Северного утра"
\date{1912-02-27}
\from{Газета <<Северное утро>>}
%Конверт
%почтовый штамп 23.2.12
%Дата
%Логотип
%Северное утро
%ежедневная газета.
%г. Архангельск
%Ст. Ярополец Волоколамского уезда Москосвской губ.
%Егору Кузьмичу Кузьмичеву

Письмо от Максима Горемыки.
%Шапка напечатано типографским способом
%Северное
%УТРО
%ежедневная газета.
%Контора и редакция:
Архангельск,
Троицкий проспект, д. Федосова.
Телеф.391.

Милый Егорушка!
Получил твое письмо. Прочитал его и знаешь что, разревелся белугой, вспомнилось далекое прошлое милое, золотое, увы невозвратимое. И вот, сидя у себя здесь в проклятой конуре, окруженный газетами, которые(ым) нужно (врать)  и рыдать, вспоминая тебя. Помнишь наш с тобой визит в общественный театр в Каретном еще ряду. Все это мне припомнилось и сердце, знаешь, так больно сжалось и слезы подступают к горлу. Не прочитал еще я конца твоего письма, захотелось поделиться с тобой на бумаге своими переживаниями, и я отложил твое письмо и пишу тебе.  Присланного в ответ на старое не читал, но заранее говорю, что (конечно) оно пойдет и (тоже) обмоет слезой твой святой для меня труд. Милый мой  ты поверишь, как это дорого внимание здесь, вдали от родных мест, где чувствуешь себя вполне одиноким.. Кругом чужие люди. Увы, чужие во всех отношениях. Приехал как-то сюда присяжный поверенный Цитрин? (смотри его статьи в Сев. Утре, А.Львов). Так откровенно говорю тебе, он пришел в ужас от той работы, которая  выпадает на мою долю.  Ты понимаешь, я один . Часто приходится отрываться от редактирования заметок и статей, чтобы принимать объявления, потому что лишнего человека держать не из чего. Львов предложил мне поехать в Петербург в какую-то  газету трудовиков (он трудовик  ) и результаты, на днях выяснятся. Трудно очень и очень. (Сплю)  пожалуй, даже и совсем (не сплю). Прочитал о Филиппе* и даже больно сжалось сердце. Ему дали тюрьму и это ужасное нечто. Бедный, тяжело ему будет. Дал о нем в Сев. Утре вырезку. Пиши, Егорушка, милый, я буду тебе писать теперь чаще, потому что чувствую  потребность писать. Боюсь, как бы мне окончательно не застыть на этом севере, коли не от длительного мороза, то от мороза душевного, который пронизывает меня насквозь.
Пиши о себе, как ты? Как твои дела? Пошли тебе Бог всего, всего хорошего. Целую тебя крепко, крепко. Увидишь наших, кланяйся от меня им и целуй их. Постарайся  увидать Филиппа до посадки его  в тюрьму. Вспоминаю о нем ежедневно. Весь твой благодарный 
                                              Максим.

27/II 1912

 * Трудовая группа(трудовики) — российская политическая организация, 1906-1917гг.
 ** Филипп Шкулев
* [1912-10-18] < Ф. Шкулев 
Дата: 18 октября 1912
Источник: Домашний архив Кузьмичевых

Конверт
Почтовая станция Ярополец
Волоколамского уезда Московской губернии
Егору Кузьмичу Кузьмичеву

Другой рукой: из таганской тюрьмы
Внутри конверта: Камера 154 Ф. Шкулев сроч.

Письмо
Штамп М. Г. Т. Корпус №1

18 октября 1912 г.

Дорогой Егор Кузьмич, очень рад твоему письму, за которое искренно тебя благодарю!
Я тебя давно бы послал письмо, но когда-то ты высказывал, что тебе не удобно получать из такого учреждения письма. Поэтому я и не решился тебе писать. Я очень рад, что что за твою дочурку, которую ты как говоришь удачно пристроил, это очень хорошо. Да, армия наша с поля литературы убывает, я очень скорблю за Захарова, о том, что он умер где там в деревне, где ни он никого не знает, ни его никто... и так бедная его могила крапивой и зарастает... печально...
Я за последний приезд твой в Москву очень хотел с тобой видеться и ждал тебя, но ты почему-то не пришел. Это тогда, когда ты пошел куда-то из редакции переписывать свои стихи, помнишь?

Новостей, конечно, у меня в данное время никаких быть не может, поэтому приходится писать только писать только о тюремном житье-бытье. Сижу я в одиночке... Конечно, очень скучаю, если бы ни чтение, то не знаю, что бы со мною и было,  вероетно дошел бы до психии. День ужасно тянется долго, а в первое время особенно.
Теперь немного привык, писать еще ничего не писал, потому что еще не получил разрешения иметь тетрадь.
Когда получу, примусь за писание настегаю своего заезженного Пегаса и буду строчить стихи хоть для детских журналов. Что нового в Литературе? Ты, кажется, получаешь "Русское Богатство", что там интересного? Кажется, там кто один из стаи славных на днях умер? Прочел я здесь в тюрьме стихи П. Якубовича, чудные стихи. Друже, будешь в Москве, разорись, купи (если ты не читал) Смайльса "Самодеятельность". Дивная книга, а тем более у тебя дети, сын большой, а это самая-самая необходимая книга в молодых годах. Купи, настаиваю, я также советовал ее приобрести Андрееву. Прочитал я здесь Американскую республику, сейчас читаю "Галерею французских знаменитостей", издание "Русского Богатства" Кудрина (Русанова). Автор тебе по Р<усскому> Б<огатству> должен быть известен, книга <тоже> чудесная. И так я теперь сижу и глотаю умственные коврижки... а эта пища очень и очень для меня полезная.
Прощай, будь здоров, пиши на каждое письмо немедленно буду отвечать.

Твой Ф. Шкулев.

Приписка: 
На свидание ко мне посторонних не допускают, и свидание пологается только 1 раз в неделю, ходит жена или дети, можно бывать брату, но брат не пойдет.

--------------------------------------------------------------------

* [1912-12-??] < от Максима Леонова
Дорогой Егорушка!

Ты прости ради Бога, прости меня, что долго не писал.
Чем причина та, что так прытко прикован к делу, что прямо-таки нет свободной минуты. Постараюсь исправиться и писать тебе чаще. Не сердись на твоего друга, который любит тебя, всегда помнит и никогда не забудет. Все, что ты прислал, все пройдет Обязательно. «Ополченцы» тоже пойдут как только закончу  эту свистопляску о «соборе святой Софии». Осталось еще фельетонов на 5-6. И пущу твои рассказы. Стихи тоже пойдут все.  Получил вчера письмо от Филиппа из (клетки).  Сидит и читает запоем и даже «увидимся в детстве» пишет детские стихи. Пиши о себе.
Если есть рождественские стихи и рассказы хотя старые, пришли для рождественского №. Пиши. Буду непременно печатать.  Мария Матв. шлет тебе привет.
           Весь твой. Максим Леонов.
--------------------------------------------------------------------- 
 
? дек 1912
* [1913-06-27] < От Павлюка, Максима Горемыки, Александра (Северное утро)
Дорогой 
Егор Кузьмич!
Простите, что долго не отвечал Вам на Ваше послание, не буду оправдываться, придумывать причины, просто откладывал со дня на день. Но покорнейше прошу не считать это тем, что я забываю своих учителей и вместе с тем искренних друзей...  Стихотворения Ваши я передал М.Л. и спешу выразить Вам от него же глубокое спасибо. Подолгу лежат у нас материалы за недостатком места, но скоро газета увеличится и тогда не будет таких проволочек. Москвичи вероятно и меня обвиняют в лени; редко по тем же причинам и мои вещи печатаются. Живу я по-прежнему, ничуть не скучаю, изредка получаю письма от Николая Ивановича и Филиппа Степановича, но от Петра Ал. И Праскунина не получаю совсем. Конечно, они люди долга и им возможно не только недосуг, но и по их суждению бесцельно писать.
Праскунин же вечно кисло настроенный, даже посвятил мне злые стихи, вероятно после прочитанного в «Утре» какого-нибудь гимна природе, ведь он враг чистой поэзии. По-моему же и по воззрениям М.Л. можно ударить раз, ну два громкими вычурными фразами по святым чувствам человека, но нельзя же до бесчувствия. Да и нужно ли отнимать Бога, веру предков наших у людей, если мы не можем дать сами ничего нового, а совсем без веры жить нудно, тяжко.
Простите, что я много наговорил здесь лишнего, хотелось бы слышать Ваше мнение.
Со мной сейчас не знаю, что делается, ведь я и сам был раньше чуть не атеист, ни во что не верил, теперь под влиянием красот вселенной, которых я раньше не замечал, да вернее и не видал, я превращаюсь понемногу в кающегося грешника.
Хорошо у нас здесь. Чудные белые ночи. Солнце светит без устали целых 22 часа в сутки. Воздух напоен ароматом, а луга сплошь усеяны золотистыми точками одуванчиков.
Выписал к себе братишку Павла. Вы вероятно знаете его и видели со мной у Травина.
Ждем от Вас ответа, конечно по возможности скорее, но отнюдь не отрываясь для этого от дела.
              Ваш Александр Андреев Павлюк.

                                               На обороте


Милый Егорушка! Обнимаю тебя и крепко целую. Родной не сердись, что мало пишу. Проклятые дела! Подумай как-нибудь навестить нас. Валите ка с Филиппом.
  Мар. Матв. Шлет привет.
  Весь твой Макс Леонов.

Дорогой Егор Кузьмич!
На днях вернулся из поездки на Мурман наш издатель и сообщил лестную для меня новость: моя «Песенка помора», напечатанная в «Сев. Утре», пришлась по вкусу, и моряки распевают ее под гармонику, что я считаю высшей наградой для себя.
                                              Ал(ександр)
* [1915-06-07] < Ф. Шкулеву 
Дата: 7 июня 1915
Источник: Домашний архив Кузьмичевых

Письмо №2
Дорогой друг
Егор Кузьмич!

Прости, что долго задержал ответом на твое послание, все <> с юбилеем.
Очень жалею, что не было тебя на моем скромном празднике.

Я <> уверен, что если бы ты был, то остался бы очень доволен, так было тепло и откровенно, словом, как то редко бывает.

Приветствия получились от таких лиц, от которых я никогда и не ожидал. Прислал также очень теплое письмо С. Т. Семенов, твой земляк.

Лично у меня новостей никаких нет, живу по-старому, работы мало,  у Травина (?) совсем нет, он ничего не выпускает.

В Москве на днях были ужасные беспорядки, походившие на 905 г. Причина, конечно, недовольство <> и проч. Но пострадало много и не немцев(?), а богатых(?). <> сначала громили, а потом грабили и жгли все, что попало под руки, были и убийства.

Я в Москве бываю редко, не более 1 раза в неделю, сижу дома проводим время на пруду или в лесу с Н. И. Волковым, он теперь в <люб...> на даче, дело свое ликвидировал совсем и теперь без делов.

Да, время переживаем ужасное, проклятая война коснулась всех положительно и наносит всем страшный ущерб не говоря уже морально, но и материально, жизнь вдорожала не по силам, и этому не предвидится конца.

Что же ты не пишешь о своих детках, как и где они? Пишут ли тебе? Живые ли? Я всегда интересуюсь этим как и о своих детях. Будешь писать, не забудь об этом черкнуть. Кстати, как твои дела? Что нового написал? И проч. о всем пиши. Напоминаю еще, не забывай. Будешь в Москве Печатники, буду очень рад, давно так не видались, я очень соскучился.

Будь здоров и не впадай в пессимистику, всем скверно, кроме богатых; и нам с тобой не привыкать слезки проливать... Так-то, крепись теперь, жить недолго, большую половину прожили, а <> как-нибудь дотянем.

Скрипи и крепись!

Твой Ф. Шкулев.

7 июня 1915 годах

P.S. При случае, если увидишь Сергея Терентьевича Семенова, передай ему от меня сердечную благодарность за письмо, на днях я ему собираюсь писать.

Ф. Шкулев

* [1915-03-25] < от В. Г. Кузьмичева 
25 марта 1915

Дорогой мой папа. Не смогу отсюда отблагодарить тебя, так ты заботишься о своем бедном пленнике. По твоему письму теперь жду получек денег. Писем я посылал в начале февраля много за <...> и деньгами. Пишите мне больше и чаще. Что о Саше не пишите <...>

* [1915-05-25] < от В. Г. Кузьмичева 
25 мая 1916

Милый папа! Спешу тебя успокоить, что адреса мои — все правильны. Мне приходилось переселяться из барака в барак, а теперь перешел в I лагерь. Если в будущем придется встретить какой-нибудь другой адрес, знай, что писал его я, а никто другой. Милые родные, вся наша горячо любимая семья! Я страшно иногда загрущу о вас и успокаиваюсь только когда получаю известие, что каждый из вас здравствует. Папаша-крестный небось состарился, а как Миша и потешный сиплый Егорушка? Крепко вас целую. Ваш Витя. Благодарю за стихотворения.
* [1915-06-02] < от В. Г. Кузьмичева 
2 июня 1915

Дорогие мои. Как я теперь рад за Васю. Представьте мое счастье. Я никогда за себя так не страдал душой, как за него! И вдруг он жив! Господь спас его. Теперь я живу спокойнее. <...> дошла, я получаю ваши посылки и деньги и <...> письма. Мне можно пока радоваться на свою судьбу и благодарить Бога за моих добрых Друзей, Витя

Милая Матрена Васильевна, вы посылаете в посылках самое нужное, что я и прошу. Ото всей души благодарю Вас и всю вашу любимую семью. Целую Вас, Витя.

Больше здесь ничего я не жду и не надо.

* [1915-06-17] < от В. Г. Кузьмичева 
17 июня 1915

Дорогой мой милый Вася. Как я счастлив, рад за тебя, что судьба тебя сохранила. У меня с души отлегла половина моего горя. И как только ты остался жив? Милый Вася много ты испытал! Борьбой на поле брани закалил свою юную <...> душу. Теперь желаю тебе утешения в тихой жизни разумного труда себе на пользу. Со слезами целую тебя.

Витя К.
Целую всю нашу милую семью. От папа письмо получил от 1 июня.

* [1915-07-15] < от В. Г. Кузьмичева 
15 июля 1915

Однозвучно звенит колокольчик
Плачет в сердце знакомый мотив
Как тревожит он душу ...
Пережитое вновь воскресив

Было время когда был свободный
Я так счастливо, весело, жил,
И бывало в минуты раздумья
Эту песню я слушать любил.

Что-то в жизни она мне сулила
Роковое на тихом пути
И разлуку, и холод могилы
Пред собою я ждал впереди.

И оно подошло. Захватило
И помчало как в бурю челнок
И погубит ли дикая сила
Или к жизни вернет меня рок?

Помню я, как она провожала.
В поцелуе мы слились в одно,
Как она на груди зарыдала;
«Ты не плачь, так случиться должно»

Иногда в настроении я занимаюсь стихами. <...>. Сейчас получил твою открытку от 27 июня. Пожалуйста, пришли мне «Самоучитель немецкого языка» - полный — конечно — свои еще.

Целую всех вас, Витя.
Конец 1915 - начало 1916

Дорогой папа. Целую тебя и всех милых родных. Поздравляю вас с праздниками, только письмо опоздает. Деньги за ноябрь получил, посылку от Матреши (30 окт.) тоже, но от 17 окт. — нету. Поступил ли Вася на курсы? Я от души советую ему учиться. По себе сужу. Пригодится наука в жизни на всяком месте. Ваши письма я скоро получаю — через 2-3 нед. <предложение стерто>
Пишу и работаю, но теперь у меня много других занятий, да и не чувствую силы на творчество. Когда же ты, папа милый, пишешь? Сидишь небось ночь напролет у столика, а думы черные, невеселые теснят душу. Очень прошу всех написать мне. Грустно. Открыток в Москве <...>.
Пришлите посылочку белья, носков <стерто>, сухари и крупы <...> В. К.



* [1915-07-15] < от В. Г. Кузьмичева 
15 июля 1915

Дорогие родные, что вы выслали, я все получил только деньги за май где-то задержаны. Прошу вас, высылайте <...> посылки через известное время определенно. Иначе теперь сижу безо всего. Милые родные, пишите как живете, как Вася устроился. Напишите ради Бога о Саше, что с ним и Матрешей. Целую маму и всех. Витя.

* [1915-11-04] < от В. Г. Кузьмичева 
4 ноября 1915

Дорогой папа Письма твои тоже получил. Денег за сентябрь нет — а за октябрь получил. Спасибо за книги. Только напрасно ты покупаешь дорогие книги. Андрюше и папаше-крестному вместе посылаю письма. Нам можно посылать в неделю (6 в месяц (?)). Адрес мой прежний и фамилия конечно. Милый папа занимаюсь я все свободное время, но самыми разнообразными занятиями. Одно сочинительство меня не удовлетворяет. Кланяюсь и целую всю семью: маму, сестер и братишек. Посылайте больше крупы и сала для каши. И сухарей.
Прощайте, много писать нельзя.


* [1916-01-13] < от В. Г. Кузьмичева 
Дата: 13 января 1916

Дорогие родные! Я пока жив и здоров, чего и вам желаю. Извещаю вас, что деньги за декабрь я получил от подателя Васи Кузьмичева пав. 21. А чтобы не было еще такой истории как с майскими д. у Ярополецкого почт. м. прошу не доверять таковым. Наверное он послал деньги в другой лагерь. Адреса пишите по-русски все — требует наша почта. Адрес мой пишите: Шнайдемиль, II л. 18 батал. 2р. за № 17315 — В. Кузьмичеву. В партии больных, отправляющихся скоро в Россию едет один наш Волоколамский с дер. Пагубина (?). Прошу тебя милый папа съездить к нему. Милый папа! Письма твои получаю, а от Васи сразу по 2 открытки от 4 и 7 января. Очень рад, что Вася такой молодец: получил награды и теперь так усердно учится. Дорогой папа, я посылаю постоянно 1 откр. В Москву, а в другую среду посылаю тебе. <...> пиши письма с адресом по-русски <...>: Германия, Шнайдермиль II л. 18 бат, 2 рот. за личным №17315. Конечно и посылки и деньги по этому же адресу. За декабрь я не получил еще ни одной посылки. Живу, занимаюсь в школе, читаю, пою в церковном хору. Погода стоит здесь удивительно теплая, но сырая. Папа, береги свое здоровье! Целую маму, п.-крест. братишек, сестр. Прощайте, Витя.

Прощайте, любящий вас В. Кузьмичев.

* [1916-01-27] < от В. Г. Кузьмичева 
27 января 1916

Милый папа! Письма твои получаю, а от Васи сразу по 2 открытки от 4 и 7 января. Очень рад, что Вася такой молодец: получил награды и теперь так усердно учится. Дорогой папа, я посылаю постоянно 1 откр. В Москву, а в другую среду посылаю тебе. <...> пиши письма с адресом по-русски <...>: Германия, Шнайдермиль II л. 18 бат, 2 рот. за личным №17315. Конечно и посылки и деньги по этому же адресу. За декабрь я не получил еще ни одной посылки. Живу, занимаюсь в школе, читаю, пою в церковном хору. Погода стоит здесь удивительно теплая, но сырая. Папа, береги свое здоровье! Целую маму, п.-крест. братишек, сестр. Прощайте, Витя.

* [1916-05-04] < от В. Г. Кузьмичева 
4 мая 1916

Дорогой папа! Шлю тебе свой сердечный привет и целую всю мою милую семью. Теперь у вас идут весенние работы желаю успеха в тяжелых трудах. Не знаю, кто же из семьи будет заниматься в снятом ягодном огороде? Как дела по мастерской? Я послал фотогр. карточки вам и в Москву — узнаете ли меня в группе учителей? Может быть я не изменился внешне, но стараюсь воспитать себя, переустроить складом мысли и характера. Ты, папа, мой идеал. Витя.


* [1916-09-28] < от В. Г. Кузьмичева 
28 сентября 1916

Дорогие родные! Деньги за август получил 53 мар 5 пф. Очень жалею, что много приходится вам посылать мне денег. Я жив и здоров — не болен и потому бодро себя чувствую, смело полагаюсь на судьбу. Работаю при школе в библиотеке. Целую вас, Витя.

Ивану Степ. Кувшинову взаимный привет.

* [1916-12-18] < от В. Г. Кузьмичева 
18 декабря 1916

Милая дорогая сестрица Лида и незабвенный Андрюша! Не могу выразить чувств благодарности за Вашу доброту вообще к нашей семье и ко мне. Известно — я переведен в лаг. Шпроттау — адрес <...> на обороте. За посылаемое не беспокойтесь — все перешлют. Деньги за октябрь получил. Целую Вас.



* [1917-01-14] < от В. Г. Кузьмичева 
14 января 1917

Милый папа! Получаете ли мои письма с новым адресом — очень меня это беспокоит — адрес на обороте. Деньги за окт. и вчера за ноябрь получил — посылки с задержкой, но переводятся из Шнайдемиля. Все ли здоровы в семье? Здоров ли ты, папа? Целую маму, п-крестного, и всех домашних. Витя.


* [1917-01-17] < от И. И. Морозова
scan: scans/3/0003
scan: scans/3/0002
scan: scans/3/0001

Почтовая карточка (на штампе дата)
Ст. Ярополец, Москов. г. Волоколамского у. д. Васильевская
Егору Кузьмичу Кузьмичеву

Штамп: Иван Игнатьевич Морозов

Добрейший Егор Кузьмич!

Все пока по-хорошему. Относительно Андрея Михайловича сделано. Нового пока ничего <> у меня есть. Рецензий не встречал.
Читаю вообще мало: слишком много работы по службе (работаю в конторе и у себя). Пишу совсем плохо и неохотно. После хвори сил поубавилось, голова плохо работает, так что ожидать успехов в 1917 году нельзя совершенно.

Жму руку <> Ив. Морозов


* [1918-05-31] < Тверской Кооператор

Глубокоуважаемый Егор Кузьмич!

С благодарностью получили Ваши письма и стихи. Два из присланных стихотворения - После снега - и - Весенняя радость - будут напечатаны, два других - гражданские, в виду нейтральности кооперативного журнала - - Развяжите руки - и - Современный рецепт - помещены быть не могут.

Журнал по указанному Вами адресу высылается.

С приветом
Николай Рогов
--------------------------------------------------------------------

1919-01-24

Письмо от сына Василия 

Москва, 24 января, 1919 года

Здравствуй дорогой папа!
Искренно желаю здоровья и крепко целую тебя.
Папа! Если бы ты знал, как я о тебе соскучился, как мне хочется послушать твои умные речи и наставления. 
Ты ведь много жил, много испытал и много перенес нужды и горя. 
Ты теперь один мой советчик, один наставитель, я теперь никогда не ступлю ногой своей без твоего совета.
Раньше я не понимал, что можно иметь и ощущать такую любовь к родителям.
Я проклинаю тот момент, когда ушел на войну добровольцем, я был тогда глуп, это верно.
Я не думал, не представлял себе, как будут плакать и мучиться мои родные:  папа, мама, братья и сестры, я так же не знал, как подорвется твое любящее меня сердце.
Да и откуда мог знать, когда на душе у меня не было ничего определенного, «Куда зачем шел» я только думал, только мечтал, что пойду туда, далеко может быть на верную смерть, но я не мог представить ее себе, что это за смерть? и за что пойду умирать. Приехав с войны я не просил у тебя прощения, но я теперь дорогой Папа! Прошу простить меня за все нанесенные мною тебе оскорбления. 
Ах! Папа, как ты поседел мне так и представляется твоя большущая голова, и все это из-за нас детей родных и никто не поймет и не понимает твоих мучений и твоих горьких переживаний.
Я чувствую, что во мне, что-то перевернулось, мое, прошлое воскресло вновь, сердце сжимается от боли, при воспоминании о прошлом. Никогда не читал я так твоих стихотворений, в них ты столько вылил скорби, столько горя и своих переживаний.
Я знаю, что когда ты писал их, ты плакал, и теперь и я понял твою душу и скорбь, и ко мне перешло твое чувство и так-же полились невольные слезы, горя и радости. 
Я написал к тебе письмо, заранее знаю что выражено у меня нескладно, но знаю, что, это написано все с чувством которое только есть во мне  и знаю что ты меня поймешь и не осудишь. Прощай папа, крепко, крепко тебя целую.
Перемен никаких нет, все по старому.
Жду ответа Вася.





* [1919-03-10] < от П. Я. Заволокина 
Дата: 10 марта 1919
Источник: Домашний архив Кузьмичевых

г. Суздаль, Влад. Губ.

Дорогой Егор Кузьмич!

Сегодня получил от Вас письмо и сегодня же пишу Вам, - не сочтете ли Вы нужным принять участие в лит. Сборнике «Автобиогр. Пролетар. Поэтов»?

Если — да! То я просил бы Вас написать Вашу полную автобиографию по <...> (циркуляру) по всем его пунктам и непосредственно от себя препроводить в Петроград П. Я. Заволокину, при чем сошлитесь на меня.

О получении моего письма прошу неотказать меня уведомить.

Остаюсь <...>

Ваш Ив. <...>

P. S. будьте добры сообщите мне адреса: Ал. Ник. Севастьянова и П. Г. Горохова.

---------------------------------------------------------------
* [1919-03-26] < от П. Я. Заволокина 
Дата: 26 марта 1919
Источник: Домашний архив Кузьмичевых
Скан: binary/letters/zavolokin/[12]

Павел Яковлевич Заволокин

Петроград, Кузнечный пер. 8, кв. 8

Многоуважаемый Егор Кузьмич!

Сегодня я получил Вашу Книгу стихов «Из тьмы» - и приношу Вам свое сердечное спасибо. Очень прошу Вас не отказать в присылке своей автобиографии, фотогр. Карточк. (позднейшей), а также 2 или 3 стихотворения, которые еще не были нигде напечатаны. Не забывайте писать на адресе Кузнечный пер. д. №8, кв. 8. т. к. у Вас было написано только <...> и все же дошло.

С почтением уважающий Вас П. Я. Заволокин. 

---------------------------------------------------------------


* [1919-03-28] > П. Я. Заволокину 
Дата: 28 марта 1919
Источник: РГАЛИ

Конверт:
Шифр    1068-1-85_002
        1068-1-85_002а

28 мар<та> 1919
Петроград, Кузнечный переулок, дом №8, кв. 8
Петру Яковлевичу Заволокину

Наклейка: Петроградский военный почтовый контроль

Письмо.
Шифр 1068-1-85  001
                001а

                
27 мар<та> 1919

Свеху в углу: Получено 27/III 919 г. П. З.

Глубокоуважаемый Павел Яковлевич!

По указанию товарища Назарова, я одновременно с тим письмом посылаю вам свои стихотворени напечатанные в сборнике "из тьмы", в котором вы увидите мою полную автобиографию. К ней имею прибавить только то, что я с 1905 года все время до революции был в опальном положении у прежнего правительства, через свои действия, смелые выступления за угнетенную бедноту, среди церковников потерял значение как лучший мастер иконостасных работ. Но, что Бог делает, все к лучшему. Я многое нашел в сельском хозяйстве, которым занимаюсь со всем моим трудящимся семейством и по сие время.

Пишите, будет ли подходить моя автобиография к вашему изданию, и если что необходимо важно прибавить, сообщите. Жду ответа по следующему адресу.

Белая Колпь, Московская губ<ерния> Волоколамского уезда, дер<еврня> Васильевская, Егору Кузьмичу Кузьмичеву.

За последнее время мои стихотворения печатаются в кооперативных журналах и в провинциальных советских газетах.

Душевно ваш, Е. Кузьмичев

1919 года 22 марта

--------------------------------------

* [????-03-28] < от П. Я. Заволокина 
Дата: 28 марта ????
Источник: Домашний архив Кузьмичевых
Скан: binary/letters/zavolokin/from/3_[12]

дер. Васильевская, Шаховского района, Москов. Округа

Кузьмичеву Егору Кузьмичу

Уважаемый товарищ!

Ваша автобиография и рукопись «Что им дала революция» - получены. Рукопись сейчас на просмотре, она, наверно, будет использована в отделе «Хроники».

Адрес редакции следующий: Москва, Площадь Воровского д. 5/21 (НКИД) 2-й подъезд, ном. 12, редакц. Журнала «Наши Достижения».

Шлите В/дальнейшие корреспонденции.
<...> Секретаря редакции 

------------------------------------------------------------------------
* [1919-04-11] > П. Я. Заволокину 
Дата: 11 апреля 1919
Источник: РГАЛИ

11 апр<еля> 1919

Шифр    1068-1-85_003
        1068-1-85_003а

Глубокоуважаемый Павел Яковлевич!

Очень трудно писать что либо о себе. Как не профессионал по перу, а лишь любитель, анимаясь литературой в часы досуга. Конечно, я многого не мог сделать, а тут еще скорбь за скорбью, удар за ударом. Сгорел дом, имущество до тла, с большой семьей деваться некуда, снова постройка, долги на долгие года закабаление в нужде. Далее проклятая война отняла старшего сына (умер в германском плену), был мне близким душевным другом... Другой сын на 15 году своей мятущейся жизни ушел за братом на войну из Москвы, - конечно, не сказавшись, пропадал около года, вернулся награжденный георгиевским крестами. Но, что мне в них, да и он получив ревматизм после разочаровался поняв, что всему служил безсознательно, а я за то время мучился в безвестности, нервничал, горела моя жизнь без толку. Неоторые мои стихотворения напечатанные в сборнике "из тьмы" были навеяны отсутствием моих сыновей.

Революция создала из меня начальника милиции, и я ровно год, до изменения того института, с верою и правдою при настоящем, и прежнем правительстве добросовестно служил родному народу. Был членом губернского исполниельного комитета (при Керенском). Но к партии близко стать ни к какой не мог. Явные раздоры даже в среде партий, меня вовершенно повергли в апатию, я изнервничался и заболел.

За весь период нашей революции пришлось написать очень мало, все-таки несколько стихотворений раскиданы по журналам и газетам, как-то за 1918 год были напечатаны в московских изданиях "Объединение", "Мирское дело" и "Обчее дело" и в местных известиях.

Прошу сообщить, как поступить с прилагаемыми здесь стихотворениями и фотографическими карточками? И биографией?

Душевно ваш, Е. Кузьмичев

Белая Колпь, Моск<овской> губ<ернии> Волоколамского уезда, дер. Васильевская.

4 го апреля 1919 года.

Жду ответа.


=

Шифр:   1068-1-85_005
        1068-1-85_006

Напечатано на пишущей машинке.

КУЗЬМИЧЕВ ЕГОР КУЗЬМИЧЕВ

Родился в 1867 г. в крестьянской семье дер. Васильевской Волоколамского уезда Московской губ<ернии>. Девяти лет окончил сельскую школу. Рано лишившись отца, я с матерью много перенес горя и нужды. До шестнадцати лет занимался крестьянским трудом. На семнадцатом был отправлен в Москву на заработки. Свои юные годы провел среди мастеровых по иконостасному делу. Окружавшая меня среда и бесшабашная мастеровая жизнь была мне не по душе. Проживши три года в Москве, я возвратился в деревню и принялся по-прежнему работать на родимых полях и у себя на дому вводить позолотное ремесло. После долгих усилий мне удалось развить иконостасное дело, через которое значительно <расширило круг моих знакомых>, ко мне обращались купцы, священники и чителя, у которых я изредка брал книги и журналы "Ниву", "Родину", "Вокруг Света", но для прочтения их у меня мало было свободного времени. Через некоторое время я, будучи в Москве, познакомился с писателем из народа М. Леоновым, который дал мне совет побольше читать русских авторов для того, чтобы лучше знать родной язык. После того я прочитал Пушкина, Никитина, Кольцова, Некрасова, Лермонтова, Надсона, Сурикова, Толстого, Тургенева, Короленко, Чехова, Белинского и других.

Чтение родных писателей открыло мне новый мир. Изредка наезжая в Москву, я стал интересоваться и посещать театры, музеи и картинные галереи. На 32 г<оду> своей жизни я написал первое свое стихотворение. Вскоре я познакомился с писателем из народа С. Т. Семеновым, который научил меня как разбираться в естественных науках, и дал полезные указания по физике и химии. Читая статьи Н. Рубакина о самообразовании и другие, я написал ему письмо и послал свои стихи, к которым Рубакин отнесся благосклонно и посоветовал мне не оставлять авторского труда. Первое мое стихотворение было напичатано в журнале "Муравей" (1901 г.), затем в "Детском Чтении", "Народном Слове", "Маяке", в "Газетке для Юношества", "Юная Россия", "Правда Божия", "Об'единение", "Мирское дело", "Общее дело" и в сборнике народных песен изд<ательства> "Посредника". 

Отдельными изданиями произведения мои вышли под названием: "Думы за работой", стихи и рассказы изд. 1904 г. "Из тьмы" стихотв. (второе изд., 1917 г), "Большой подряд" рассказ (1906 г.), а также книжка небольших рассказов "Под крестом" (изд. писателей из народа).

Подпись рукой: Е. Кузьмичев

---

Шифр:   1068-1-85_007

Стихотворения Е. К. Кузьмичева

I

МЫСЛИ

Пусть мысли также будут чисты,
          Как снег белеет за окном, 
Резвы как молния, лучисты
          Как грань алмаза пред огнем.

Пускай оденутся в порфиру
          Весны цветущей золотой.
В лучах на славу всему миру
          Заблещут светлой красотой.

Летят с седыми облаками
          В лазурь далекую небес,
Весной с зелеными лесами
          Плетут узорчатый навес.

Плывут за легкими ладьями,
          Кружась и пеняся в реке.
И пусть над звонкими ручьями
          Как лебедь вьются вдалеке.

Пусть реют в пламенном эфире,
          Гоня усталость темных вежд,
Чтоб с новой силой в грешном мире
          Искать спасительных надежд...

Подпись: Е. К. Кузьмичев
          
---

Шифр: 1068-1-85_008

II

МЫ


Мы душою пламенеем,
Алой радостью горим,
От восторга мы пьянеем,
Что свободно говорим.

    Что свободно, вольно дышим
    В вихре солнечных лучей,
    Крика дерзкого не слышим
    Кровожадных палачей.

Цепи ржавые разбиты,
В прах поверженная власть.
Мощь народная разлита.
Как вулкан бушует страсть!

    Прочь, трусливые, с дороги!
    Время смелых впереди!
    Места нет больной тревоге
    С верой пламенной в груди.

С нею все преодолеем,
Мир коварный победим,
Правду светлую взлелеем,
Царство братства создадим!
                           
Подпись: Е.Кузьмичев

---

Шифр: 1068-1-85_009

III 

ВСХОДЫ


Пасмурная осень
     Все вокруг немеет
И полей широких
     Красота бледнеет. 

Потянуло грустью,
     Злобно ветры веют,
Только всходы шелком
     Ярче зеленеют.

Им не страшен холод
     Зимней вьюжной бури.
В юных светлых грезах
     Видят край лазури.

Обогреет солнце
     И весна разбудит.
Хорошо расти им 
     Под лучами будет:

Колоситься, рдется
     Наливаться тучно,
Жаворонка слушать
     С трелью сладкозвучной.
     
Подпись: Е. Кузьмичев

---

Шифр:   1068-1-85_010

IV



РАБОТНИЦА ЗА ЛЬНОМ


Тяжела со льном забота,
А еще трудней работа
          Выбирать, стелить и мять.
Потемнеешь вся от пота
          День и ночь его трепать.

По зорям спешишь, молотишь,
Горы пыли с ним проглотишь
          В руки тысячи заноз
И мозолей наколотишь,
          Молотя за возом воз.

Что не взмах, то пыли боле,
И закашлешь поневоле,
          Изнурясь в шальном чаду.
Но забыть о лучшей доле
          Не забудешь и в аду.

Вспомнишь, есть на свете люди -   
Точно яблоко на блюде,
          Холят белое плечо,
И сильней забъется в груди,
          Вспыхнув сердце горячо!

У людей тех роза в щечках,
Им напомнит ли сорочка
          О мучительном труде?-
Как за льном не спится ночка
          Нам, работницам, в нужде...
          
Подпись: Е. Кузьмичев


-----------------------------------------------------------------

* [1919-04-11] > П. Я. Заволокину 
Дата: 11 апреля 1919
Источник: РГАЛИ

1068-1-85_011
1068-1-85_012а


Конверт

Заказное

11 апр 1919

Петроград. Кузнечный пер. дом 8, кв. 8.
Гр. Павлу Яковлевичу Заволокину

от Е. Кузьмичева,
Белая Колпь, Моск. губ. Волоколамского

Штамп 2 дек

от вас получено письмо 20 ноября ровно через месяц

Глубокоуважаемый Павел Яковлевич!

Благодарю вас за сообщение адреса. Но сказать что0либо новенького не имею. Мое захолустье окончательно меня давит - никуда не выезжаю, сижу и отчитываюсь тем, чего не пришлось прочесть прежде, да и тут горе - нет освящения - больше скучаю в потемках.

Был запрос моих работ во многие журналы, и посылал, и приняты многие стихотворения, но журналы один за другим закрываются. Почему? Вероятно вам лучше моего известно?

Как видите, жить мне скучно. Нет ли у Вас для меня более утешитвельных новостей? Жду ответа.

Ваш Е. Кузьмичев 21 ноября

-

На обороте: Белая Колпь, Волоколамского уезда, Московской губ.

---------------------------------------------------------------

* [1919-09-12] < от Семенова
конверт 
Белая Колпь
(Моск. Губ. Волоколамского уезда)
дер. Васильевское
Егору Кузьмичу Кузьмичеву
штамп 
Новосокольники-21
12  9  19

Письмо
12 сентября

Вчера, возвратившись из Москвы, нашел Ваше письмо. Я думаю, что в четверг на будущей неделе я буду дома и, если Вы в этот день приехали бы, было бы хорошо.
В Москве плохо. Журналы «Общее дело» и «Мирское дело» закрываются. Закрываются подобные журналы и в провинции. Пишу в Ярославский союз, чтобы узнать, как обстоят дела там.
Ну так до свидания.
Всего лучшего
(Подпись) С. Семёнов
* [1919-09-25] < Политический отдел северных жел. дор.
Многоуважаемый Георгий Кузьмич.

Согласно личных с Вами переговоров и выраженного Вами желания, покорнейше прошу Вас прибыть в Москву в Упавление Северых ж. д. для переговорово Вашем поступлении сотрудником в Политический Отдел Сев.
Волоколамский Уездный Совет прошу неотказать в выдаче разрешение на право въезда в Москву.

Председатель Политического Отдела Сев. ж. д.

Смирнов

* [1919-10-07] < от Семенова
конверт
Белая Колпь
Егору Кузьмичу Кузьмичеву
штамп 
8  10  19

Письмо
7 октября 1919
штамп
Сергей Терентьевич СЕМЕНОВ
Адрес почт. Отд. 2 Бухолов Московск. Обл.

 Подписано рукой дер. Андреевское

Я затрудняюсь, милый мой Егор Кузьмич, сказать Вам положительно, что я могу взять редактирование Ваших стихов на себя. Дело это для меня неподходящее. К стихам у меня ведь известно какое отношение. Я боюсь в них спутаться и принести Вам не столько пользы, как это может сделать более привычный ценитель такого рода искусства. Да я написал уже и Белоусову, хотя ответа от него еще не получил.  Оставим пока этот вопрос открытым. Относительно же той стороны дела — где печатать книжки: не в госуд.(арственном) из-ве(издательстве), а у Бонч-Бруевича или в «Завтра»*, то на это скажу, что с Бонч-Бруевичем не стоит связываться. Он (порядочный) (путаник), а в «Завтра» едва ли еще есть такой отдел, где бы могла выйти Ваша книжка. Я узнаю при случае и, если окажется, что возможно, с удовольствием окажу содействие, а пока подождем. Книжечку о льне я пересмотрел вновь. Она очень хороша, одна беда: все (???) уже поотстало от условий хозяйственной жизни крестьян и издатель,пожалуй, опираясь на это, не так охотно согласится переиздать ее. Все-таки я покажу ее Кильчевскому, когда буду в Москве.
Ну пока, всего хорошего. Какова погода-то, прямо золотая осень
Привет всем Вашим.
Ваш Семенов.

   * В 1919 году Е.Замятин, К. Чуковский и А. Тихонов планировали издавать поэтический журнал «Завтра». Планы  не осуществились. 


* [1919-12-29] > П. Я. Заволокину 
Дата: 29 декабря 1919
Источник: РГАЛИ

Шифр:   1068-1-85_013
        1068-1-85_014
        1068-1-85_014а

Штамп: 2 дек<абря> 1919

Петроград, Проспект 25 октября (б. Невский пр.) дом №51, кв. 16

Павлу Яковлевичу Заволокину

Сбоку: Ответ отослан 7/XII 919

--

Глубокоуважаемый Павел Яковлевич!

Извиняюсь за долгое молчание на ваше любезное письмо, как то не писалось совсем, что-то в душе таится пассивное, нехорошее - вероятно окружающее действует очень дурно. Вокруг все злоба и злоба, не могу выносить.

Пожалуйста, не откажитесь внести следующую поправку в мое стихотворение "Работница за льном", предназначенное для сборника "пролетарских биографий".

Снизу 7-я строка -
Вспыхнет злоба в чахлой груди
и 6-я строка
Станет сердцу горячо.

Как о вам покажется? Мне думается, эта поправка ярче, живее? А впрочем и того не могу отличить - прошу вашего совета. Время еще нескоро пожалуй позволит вам распорядиться изданием сборника? Пишите мне - что и как?

Все же лучше на душе становится, когда встречаем живое - тем более ласковое слово товарища!

Вы на меня, пожалуйста, не сетуйте много за дурное настроение - это очень скоро пройдет и слова снова постараюсь начать человеческую жизнь, настоящую светлую в моей мятущейся душе.

Желаю вам всего лучшего в жизни - главное душевного спокойствия.

Ваш Е. Кузьмичев

1919 года Декабря 29 дня

Приписка сверху:

Для сборника "Сельское хозяйство в красной позии" по вашему совету стихи посылаю одновременно с тим письмом. Пишите.






* [1920-02-23] > П. Я. Заволокину 
Дата: 23 февраля 1920

Шифры:  1068-1-85_015

Конверт

Петроград, Проспект 25 октября (б. Невский), дом № 51. кв.16. Павлу Яковлевичу Заволокину.

Штамп: 9 янв<аря> 1920
Сбоку красной ручкой: ответ послан 17/II 920 г

--

Письмо

Шифр: 1068-1-85_016

Глубокоуважаемый Павел Яковлевич!

С исправлением моего стихотворения "Работница за льном" я совершенно согласен и спешу его отослать вам обратно.

В редакцию журнала "Вольный пахарь" по вашему совету я что-то посылал, но не получал оттда ответа, приняты ли мои стихи? Очень буду благодарен вам, если по выходе 1-го №-ра пришлете его по моему адресу. Стоимость журнала вышлю вам с благодарностью.

С искренним приветом остаюсь к вам благодарный Е. Кузьмичев

23 февраля 1920 г.

---------------------------------------------------------------------------

* [1921-07-23] > П. Я. Заволокину 
Дата: 23 июля 1921
Источник: РГАЛИ
Шифр: 1068-1-85-017, 1068-1-85-018

Конверт

Заказное

Павлу Яковлевичу Заволокину.
Петроград, Проспект 25 октября (бывший Невский) дом 51, кв. 16

Штамп 4 мар<та> 1920

от Е. Кузьмичева

Белая Колпь, Московской губернии, Волоколамского уезда.

---

Многоуважаемый Павел Яковлевич!

Спешу ответить на вашу просьбу, и прилагаю три стихотворения "Агитатору", "Родине", "Счастливый век", на которые прошу немедленно ответить мне. Что и как? Жду письма.

Благосклонный к вам Е. Кузьмичев

23 июля 1921 года.

-----------------------------------------------------------------

* [1921-07-29] > П. Я. Заволокину 
Источник: РГАЛИ
Шифр: 1068-1-85_019

Конверт

Заказное

Синим карандашом: получено 29/VII 1921

Петрогдад, Проспект 25. октября (бывший Невский), дом 51, кв. 16.

Павлу Яковлевичу Заволокину.

от Е. Кузьмичева
Белая Колпь, Московской губернии.

---

1068-1-85_020

Кузьмичев Егор Кузьмич

р. в 1867 г.

карандашом: 

1). Автобиография - 2
2). Стихотворения - 4
3). Письма - 6 




---------------------------------------------------------------------
* [192?-??-??] > Сыну Василию
Письмо Е.К. Кузьмичева своему сыну Василию

Без даты

Здравствуйте дорогие Вас(), Оля!
С удовольствием разглядываю еле заметную отметку х) на  Васиной открытке он проживал во время пребывания в Сочи. Какие великолепные постройки, знаменательные виды, а самое приятное чувство является от того, что здесь находился близкий -любимый человек. Да милы Вася, ты мне близкий не только по родству, а и по многим чертам твоего характера. Отчужденность если и появляется (???)час у тебя, но она какая-то напускная, от какой-то чужой  мне внешней обиды. Вас много там... близких мне, и, иногда в ненужном споре треплется без вины мое имя... Лично я ни в чем никогда не был, и не буду я тому виною. Я любил тебя с детства, и эта любовь не уменьшил, а скорее увеличил к тебе во всех отношениях. Вместе с тобою люблю и уважаю Олю. Никакие разговоры неприятно касающиеся Вас я не способен слышать… Да их при мне и не бывает…
В настоящую пору моя (жизнь?) тянется не для себя. Но я бываю очень доволен от переживания общих успехов. Я с любовью встречаю людей которые способностями выдвигаются вперед. Из таких в будущем почетное место в моих желаниях твое и твоих братьев, но это не по родству, а по Вашему достоинству, по Вашим поздним, но пламенным стремлениям к науке на пользу Советского социалистического строительства. 
Ваше движение вперед уносит и меня с Вами!..
(стерто)-то с горечью сказал людей таких как я, стремящихся к науке, но не имеющих возможности достигнуть — жаль. Да мой дорогой. Это было сказано вполне по человечески т. е. искренно и разумно. Похожее на это отношение обо мне было напечатано в газете «За коллективизацию» прошлою зимою. Ты пожалуй не читал? Напрасно, пусть я учен поповной… А в моей биографии есть много интересного — особенно для Вас — людей науки… Пока довольно! Приветствую Вас мои дорогие и конечно желаю всего хорошего.
Любящ. Вас Ваш папа,
(приписка сбоку ) Мама целует Вас и благодарит за ласку.

К письму приложена фотография. ( на фото первом ряду в центре Вася, слева его жена Ольга).  

С обратной стороны написано:
Снимались в Тройцин День, 29 июня — 1921 г.
Подписи: М (Г или Л)ущиков, (М или Т)ичурин, Д. Ченд(?)ав (о или а),  (стерто), А.М. Долгов. 
Рукой А.П. Кузьмичева -  дядя Вася и тетя Оля жена 

* [1923-04-27] > Ивану Ивановичу 

Источник: РГАЛИ
Шифр: 122-1-775
Дата: 27 апреля 1923

электронная версия документа выкуплена для семейного архива

Глубокоуважаемый Иван Иванович!
По Вашему желанию - посылаю вам стихотворения памяти С.Т. Семенова. Одно из них «Венок на могилу» было напечатано в газете «Беднота». Другое вероятно будет напечатано в журнале «Новая деревня». Если представится случай вам — напечатать в каком либо сборнике? - Пожалуйста используйте. Буду очень благодарен, если вышлите книжечку.
Душою с вами Е. Кузьмичев
27 апреля 1923 г.
Почтовое отделение Белая - Колпь, Волоколамского уезда, Московской губернии,


Почт. отд. Ярополец, Московск. губерния Волоколамск. уезд
Глубоко уважаемый Иван Иванович!
Одновременно с этим письмом посылаю вам свой сборник стихотворений «Из тьмы». Может быть что нибудь подойдет из сборника для ваших мелких изданий песенников, Очень буду рад если что нибудь выберете, и напечатаете в ваших изданиях.
Мои стихи я встретил в четырех вами подаренных мне книжечках, в следующих сборниках «Лгун ?». Красная девица». «Милый друг». и «Молодая казачка». Чему глубоко радуюсь.  Радуюсь всякой  ?, которая хоть чуть чуть меня сближает с вашим достоуважаемым книгоиздательством.
Примите уверение в глубоком уважении к вам.
Ваш покорный слуга Е. Кузьмичев



На смерть С.Т. Семенова
зверски убитого 4 декабря 1922 года

В пылу бушующего зверства
  У темных сил пощады нет!
В хаосе диком изуверства
  Семенов пал во цвете лет.

Коварный враг добра и света -
  Клевет незримая змея -
Не пощадила в нем поэта -
  Нашлась преступная семья.

С винтовкой бросилась, с наганом!
  И честный гражданин угас,
Как солнце вдруг, — лучом багряным
  Вдали блеснув последний раз...

Но после жертв — сильнее всходы!
  Ведь ночь не вечна, тьма пройдет.
Уже горит заря свободы,
  И солнце красное взойдет!

Повсюду светлый дух воспрянет,
  И с ним Семенов будет жив!
А злых убийц народ проклянет,
  Позором черным заклеймив!..

Крестьянин Кузьмичев





Венок на могилу
крестьянского писателя
С. Т. Семенова     

Схоронили друзья без попов...
  Там, где озимь его трудовая
Будет нежно шептаться без слов,
  Тучный колос к могиле склоняя.

Налетит ли шалун-ветерок,
  Или туча с грозой пронесется,
Снова в тихий к нему уголок
  Взглянет солнце и вдруг - улыбнется.

Точно знамя - что пахарь любил
  Труд и нивы полоску родную,
Выбиваясь на пашне из сил,-
  И согреет могилу сырую.                 

  Крестьянин Е. Кузьмичев
  
  Волоколамск.

-------------------------------------------------------------------
* [1926-05-11] > Е. Вихреву 
Дата: 11 мая 1926
Источник: РГАЛИ
(электронная версия выкуплена для семейного архива)

На конверте:

Заказное
В Редакцию жрнала "Город и деревня".

Москва, ул. 1го мая,
(Б. Мясницкая)
секр. Ред: Еф. Вихреву.

от крестьянина Е. К. Кузьмичева,
Белая Колпь, Волоколамского у.
Московской губернии.


= Письмо:

Уважаемый т. Вихрев!

Ваше письмо все таки не совсем расхолаживает. "Отсталая Весна" только опоздало! Одно мое стихотворение ,(?) общими силами было напечатано в Вашем уважаемом журнале "Город и деревня" (за)? 1925 год. Поэтому имею надежду на лучшее будущее, и на Ваши советы "Пишите". Написал и спешу отослать к Вам на просмотр. Если окажуться годными прилагаемые стихотворения "Одинокая" и в "Союзе Се-се-ер" то пожалуйста посодействуйте поместить в Вашем журнале "Город и деревня".

В ожидании желательных результатов
С искренним приветом к Вам остаюсь
Е. К. Кузьмичев

11 мая 1926 г.

------------------------------------------------------------------------

* [1929-05-29] < Ширин-Юреневский Дмитрий Ильич
От: Ширина-Юреневского

Дорогой Егор Кузьмич!
Письмо Ваше получил — искренно благодарю.
Все, что Вы пишите несомненно частица есть, т.е. ставка на имена.  Погоня за качеством и (только?) не совсем благополучно, есть напечатанные стихи по своей теме никуда не годны. Это заключается поспешностью и невнимательностью браковщиков. У Вас (?) очень искренние стихи без фальши, без вычурности. Я очень внимательно прочитал Вашу книгу стихов «Из тьмы», она наполнена правдивой поэзией. Мне думается, или журнал предназначен для крестьянского читателя, то он должен иметь материал для указанной аудитории. Вот, как я понимаю. Теперь относительно Ваших стихотворений отданных в журнал «Взаимопомощи» одно взяли, когда используют не знаю — говорят скоро, остальные у меня. 3 июня открывается съезд крестьянских писателей. Съезд будет интересным — приезжайте Е.К.. Приглашены М. Горький, С. Подъячев, С. Дрожжин и др. Будешь в Москве заходи поговорим. Желаю быть Вам здоровым.
С искренним приветом (Размашистая подпись Ширин-Юреневский).
29/V29 г.
Москва ул. Арбат д.4 кв 5
--------------------------------------------------------------------------------------------------------

1921-06-29

Письмо Е.К. Кузьмичева своему сыну Василию


Здравствуйте дорогие Вася, Оля!
С удовольствием разглядываю еле заметную отметку на  Васиной открытке, он проживал во время пребывания в Сочи. Какие великолепные постройки, знаменательные виды, а самое приятное чувство является от того, что здесь находился близкий -любимый человек. Да милый Вася, ты мне близкий не только по родству, а и по многим чертам твоего характера. Отчужденность если и появляется подчас у тебя, но она какая-то напускная, от какой-то чужой  мне внешней обиды. Вас много там... близких мне, и, иногда в ненужном споре треплется без вины мое имя... Лично я ни в чем никогда не был, и не буду я тому виною. Я любил тебя с детства, и эту любовь не уменьшил, а скорее увеличил к тебе во всех отношениях. Вместе с тобою люблю и уважаю Олю. Никакие разговоры неприятно касающиеся Вас я не способен слышать… Да их при мне и не бывает…
В настоящую пору моя (жизнь?) тянется не для себя. Но я бываю очень доволен от переживания общих успехов. Я с любовью встречаю людей, которые способностями выдвигаются вперед. Из таких в будущем почетное место в моих желаниях твое и твоих братьев, но это не по родству, а по Вашему достоинству, по Вашим поздним, но пламенным стремлениям к науке на пользу Советского социалистического строительства. 
Ваше движение вперед уносит и меня с Вами!..
(стерто)-то с горечью сказал, людей таких как я, стремящихся к науке, но не имеющих возможности достигнуть — жаль. Да мой дорогой. Это было сказано вполне по человечески т. е. искренно и разумно. Похожее на это отношение обо мне было напечатано в газете «За коллективизацию» прошлою зимою. Ты пожалуй не читал? Напрасно, пусть я учен поповной… А в моей биографии есть много интересного — особенно для Вас — людей науки… Пока довольно! Приветствую Вас мои дорогие и конечно желаю всего хорошего.
Любящ. Вас Ваш папа.
Мама целует Вас и благодарит за ласку.

К письму приложена фотография. ( на фото первом ряду в центре Вася, слева его жена Ольга).  

С обратной стороны написано:
Снимались в Тройцин День, 29 июня — 1921 г.
Подписи: М (Г или Л)ущиков, (М или Т)ичурин, Д. Ченд(?)ав (о или а),  (стерто), А.М. Долгов. 
Рукой А.П. Кузьмичева -  дядя Вася и тетя Оля жена 

----------------------------------------------------------------------------------------------------

Письмо Деева-Холмяковского Григория Дмитриевича*


4.02.1929

Давненько дорогой друг не писал ты мне.
Жив ли ты Егор Кузьмич? А я ведь о тебе нет да нет и пару строк напишу. В №7 «революция и культура»** наверное видел. А вот сейчас еще дал статью в «Учительскую газету» и «Селькор». И везде тебя вспоминаю. Не хорошо Егор Кузьмич, старых друзей то забывать. Как ты живешь, что поделываешь? От чего ты мне не подарил книжечку с выпущенными твоими стихами ВОКП, ведь собирал то их не то иной, а кажется и я. Да и вообще тебя я не забывал, а чем мог помогал. Ах, Егор Кузьмич, вот Вы какой народ: Был Д-Х — издатель ну и тут все к нему, а как я «с глаз долой так из сердца вон»… Вот то то оно и есть. Не в обиду сказал тебе, а так к слову. Сейчас мы с небольшой группой людей преданных организовываем кружок писателей «Дружба». Мал он, по деловому то, по пролетарски подходит к поставленным партией задачам!
Работа хорошая и сердечная. Ну пока, не забывай, а что нового есть пришли.
С приветом Деев-Холмяковский
Москва - (?).М. р. Шереметьеская ул.3.




*Деев-Хомяковский, Григорий Дмитриевич 
(род. 1888) — литературный деятель, поэт. Член ВКП(б). До Октябрьской революции — пастух, рабочий, потом — педагог, секретарь "Будильника", руководитель "Суриковского литературно-музыкального кружка". После Октября — редактор нескольких журналов и активный член и учредитель разнообразных литературных организаций. Автор ряда исторических и литературно-критических статей, пьес, рассказов, стихотворных сборников. Литературные произведения Д.-Х. по б. ч. лишены художественности. Известность он приобрел в качестве руководителя крестьян-литераторов, узко ориентировавшегося на т. н. самородков-самоучек.
Лит. см. Владиславлев И., Литература великого десятилетия, том I, М.—Л., 1928. 
http://dic.academic.ru/dic.nsf/enc_biography/18962/%D0%94%D0%B5%D0%B5%D0%B2


**Революция и культура (журнал)— культурно-политический журнал журнал ВКП(б). Выходил с 1927 по 1930 год, как издание газеты Правда. Печатался раз в две недели. Статьи об изобразительном искусстве, как и некоторые другие, сопровождались иллюстрациями. Выходил под редакцей Н.И. Бухарина и А.В. Луначарского.

*** ВОКП -  Всероссийское общество крестьянских писателей. 



* [1929-07-24] < Михайлов Н. Е.
Светлые Горы

Простите, дорогой любимый Егор Кузьмич, что так долго не писал Вам. Главная причина та, что я никак не мог застать на квартире Н. А. Степного.
Да, его трудно застать (был 7 раз). В последний раз он передал соседям по квартире (он одинок сейчас), если будет Михайлов, дайте ему ключ от моей комнаты и пусть ждет меня, но я не не дождался его все-таки так как и у меня в Москве времени совсем не бывает. Н. А. говорит в своей биографии, что "мать снабдила меня восточной пассивностью. Отец непоседливостью, кипучестью. И в этой скале колебаний от низкого до высокого, бьется пульс моей жизни..."
Да, эти строки правдивы: Степной такой и есть.
Слышу (позже?) о М??.Игоре Степной разъезжает по ??гову. Я решил свидание с ним оставить до осени.-
II-VII, случайно в тресте ТЭНОЭ (Никитская Ю), я встречаю Степного (кстати в это время вынув из портфеля, он подарил свою книгу "Молодое" X том, между прочим, этим томом он и закончил свое полн. собран. сочинений) я ему передал, как его дождался.Знаю, знаю тов. Михайлоов, но это поправимо.
Давайте числа 15-20 VIII соберемся у меня в комнатке, пишите Кузьмичеву я рад ближе познакомиться с ним.
С удовольствием, с удовольствием. Раньше не могу, ведь я разъезжаю по городам и ставлю свою последнюю пьесу (не запомнил названия это пьесы). Сейчас еду в Ярославль, дальше в Нижний, к 15-20 VIII я буду в Москве. Заходите заходите с Кузьмичевым, я рад». - и мы расстались.
Хорошо бы приурочить Вашу поездку в Москву к этому времени и нам встретиться, предварительно списавшись еще раз, узнав прибытие Степного в Москве точнее.  Я это сделаю. - Я добивался беседы на квартире у Ник. Ал. Потому, что у меня созрела мысль, что через Степного я думаю, удастся издать Ваши стихотв. разбросанные в газетах, журналах и т. п. за последние 10 лет. 
Как хорошо бы!
Как Вы сейчас живете, давно ничего не знаю о Вас, как здоровье, самочувствие, благополучие и душевный покой?
На крест. съезде писателей не работали? Какое впечатление? По печати видно, что этот съезд происходил как-то по казенному. Я плачу тихо. Дни, месяцы идут неустанно к роковой? неизвестности, которую все не минуют рано-ли поздно-ли.
Толя хромает по 4 предметам, в последние месяцы взялся за себя и с успехом перешел в VII группу. Учителя довольны, что их предположения о способности его оправдались, а я тем более. -  Пока кончаю.
До свидания, дорогой, старый друг, еще раз простите за долгое молчание.
Любящий Вас  (размашистая подпись) НМихайлов.

------------------------------------------------------------------------
* [1929-??-??] < от А. А. Демидова
Дата: предположительно 1929 год
Источник: Домашний архив Кузьмичевых
Шифр: leonov/1

Дорогой Егор Кузьмич!

 

Получил и Ваше письмо от 9/I с несколькими вопросами. Спешу ответить на них. Пусть не удивляет вас <...>: он в отношении всех одинаков, его <...>! В нашей организации произошли большие изменения, все переругались, <...> руководящий состав и до сих пор нет оргсекретаря в Центр. Совете! Совет переехал в здание Крестьянской газеты и я не вижу никого... А <...> какие-то новые, молодые ребята, которые едва-ли знают членов нашего общества и ждать от них вызова не приходится, да они и не знают зачем кого вызывать...

Говорят, скоро организация обновится и заработает, авось тогда и с вами установят более живую связь.

Я тоже думаю, что вы не утратили и не утратите писательских прав из-за невзноса в горком сборов, и, несомненно, остаетесь членом Лит<...>, раз вы платите взносы. Это Вы продолжайте делать, хоть сейчас они ссуд и не выдают, но пригодится на будущее.

Что касается моих новый вещей, то я боюсь сказать точно, когда увижу их в печати! Около трех лет <тружусь?> над романом «Лес» и вот теперь снова когда «Лес» еще не созрел, не вырос, я его вырубаю! Уже 7-й или 9-й раз местами... Хочу кончить месяца через два-три. Сначала он будет печататься в ж. «Земля Советская», а потом будет издан в газете «Федерация» - договоры заключены. Во всяком случае, как только выйдет этот роман книгой, я ее Вам пришлю — в порядке товарищеского обмена и во исполнение Вашего желания прочитать ее.

Переиздавать в 1932 году ни одной книги <...> не будут — из-за отсутствия бумаги и многие из нас, кто не даст новые книги, вынуждены будут искать службу в учреждениях: не одного вас коснулась эта беда...

Но я надеюсь, что производство бумаги через год увеличится и снова увеличится выпуск художественной литературы и тогда, - чему бы я был от души рад, - авось увижу книжку Ваших стихов последних лет!

 

А пока — крепко жму Вашу руку и желаю вам здоровья и всего хорошего.

------------------------------------------------------------------------
* [1930-01-17] <
scan: 0004


Дорогой друг!

Сейчас только что окончилось общее собрание (12 ч. ночи). На этом собрании кроме наших служ. были еще 10 ч рабочей бригады, присланной работать на селе по подготовке посевной комп. и коллективизации. Жить будут у нас в крас. Уголке, месяц вероятно. В феврале или марте, у нас перевыборы Месткома. Мне как секретарю М.К. придется поработать очень и очень изрядно. А и то стараюсь удлинить свой день, ложусь впать все позже, встаю все раньше, а время для личной жизни остается все меньше и меньше. Работая в 3х пепрвыборах секретарем М.К.и в других комиссиях( с самогомоего поступления в Л-< >)Меня это очнень утомило. Сейчас я буду очень довольна, если меня никуда не выберут. Хотя немного отдохну и приведу в порядок свои дела, кототорые сейчас заброшены. Сечас М.К. даны большие права, но еще большие обязанности, которые отнимают много времени. Начиная с Воскр. всю эту неделю работала в амб. с новыми врачами.Он ничего. Вежлив.Общителен. Не суетлив.Сегодня речь говорил на собрании. Разумно и тактично. Лицо у него довольно добродушное. Что дальше будет не знаю. Сегодня я отличилась. Поздравте.
Прихожу в амб. утром, а там в перевязочной стоят оба врача. А как вошла, так прежде всего взглянула на часы. Там было 11 ч. и ровно 15 м. Они оба это видели. Ничего они мне не сказали и я им тоже. (на 15 м я опаздала). Но мне было очень не приятно. Вы конечно поймете почему ( Не боязнь выговора) а принять одолжение.Это верояно меня навсегда отучит от опаздываний. Когда он пришел в амб., то спросил где я и хотел послать сиделку за мной. После< >т.е. моего прихода, минут через 5, он ушел домой и сейчас же вернулся.(вероятно проверял часы)Зубврач перд этим ему говорила, что часы эти спешат (желая выгородить меня) и такая неволя предстоит мне до самой смерти. Я же свое рабочее время постоянно почти дарила по царски лечебнице и больным и никогда не считала.
Довольно о скучных вещах.
Мне тоже очень жаль, что Вы уехали из Москвы и письма стали запаздывать. Очень хорошо, что Вы пишите дневник. Может быть, когда нибудь, Вы мне прочтете из него, хотя немного. Там, вероятно, так же много красоты, как и в Ваших письмах.
Читая Ваши письма,кажется, что Вы как чародей, ведете меня к какому то волшебному дворцу и открывая дверь за дверью, показываете мне, залитые Солнцем залы, наполненные разнообразными сокровищами, котрые храняться там.
Ваше теплое письмо, полное дружескокого участия, дает мне луч надежды на лучшее будущее. Мне снова кажется, что моя счастливая звезда, сопутствовавшая мне в моей жизни еще непомеркла.
С искренним приветом Ина
17/I 1930 г.
5 ч. утра, а я еще не ложилась спать.
Пишите

---------------------------------------------------------------------------

* [1930-01-26] Михайлов
26.01.1930
Открытка от Михайлова
Куда: п.о. Белая Колпь, Волоколам. Район, Москов. Области
Кому: Поэту Егору Кузьмичу Кузьмичеву

Печатка — Николай Егорович Михайлов 
«Светлые горы» 26-I-30
Из открыточки мой дорогой Егор Кузьмич я заключил что: о своем положении Во мне будете еще писать, но Вы молчите, а мне так хотелось бы узнать о Ва  подробнее. Почему Вы в Москве (пожили?) больше 2 месяцев, а может быть и сейчас и живете? - Знаете, что С.Д. Дрожжин написал свои воспоминания за 60 лет (сейчас они  в печати).
Я думаю о Вас, а почему  Вам не написать за 60 лет (?) (?) Вы лично видели и хорошего и плохого, встречались с людьми известными многим, хорошо бы Вы поделились о них и о себе в воспоминаниях. Жду подробного письма. У меня опять скоро будет переменена в жизни. Подробнее в закрытом. Любящий Вас (подпись Михайлова).
Приписка
Белоусов И.А. умер — как жаль.
* [1930-06-06] < от Шкулева
Письмо Шкулева от 6.06.1930.


Конверт
П/О Белая Колпь, Шаховского района, Московского округа
Егору Кузьмичу Кузьмичеву

Адрес отправителя
станция Люблино-Дачная, Моск. Кур. Ж.Д. деревня Печатники, Вокзальная у., д. 9 Ф.С. Шкулев


Письмо
Добрый день 
Егор Кузьмич!
Я так и полагал, что наши письма разойдутся, но это ничего. Отвечаю на твое 2е письмо. За посвященные мне стихи глубоко тебя благодарю и мысленно крепко целую. Только сегодня узнал, юбилейный вечер будет 11 го июня в доме Герцена. О чем тебя и извещаю, билета не посылаю тебе, потому что не имею сам.
Председателем Юбилейного вечера Демьян Бедный. Это я пишу для того, что лишние знакомства нам с тобой не мешает. 
Получил письмо из редакции газеты «За коллективизацию» от заведующего литстраницей, который пишет, что твои стихи и биография которую я писал, будут напечатаны в ближайших №№ Лит-страницы. Значит почитаем, это меня радует, что наши труды не пропадают даром.
Если бы не моя болезнь, я бы теперь работал вовсю (среди?) на литературной ниве. Это моя была мечта. Но сбыться ей не удалось, такова значит участь, ну что делать.
Теперь несколько слов о погоде последних дней. У нас ночью заморозки доходили до 3 градусов, что очень плохо отразилось на некоторых овощах. Огурцы померзли, уже пересеянные 2й раз. Также померзла ботва молодого картофеля. Какая была погода у Вас в эти дни 29, 30, 31 мая и начала июня 1,2,3..
Что будет нового и интересного сообщу, а пока будь здоров.
Твой Ф. Шкулев

6/VI-30 г.
Станция Люблино-Дачная Моск. Кур. Ж. Д. 
деревня Печатники, Вокзальная ул., д № 9
* [1931-03-08] > А. А. Демидову 

Источник: РГАЛИ
Шифр: 165-2-43 003
Дата: 1931-03-08

Дорогой товарищ
Алексей Алексеевич!

Пожалуйста найдите возможным написать две-три строки для меня о том, окончен ли пересмотр членов МОКП и что сулит мне этот пересмотр?.
Мне кажется, кляузное письмо еще крепко держится в руках комиссии. Проклятой клеймо иконостащика, учеба моя на медных пятаках вероятно заставит быть исключенным из организации ВОПКП, несмотря на на то, что я революционно воспитал свою семью, воспитывал массы и сам с сыном коммунистом агрономом не за страх, а за совесть, не покладая рук до сих пор работаю на прорыве.
Комиссия не позволяет лично мне повторить выступление. Ужели комиссии пятнадцать минут времени дороже той правды, которую она сама стремится выискать в каждом проверяющемся члене?
Дорогой тов, загланите в правление МОКП, узнайте, получены ль заявления с справками от меня и моего сына. Мы оба еще надеемся найти в т.т. Достаточно гражданского мужества для защиты неопровержимой правды о том, что я всю свою жизнь стоял за свободу пролетариата.
И теперь, несмотря на происки моих врагов завистников как селькор, верным красным стражем стою за центральную линию социализма, за нашу руководительницу ВКП(б). Подумайте, надо ли меня искусственно толкать в контрреволюционный лагерь? Жду ответа.

 С тов. Приветом исренно уважающий Вас
Е.К. Кузьмичев
8 марта 1931 г

Подписано сбоку:
Волоколамск, Моск. обл. Советская ул. дом 25


-------------------------------------
* [1931-05-09] > А. А. Демидову 
Дата:1931-05-09
Источник: РГАЛИ
Шифр: 165-2-43

Конверт:
Куда: Москва, Центр.
Никольская ул. 1(4|7)?
Кв. 1. (во дворе)
Кому: т. Алексею Алексеевичу Демидову



Письмо:

Шифр: 165-2-43 001

9 мая 1931 года
Дорогой тов. Алексей Алексеевич!
Звание писателя - звучит хорошо.
Но еще громче звучит человек! Вот к тому вечно громкому названию вы и принадлежите. Вы один из прекраснейших людей откликнулись на мою справедливо-вопиющую жалобу. В небольшом письме ко мне Вы сказали все, что необходимо в настоящее время для нашего успешного Советского строительства. Вы снова дали мне доброе настроение подняли мои усталые члены 65 летняго ратоборца на пользу укрепления коллективизации. На первомайские торжества слетом колхозников Волок. района я был откомандирован в Москву Торжественность небывалая. Чувствую вторую молодость, жаль что не видел Вас, ну-да лучшее время впереди. Может быть снова напишете старику?
Ваш Е. К. Кузьмичев. Волоколамск, Советская ул. 25



-------------------------------------
Вырезка из газеты
Источник: РГАЛИ
Шифр: 165-2-43 002
Дата: 1931-09-06
Примечание: из переписки с А. А. Демидовым

Газета "Волоколамс<кий> колхозник" 52 (88) 6 сентября 1931 года

Молодежному дню

Повсюду чахнет капитал...
Все грани скоро будут стерты,
Но с божьей проповедью, с чортом
Вредить злой враг не перестал!
Международный Юный День, -
С неисчислимыми толпами,
Цветя кумачными флагами,
Рассей враждующую тень!..
Над нашим темпом "пяти лет"
Она как хищный ворон машет...
А над полями свежих пашен - 
Неугасимый, яркий свет!
От нас прожектор сноповой
Лучи рассыпал над землею.
Колхозы тесною семьею
Стоят на страже мировой...
В броню одет рабочий люд
И комсомольцы, пионеры...
С горозой бессильны лицемеры,
Когда так зорко смотрит МЮД!

Е. К. Кузьмичев

Ниже заголовок
"К XVII МЮД`у"


-------------------------------------


* [1931-06-11] > А. А. Демидову 
Источник: РГАЛИ
Шифр: 165-2-43 004
Дата: 1931-06-11

Дорогой Алексей Алексеевич!

Глубоко тронут Вашим хотя и маленьким, но искренне душевно-теплым письмом! Пусть комиссия ошибается. Пусть несправедливо огорчает, но имея таких в лице Вас истинных ценителей справедливости, революционной чуткости, я по Вашему совету - действительно успокаиваюсь.

Недвано прочел в "Известиях" Вашу вещь "Штурм леса". К прежнему, что я Ваше читал, начиная с "Жизни Ивана" "Вихрь" "Село Екатерининское", и чисто художественное произведение "Зеленый луч". "Штурм Леса" имеющая современный характер нашего Советского Строительства, - незаменима (подчеркнуто). В ней четко отображена успешная работа ударников. Каждому читателю видно, что автор был тесно связан с этою коллективно-успешной работой, читатель видит перед глазами отображенный под снеговым покровом лес. Людей отаптывающих корень сосен, людей спешно безшумно подрубающих, подпиливающих. Видит как столетний Красный лес валится дрг на друга рисующий Советскую Коллективную Елочку. Великолепно все описание. Нет за чтобы зацепиться сомневающемся читателю. Писатель был в вековом лесу, видел, и как будто сам участвовал физически, работал с красными ударниками в советском строительстве.

Если бы Вы без сомнения поверили моей искренности Как Ваши письма оживляют меня старика. То никогда бы их не оставили без ответа. Ваши письма для меня служат жизненным элексиром и сладким отдыхом. 

Приветствую Вас дорогой Алексей Алексеевич, и желаю всяческих успехов в Вашей Литературной работе существующей на великую пользу Советского социалистического строительства!

Душою с Вами, Е. К. Кузьмичев. 11 июня 1931 года

Сбоку: Волоколамск, Советская ул. дом 25.

* [1932-04-09] Михайлов Н. Е.
09.04. 1932
Письмо от Николая Егоровича Михайлова 
Москва 9-IV-32
Дорогой Егор Кузьмич!
Только сейчас приехал с курсов (?) в Химки. Часы показывают полвторого ночи, спать почему то не хочется. Какие-то (?) (?) и решил написать Вам. - 
Вот месяц, как я не писал Вам; так складываются обстоятельства, что буквально не выберешь времени, чтобы Вам написать. Как получил Ваши строчки, все время собирался написать, но всякий раз, что-нибудь да помешает.
Только полночь дает мне покой, но так устаешь, что не пишет рука, да и в голове какой-то сумасброд от дневной сутолоки с напряженными (?); так  что Вы меня не браните, если я изредка буду молчать. 
Что же делать, приходится мириться мне и с тем, что я не имею возможности даже почитать. Вот  у нас в Москве были литературные вечера; Ник???горова, Новикова прибыл, Мстиславского, Вересаева Пастернака, Гусева В., (поэт) но (?) (?), при всем (желают?), быть не пришлось, а так хотелось бы.
Печатали ли Вы в «Вол. Колх.» последнее Ваше стихотворение «Гражда???» (к 8/III) Пришлите вырезку. Получил ли книжечку Максима Горького «В. И. Ленин». Хорошо написано, искренно, но я все таки не могу понять почему Алексей Максимович кое-что  выпустил, что было написано им тут же после смерти В. И. Ленина. Кое что он прибавил и нового. Читается легко. Я очень рад, что Вы теперь имеете возможность читать наших и иностранных классиков, в деревне Вам труднее было в этом отношении.
Чувствую, что скоро Вы возвестите в стихах о семенах, о севе, о (?) на колхозных полях, как приятно читать любимого поэта дерев???. 
Пишите, пишите, дорогой мой Егор Кузьмич.
Всего хорошего.
Любящий Вас ( подпись Михайлова)
* [1932-??-??] > А. А. Демидову 
Дата: 1932
Источник: РГАЛИ


Шифр:   165-2-43 006
        165-2-43 006 -7
        165-2-43 007 -8

?1932(1?)

                                              Дорогой глубокоуважаемый тов. Алексей Алексеевич!

Нет человека верней, - кого бы вновь спросить, а как все-таки разрешила комиссия по чистке окончательно вопрос, в котором так мерзко меня оболгали?.. Снова с нетерпением жду Вашего ответа, ибо надеюсь, что Вы проверите. Спросите МОКПе. Мне официально ничего не сказано. На днях мой сын вносил за меня членский взнос в Местком писателей. - Приняли.
Я думаю, что, если бы выключили из МОКПа, то и постараются стереть мое имя и из Месткома? Может я очень забегаю вперед? Однако, оставаясь в неведении свойственно додумываться вплоть до галлюцинаций. Ведь с 1904 г значусь организатором товарищеских кружков, не боясь жандармских преследований. Никакие угрозы-лишения не отводили меня от пера самоучки. А теперь, когда вышли на путь свободы, на путь завоеваний., нас, стариков игнорируют. Даже, если и выключили, не стараются честно, по-товарищески, известить о решении!
Мой правильный адрес есть в Месткоме, но приглашений на собрания не получаю. Тем более, из МОКПа давным давно. Я чувствую, что многого проявить себя не могу. Но я из таких ведь не один. Прочие т.т. до сих пор все-таки пользуются приглашениями, а значит и уважением? Что же у меня-то разве рога на лбу выросли? Лучше бы спросили наш волоколамский райком или наше местное ОГПУ относительно людей политической благонадежности, если наши руководители так толстокожи, что никакими стихотворными излияниями их чувства не пробудишь. Впрочем я чувствую, что моих произведений они не читали, не смотря на то, что эти произведения возглавляли каждую выставку. Все нарушено и осмеяно. Мне кажется, вряд ли есть лад и с Вами, и с прочими т.т. Следя за литературной газетой, я теперь вижу «одну головку», что все-таки не мешает говорить о прочности дружбы с выдвиженцами рабочими и колхозниками. Вновь учат, а старых, положивших всю жизнь,  пожертвовавших делу Революции, теперь пожалуй и в клуб фосп(?) не впустят. Я не надеюсь будучи в Москве туда протесниться. Напишите, как Вы смотрите на это?. Ваш Е.К.


-------------------------------------

* [1932-??-??] < ?????
Волоколамск Московской обл. Советская, 25
Е.К. Кузьмичеву

Дорогой Е. К.!

Ваше стихотворение из письма моя жна перепечатала и отнесла в ж "Работница" - Тверская 8.
Принявшая стихотворение, оказывается, знает Вас и напишет Вам ответ непосредственно.

Всего хорошего!

Подпись (неразборчиво)

* [1933-02-22] > А. А. Демидову 

Письмо
Источник: РГАЛИ
Шифр: 165-2-43 005
Дата: 1933-02-22
img: letters/writers/demidov/1933-02-22/1

Дорогой Алексей Алексеевич!

Глубоко благодарен Вам за утешительное письмецо... После утраты сына впервый раз попробовал свои силы в любимом занятии стихотворства, и прилагая здесь, очень прошу проверить и, искренне ответить мне письмом, как стихотворение реагирует на читателя! Главное мне необходимо то, имеется ли смысл назвать т. Сталина "оплотом Красной армии"? Я назвал может быть не продумав, хочется помощи, пожалуйста неоткажите, Вы человек партийный. "Ленинизма завет", Помоему "интернационал" есть ленинизма завет! И поется свободно у нас только после Октябрьской победной революции. Мне кажется нет в стихотворении темных мест, и между прочем я все-таки неуверен, желательно подтверждение со стороны безкорыстно рассуждающаго человека, каким я Вас знаю, и прошу немедленно меня старика уведомить.

Еще раз прошу незабыть обещанного экземпляра"Лес", аожет недождался издания книги Пришлите №(?) журнала, где роман печатался в прошлом году. Очень желательно - пока жив, ознакомиться...

В заключение Приветствую Вас мой дорогой. Приветствую Ваше милое семейство на скорое письмо. Желающий Вам всех светлых благ в жизни

Ваш Е. Е. Кузьмичев

22 февраля 1933 года

Сбоку: Волоколамск, Советская ул. дом 25.




-------------------------------------
* [????-10-07] > А. А. Демидову 
Источник: РГАЛИ
Шифр:   165-2-43 007 -8
        165-2-43 008
Дата: ????-10-07
                                                                       
07.10.(1932?)


                                                Дорогой 
                                      Алексей Алексеевич!

  Прилагая стихотворение как колхозника-производственника прошу высказать о нем Ваше мнение и, если найдете интересным, то направте в наше ВОПКП(?). Может быть используют в сборнике производственных стихов. Хотя вещь газетная и ударная, должна увидеть свет в самом скором времени, именно теперь, когда начинается мятье льна.
Не обижайтесь мой друг, что я Вам напомню о прежней беспокойной просьбе моей, т. е. о том, как обстоят дела с проклятой чисткой МОКПа, касающейся моего оскорбления.
Думаю, что мои духовные палачи не распутали, а еще сильней запутали чуждый их совести вопрос... Ведь в сущности эта не чистка была, а выгонка стариков-Суриковцев, кого может быть заслуженно, а таких, как меня, за компанию. Сами же себя палачи совсем отвели от удара, минуя как святыню, не желая показать добросовестного примера...
Простите мой друг. Вы получите письмо как раз 10-го, когда должен возвратиться Дементьев, который пожалуй тоже увильнет от прямого ответа. Пишите, что меня ожидает. Готовлюсь к худшему!

Ваш Е. К. Кузьмичев, 7е окт

Сбоку: Волоколамск, моск. обл. Советская, 25.

-------------------------------------

* [????-??-??] > А. А. Демидову 
Источник: РГАЛИ
Письмо (о чистке)
Шифр РГАЛИ:     165-2-43 009
                165-2-43 009 -10
                165-2-43 011
                                                                                                                Дата???(о чистке)

                                                              Дорогой
                                            Алексей Алексеевич!

Мой приезд в Москву все-таки оставил за собой след. На слет были приглашены только активисты-селькоры. Меня, как старого работника газеты «Московская деревня», отдельно от других засняли и попросили к 1000-му № «За коллективизацию» написать мои воспоминания о работе со дня выхода «Моск.дер.». Я привел в исполнение просьбу, указав на 
литстраницу данную «Моск. Дер.» в 1925 году, где была приведена моя биография и стихи. Вся эта работа вместе с увеличенной моей фотографией была помещена в стенгазете, выставленной у дверей редакции. Далее может быть помещена в газете «За коллективизацию» ибо фотография снова перепечатана.
Мое выступление как ударника заинтересовало заведующего издательством тов. Ройцеса, который пригласил в кабинет для ознакомления. И вот тут-то выплывает вонючая история с чисткой нашего МОКПа. Как же после этого не волноваться человеку, прожившему всю жизнь прямо-неуклонно идя весь свой долгий век путями революции?.. Заведующий меня успокаивал, что кляузная история чистки МОКПа мало его интересует. Но чувствуется, что неизмеримо лучше бы всякое недоразумение устранить!
Думают ли об этом когда-нибудь те люди, что теперь поставлены у нас в организации на видные места?.. Нет, видимо не думают! Еще весной при встрече со мной и вопросе моем что и как делается комиссией, член ее тов. Гамайский с иронией заявил, что ему интересней скорей найти пачку папирос... В таком случае не их ли необходимо стране надо чистить в организации, чем стариков, отдающих все свои чувства, все способности общественному делу. А их совсем причислили к «лику святых», - не чистили. Вы являетесь невольным свидетелем, имеющим право подкрепить мои недовольства к работе комиссии, ибо Вам при мне по телефону ответил т. Тартан, что он ничего не знает о решении комиссии, так как не был при ее окончательной работе. По моему мнению кому другому, а Тартану совсем непростительно уклоняться от этой путаницы. Он где-то нашел кляузно-подметное письмо. Он его не постеснялся объявить мне в лицо при собрании. Он после этого читал мои оправдятельные документы, признавая их сильным доказательством, служащим для оправданий. О н из первой комиссии по чистке перешел в другую. Он согласился при мне с Алексеевым, что дела мои сами за себя оправдывающе говорят за меня, и вдруг перед самым решением сумел стереться — назваться ничего не знающим?.. Позор-позор ложится-виснет не мне на голову, а нашим т.т., попавшим видимо по ошибке на место судей недостойным общественного доверия. Эти люди какой-то крокодиловой кожей обтянули-затянули себя, чтобы не чувствовать чужого беспокойства, чужой скорби...
Вы меня еще раз простите, мой дорогой Алексей Алексеевич! Поверьте, мне очень больно! Больно старику, прошедшему тяжелую дорогу, пережившему жизнь в нужде, порою в крайней бедности, в то же время не уклонялся от ударов царской жандармерии. Все переиспытавшему за интересы крестьянства. Мне, старику, по праву принадлежит все лучшее при советской власти, ибо я всю семью свою с собою вместе поставил красной строкой на фронте Социалистического Советского строительства, внеся все имущество, силу, энергию в коллектив тогда.
* [????-??-??] < Пролетарий (газета)
Уважаемый тов. Кузьмичев!

Ваше предложение о сотрудничестве в нашей газете принимаем. К сожалению, первую Вашу заметку "Вечер-смычка" напечатать не можем. В ней нет никаких фактов о работе колхозов вашего р-на, а общие фразы и сообщения о "пламенных приветствиях" колхозам читателя не интересуют.

Сообщите нам конкретные факты о том, как коллективизируются деревни шаховского р-на, о классовой борьбе, о хлебозаготовках, работе парторганизаций и всем том, что интересует батрацко-бедняцкие и середняцкие массы р-на.
Ждем Ваших деловых и интересных заметок.

С товар. приветом.
Зав. массов. отделом. АК

* [????-??-??] < от Максима Горемыки
                                Родной Егорушка!
Милый, славный и все прочее такое.
Все получил и стихи, и оба рассказа. «Все будет» напечатано от странички до странички. «Домой» уже набран и пойдет на днях после рассказа Лобачева. Стихи твои тоже все будут напечатаны, я их получил. Пущу залпом подряд. Пожалуйста, родной мой,не обижайся, что не пишу. Кипяток, в котором я сижу по самые уши, причиной тому. Будут скоро два праздника, напишу тебе несколько больше. Будь здоров.
   Придумай-ка с Филиппом погостить ко мне, ей Богу, соберитесь серьезно.
                    Твой Макс
Увидишь Филиппа, поцелуй его.

* [????-02-12] < от Подъячева
Письмо Подъячева Е. К. Кузьмичеву

Без конверта? (Год надо найти по православному календарю)

Письмо

12 февраля
Прощенное воскресение!
Дорогой мой и искренний поэт!
Вы на меня прошу Вас не (?),  что так долго не отвечал на Ваши письма. Было  и недосужно и вообще как то за последнее время я закрутился. Но только Вы раз на всегда знайте, что я Вас помню и бога ради не думайте (если случиться долго не отвечаю). Что делаю я это потому что не хочу отвечать.. Напротив, я очень ценю и люблю Вашу хорошую душу которая выливается в стихах нисколько не хуже и не меньше чем хотя бы у Дрожжина.
Ваших книг у меня уже нет. Я их отослал кое куда. Но сам ей богу еще ничего не мог написать по поводу Ваших стихов. У меня сейчас горе: иду отправлять новобранца и т.д и т.д. Это уже и (?) (?).
Есть к Вам просьба: Вы пишите что читали мои рассказы (И?) Е.Ж.  А нет ли у Вас моего рассказа напечатанного в «Русский бог под названием «(Ш?)». Я нигде не могу найти и забыл в каком году — а очень нужен. И потом если у Вас найдутся  мои рассказы в Ежемесячном журнале (у меня тоже нет), то напишите какие и если будете в Москве то захватите их с собой. Очень буду благодарен. Я так небрежно отношусь к своим вещам, что растерял половину, а сейчас как раз нужно для издания.
Все то что Вы пишите в письме Вашем мне очень знакомо. И не стоит об этом говорить. Надо глядеть на жизнь по другому. Все это ведь временно. А то, что есть в Вас то вечно, так об чем же толковать. Вы носите в себе Бога живого и довольно с Вас, а остальное все поверьте наплевать.

Жму руку.
Искренно уважающий Вас С.П. Подъячев.

По случаю прощенного воскресения 
1 Христа ради простите меня
Всего, всего хорошего!
P.S> Работайте и не скулите!!..

* [1938-04-16] < Ширин-Юреневский
16 апреля 1938 г 
Письмо  от Ширина-Юреневского (отпечатано на машинке)

Здравствуй дорогой Егор Кузьмиче
Довольно продолжительное время я не вижу Вас, пропал наш старый певец. Пишу и не знаю, жив или нет? наш бодрый поэт. Хотел спросить Федю Чернышова, не вижу за последнее время.  Правда я сам долгое время в отъезде. Но все же не забыл Е. К. славного старика.
Если жив, жду ответ. Недавно были с В. В. Горшковым на похоронах писателя П. С. Романова, говорили о Вас.
Я решил написать Вам письмо, авось откликнется Егор Кузьмич.
Жму руку Д. Ширин-Юреневский
16 апреля 1938 года
Москва
Подпись
Москва у. Арбат, д 4, кв. 51
ва у. Арбат, д 4, кв. 51
