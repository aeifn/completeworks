\section{Пименов}

В недавно открытые, по случаю летнего времени, двери чайной в селе Заречье вошли два посетителя. Один был плотничий подрядчик Шилов, мужик средних лет с гсто русой бородой, довольно скромный на вид; другой заречский богатей Пименов. Пименов был юркий, еще довольно свежий старик низенького роста, с хитрым лицом и редкой цетинистой бородою. Волоса его были причесаны прямым пробором и примаслены маслом. Ястребиные глаза Пименова всегда зорко бегали по предметам.

Пименов был <<давнишний>> --- за свою жизнь он много испытал, знал нужду и работу; но теперь он давно уже жил в необычайном довольстве. Он содержал огород в Заречье, имел лавку, где можно было встретить всевозможные товары: чай, сахар, крупу, муку, деготь, сбрую и <<супоги>>, как говорил Пименов. До монополии Пименов содержал трактир и очень любил трактирское дело. <<Золотое дно, --- говорил Пименов, --- на постелю еще деньги несут и почтенье отменное воздают, ни от какого товара такого барыша не наживешь>>. Но как только провалилось <<золотое дно>>, то Пименов стал говорить, что чайная торговля одна суета, и сдал прежний трактир в аренду одному молодцу из ближайшей деревни, прежде жившему в Петербурге и за что-то высланному на три года оттуда административно.

В семье у Пименова была старуха жена, да дочь горбатая. Был у него когда-то и сын, но он давно ушел в один дальни монастырь и там постригся. На расспросы не знавших Пименова, почему сын задумал идти в монастырь от всего отцовского богатства, Пименов привык уже всем отвечать, что так Богу угодно, при этом он старался объяснить, что мирские блага --- тленные блага, что единая благодать для человека чистота души, а она приобретается молитвою и постом.

Пименов был большим церковником, много лет служил в своем приходе старостой, любил поговорить о божественном. За это он пользовался большим уважением. Успех его в жизни и вторговых делах многие приписывали его радению о храме, в которых он когда-то отлил колокол и поддерживал кое-какие украшения. <<Он для Бога и Бог для него, --- говорили многие, --- когда покупает он, цены-то стоят грошевые, а продает --- рублевые; а цену кто же строит как не Бог?>>

Когда Шилов и Пименов вошли в чайную, там было пусто: были будни и последние дни весны --- кто доставал сев, кто копался в огородах; только за буфетом помещался новый хозяин чайной, мужчина лет 30, приятной наружности, с коротко остриженной темно-русой бородкою, с веселым, живым взглядом, просто, но опрятно одетый. Он сидел за прилавком и читал газету. При входе гостей он отложил газету в сторону и поклонился им.

--- Будет читать-то, зачитаешься, --- со смехом сказал Пименов, протягивая ему руку, --- собери-ка нам чаку.

--- Двоим? --- спросил приятным голосом хозяин.

--- Двоим нам, да себе третьему, значит троим.

Хозяин молча стал собирать чай. Пименов и плотник расположились за столом. Когда чай был собран и к гостям подсел хозяин, Пименов спросил:

--- Ну, как поторговываешь?

--- Плохо, --- ответил хозяин.

--- И в лавочку не ходят?

--- Из лавочки вчера отпустил на красненькую, да и то в долг.

--- Кому же?

--- Бабенке одной --- приходила из Лобутина.

--- Эва, ты уж бабенок приучать начал, --- проговорил Пименов, осклабившись, как кот при запахе мяса, --- ну, если так, то жди барышей!..

Хозяин однако оставался серьезным и проговорил:

--- Тут не до барышей.

--- Так значит штучка хороша! --- опять воскликнул Пименов и засмеялся гадким смехом. Хозяин с укором взглянул на него.

--- Можно вам смеяться!

--- Что же мне, плакать теперь?

--- Да и веселого-то мало... Тут такая история...

--- Если занятная, расскажи, я истории люблю слушать, --- проговорил Пименов и стал разливать чай. Когда посетители принялись за чай, хозяин начал:

--- Меня она удивила даже. Сижу я вот так, она входит, гляжу: молодая еще, лицо красивое, только смотрит грстно-грустно и дета плохо. Подошла, поклонилась; я, это, гляжу; а она: <<Иван Федорович, будь отец родимый, выручи из беды>>. --- Что такое? --- <<Помоги моей нужде>>. --- Какая у тебя нужда? --- <<Замуж дочь отдаю, а подняться не с чем, отпусти кой-чего на свадьбу>>. Я гляжу и глазам не верю: очень молода-то, а уж дочь невеста! --- Как, говорю, дочь --- свою родную? --- <<Родную>>. --- Давно ль же ты замужем? --- <<Десять лет>>. --- А сколько ж лет дочери? --- <<Шестнадцать>>. --- Что же она у тебя, до замужества прижита что ли? --- <<До замужества>>, --- говорит, а сама слез проглотить не может...

--- Я думал дело! --- сердито перебил рассказчика Пименов.

--- Отчего ж это не дело? --- изумился хозяин, --- вы послушайте, что дальше будет: --- Ты такая молодая, говорю, каких же лет ты сделалась матерью? --- опять спросил я. --- <<Шестнадцати годов с половиной>>. --- Что ж, знать полюбила очень кого? --- Она оправилась и говорит: <<Полюбила. Была я, --- говорит, --- без отца, без матери, жила, говорит, в работницах, мне и полюбился хозяйский сынок; такой хороший, ласковый он был и меня полюбил; <<не оставлю, говорит, я тебя, Дунюшка>>, стало быть, хотел жениться на мне, а отец его, хозяин-то мой, и не позволил. Как он добивался, как
