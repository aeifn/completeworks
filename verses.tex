%% Текст сверяется с оригинальным изданием, вносятся очевидные грамматические исправления, при этом приводится пояснение (комментарий) и ссылка на академический источник-подтверждение. Оригинальными сохраняются также отступы. Используется окружение altverse. Для отступа в общем случае установлено 4 пробельных знака. 


%134
\poemtitle[<<Тогда народ прославит Бога...>>]{***}
\settowidth{\versewidth}{Тогда народ прославит Бога}
\begin{verse}[\versewidth]
Тогда народ прославит Бога,\\ %%Поставили запятую, в тексте оригинала (правлен автором поверх) ее не было
Святую истину поймет ---\\
Завет у мирного порога\\
Христа Спасителя найдет,\\
%К богатству, --- славе охладеет,\\
К богатству, славе охладеет,\\
Оценит он тяжелый труд,\\
Любовью теплою согреет\\
%Сирот, и темный бедный люд,\\
Сирот и темный бедный люд,\\
Тогда сольется он душою\\
%В свободный общий братский вздох\\
В свободный общий братский вздох,\\
%Воскликнет радостной толпою\\
Воскликнет радостной толпою,\\
%Что в мире есть любовь и Бог, ---\\
Что в мире есть любовь и Бог,\\
%Когда святое вдохновенье, ---\\
Когда святое вдохновенье,\\
Зажжет ему угасший взор\ldotst\\
%Постигет кроткое смиренье\\
Постигнет кроткое смиренье\\
%И, чистой совести укор\ldotst
И чистой совести укор\ldotst
% Одна из первоначальных версий стихотворения
\if 0
Тогда, мой друг, прославишь Бога,\\
    Святую истину поймешь, \\
Завет у мирного порога\\
    Христа Спасителя найдешь, \\
Когда презришь ты лицемерье\\
    Души погибельный позор,\\
Постигнешь кроткое смиренье\\
    И чистой совести укор,\\
К богатству, к славе охладеешь,\\
    Оценишь ты тяжелый труд,\\
Любовью теплою согреешь\\
    Сирот и темный, бедный люд;\\
Тогда сольешься ты душою\\
    В свободный общий братский вздох,\\
Воскликнешь радостно с толпою,\\
    Что в мире есть Любовь и Бог.
\fi
\end{verse}
1899 г.


% сначала идут стихотворения издания 1904 года

\poemtitle[<<Над моею над головушкой...>>]{***}
\settowidth{\versewidth}{Не грозой с небес и не дождиком}
\begin{verse}[\versewidth]
Над моею над головушкой\\
Туча темная собиралася,\\
Не грозой с небес и не дождиком,\\
Людской злобою разражалася;\\
Да Господь послал солнце красное ---\\
Сквозь седой туман появилося,\\
Улобнулось мне так приветливо,\\
В серде радостью отразилося.
\end{verse}
1904

\poemtitle{Дорогое детство}
\settowidth{\versewidth}{\vinВ родной деревне прожитое,}
\begin{verse}[\versewidth]
\begin{altverse}
Люблю я детство золотое,\\
В родной деревне прожитое,\\
Люблю порою вспоминать:\\
Как в поле с матерью, бывало,\\
% По правилам пунктуации добавили запятую после вводного слова ``бывало''
% Ссылка http://new.gramota.ru/spravka/punctum?layout=item&id=58_76 
Идешь в работе помогать,\\
% Вместо оригинального двоеточия после помогать поставлена запятая (по смыслу) 
Мы утро с радостью встречали,\\
Меня тогда не омрачали\\
Заботы, бедность и тоска.\\
Я знал одно святое счастье, ---\\
Любовь родимой глубока\ldots\\
Лишь взглянет мать: уж чуешь ласки,\\
Бежишь, горят, как угли, глазки,\\
Огнем пылают сердце, грудь,\\
И знаешь, верно, у родимой, \\
% Перед словом верно добавлена запятая по правилам пунктуации
% http://new.gramota.ru/spravka/punctum?layout=item&id=58_267 
Гостинец спрятан где-нибудь.\\
Иль ночью, в зимние метели,\\
В избушке теплой на постели\\
Обнимет ласково рукой,\\
Опять в груди почуешь радость, \\
Прильнешь к родимой головой\ldots\\
Давно то время миновало ---\\
Хранит земное покрывало\\
В могиле тихой милый прах.\\
Уж стар и я, белеет иней\\
В моих кудрявых волосах.\\
Но вспомнишь только лишь порою, ---\\
Воскреснет детство предо мною,\\
И мир мне кажется другой\ldots\\
И снова матери желанной\\
Я вижу образ дорогой.
\end{altverse}
\end{verse}
% Сверено с оригиналом. Исправлена пунктуация. Егор, Ольга 19/12/15
1904

\poemtitle{В получку}
\settowidth{\versewidth}{Загуляли, закружились}
\begin{verse}[\versewidth]
\indentpattern{001}
\begin{patverse*}
Загуляли, закружились,\\
Словно разума лишились,\\
Дачке рады.\\
Удальцы мастеровые\\
Деньги сыплют трудовые\\
Без пощады.\\
Эй, сдавайте поскорее,\\
Нам сыграть всего милее\\
С перетемкой.\\
Эй, за водкой, четвертную,\\
Да зайди еще в съестную\\
За печенкой!\\
Словно море волновалось,\\
Пиво с водкою мешалось ---\\
Песни, пляска\\
Продолжались до рассвета\\
И остались не допеты ---\\
Вышла встряска.\\
А хозяин за стеною\\
Говорит, шутя, с женою,\\
Прозимуют.\\
Вот пропьется как до нитки\\
Неминуют сделать скидки,\\
Неминуют.
\end{patverse*}
\end{verse}
1904

% Page 156  
\poemtitle{Поздней осенью в лесу}
\settowidth{\versewidth}{\vinНе веет он волшебной сказкой}
\begin{verse}[\versewidth]
% Порядок строк поменян в соответствии с авторской правкой
Листы опали, даль бледнеет,\\
И заковало речку льдом,\\
И солнце красное не греет,\\
В лесу угрюмом и пустом;\\
\begingroup
\leftskip1em
\rightskip\leftskip
Не веет он волшебной сказкой\\
На думы грустные мои,\\
Туманя взор седою краской\\
Осин и тлеющей хвои.\\
\endgroup
Угасла жизнь, лишь мышь листвою\\
Шурша, нарушит тихий сон,\\
Да скрипнет дерево порою,\\
И эхо даст протяжный стон.
\end{verse}
1904

\poemtitle{Ждет}
\settowidth{\versewidth}{Дождь стучит в окно худое}
\begin{verse}[\versewidth]
Дождь стучит в окно худое,\\
В щели злобно ветер воет;\\
Мужа ждет жена бедняжка,\\
Ждет за-полночь, --- сердце ноет,\\
Давит горе грудь больную;\\
Она знает --- муж напьется\\
И за прежние упреки\\
С грубой бранью придерется.\\
А ведь жалко, если сгинет,\\
Страшно бедно, ночь не спится,\\
Ветер чуть лишь ставней хлопнет,\\
Ей уж чудится, стучится, ---\\
Много, много раз подходит\\
Она к тусклому окошку,\\
Отойдет тогда лишь только,\\
Как заплачет в люльке крошка.
\end{verse}
1904

\poemtitle{Горе}
\settowidth{\versewidth}{От тяжелых дум, от заботушки}
\begin{verse}[\versewidth]
От тяжелых дум, от заботушки,\\
Голова моя разболелася,\\
Всю я ноченьку не сомкнул очей\\
Вот и зорюшка заалелася,\\
Солнце красное из-за рощицы\\
Дню веселому улыбается.\\
По селу народ на работушку\\
С силой свежею поднимается;\\
Только я лежу словно связанный,\\
Щеки бледные провалилися,\\
Надо мной жена с ребятишками\\
У постелюшки наклонилися,\\
Как цветы в мороз поздней осенью;\\
Увядаючи осыпаются,\\
Чуют горюшко мои родные,\\
Плачут бедные, надрываются.\\
Пронеси Господь немощь черную\\
По полям, лугам во дремучий бор,\\
Сохрани детям Ты кормилица,\\
Не тумань слезой их невинный взор,\\
Без отца пойдут по чужим людям,\\
Скоро с злобою повстречаются,\\
Надорвут в борьбе свою силушку,\\
Как былиночки закачаются.
\end{verse}
1904

\poemtitle{Весна}
\settowidth{\versewidth}{Блестит воскресший храм природы}
\begin{verse}[\versewidth]
\indentpattern{01001}
\begin{patverse*}
Блестит воскресший храм природы\\
И даль лазурна и ясна,\\
Целует землю луч свободы,\\
В реках струятся шумно воды,\\
Ликует весело весна,\\
Лаская нежно манит взоры\\
И льет приветливо тепло.\\
В лесах раздались причек хоры,\\
Оттаяли древьев коры,\\
Смолой душистой понесло.\\
Поля родные зеленеют,\\
Пестреют ранние цветы,\\
Высоко жаворонки реют\\
И песней радость всюду веют\\
И будят светлые мечты.
\end{patverse*}
\end{verse}
1904

\poemtitle{Тоска}
\settowidth{\versewidth}{Перестань безотвязная дума}
\begin{verse}[\versewidth]
Перестань, безотвязная дума,\\
Мое сердце тоскою сушить,\\
Или вечно ты мрачно, угрюмо\\
Будешь душу больную крушить,\\!

Чтобы горе слезой безысходной\\
В звуки песни живой перелить\\
И во тьме над толпою холодно\\
Одиноко носиться и ныть.\\!

Не услышит мой голос усталый\\
За работой измученный брат,\\
Он заглохнет в груди исхудалой\\
И замрет от суровых преград.\\!

Так умолкни же, грустная дума,\\
Дай изнывшей душе отдохнуть,\\
Ах, зачем же упрямо угрюмо\\
Точишь сердце и слабую грудь?
\end{verse}
1904


\poemtitle{***}[Время отдыха настало]
\settowidth{\versewidth}{Время отдыха настало}
\begin{verse}[\versewidth]
Время отдыха настало ---\\
Трудный кончился денек,\\
Ныть сердечко перестало,\\
Вспыхнул жизни огонек.\\
Дети сыты, слава Богу,\\
На полати легли спать,\\
За окошком понемногу\\
Буря стала утихать.\
Словно ра в родной избушке,\\
Снова счастье и покой,\\
Уж давно постель, подушки\\
Взбиты нежною рукой.\\
Снова милой манят ласки,\\
Щечки пламенно горят,\\
Как у доброй феи в сказке,\\
Глазки радостью блестят.
\end{verse}
1904

\poemtitle{Думы у фабричного станка}
\settowidth{\versewidth}{С тех пор измученный недугом}
\begin{verse}[\versewidth]
Ребенком я расстался с плугом,\\
С тех пор, измученный недугом,\\
В фабричном тягостном труде\\
Заглох, задавленный в нужде.\\
А помню парнем был отважным,\\
Тепень я стал рабом продажным,\\
Все трешь, да пилишь день деньской\\
Под визг машин и стон людской.\\
С работы ли придешь голодный\\
В свой угол темный и холодный\\
И там все тот же ад, содом\\
И стынет, стынет сердце льдом.\\
Жена тоскует, плачут дети,\\
За что ж наказан я на свете?\\
Ужели тем я согрешил,\\
Что честь и правду я любил?
\end{verse}
1904

\poemtitle{Скорбь}
\settowidth{\versewidth}{Опять звучат мои рыданья}
\begin{verse}[\versewidth]
Опять звучат мои рыданья,\\
Опять слеза туманит взор,\\
Кипят несносные страданья\\
И душу жмет немой укор\\!

За миг один коварной славы\\
Презренную златую нить\\
Мне чашу, полную отравы,\\
Судьба дает порою пить.\\!

Напрасно все мои усилья\\
Сорвать с души тяжелый гнет ---\\
Оковы дерзкого насилья\\
Сковали цепью мой полет.\\!

Я раб от самого рожденья ---\\
Нужда и скорбь даны в удел,\\
Смирись душа без рассужденья\\
Ведь негде взять чего хотел.
\end{verse}
1904

\poemtitle{Ранний снег}
\settowidth{\versewidth}{Дрогнула речка и в страхе застыла}
\begin{verse}[\versewidth]
\begin{altverse}
Саваном белым природа одела\\
Полные жизни родные поля,\\
Прелесть деревьев завять не успела,\\
В зелени пышной уснула земля.\\
Дрогнула речка и в страхе застыла,\\
Скрылся за тучею ласковый феб;\\
Вьюга над полем уныло завыла,\\
Снегом заносит неубранный хлеб,\\
Грусно хоронит богатство людское\\
В поле добытое тяжким трудом.\\
Горе-то, горе беднягам какое\\
Слышится, стонет метель за окном.
\end{altverse}
\end{verse}
1904


\poemtitle{Мечта}
\settowidth{\versewidth}{За всех молиться здесь, в цветах}
\begin{verse}[\versewidth]
\indentpattern{01}
\begin{patverse*}
\vinМогуч стоит зеленый бор,\\
Блестит весеннею красою,\\
Шатром раскинув свой убор,\\
Зовет под тень вздохнуть душою,\\
Забыть суровую печаль,\\
Забыть вражду, забыть страданья,\\
Взглянуть в синеющую даль\\
Спокойным взором упованья\\
С надеждой светлою в глазах,\\
В грядущем счастья и свободы\\
За всех молиться здесь, в цветах,\\
Среди ликующей природы.
\end{patverse*}
\end{verse}
1904




\poemtitle{Заглохшие мечты}
\settowidth{\versewidth}{Заглохли в душе моей вольные мысли}
\begin{verse}[\versewidth]
\begin{altverse}
Заглохли в душе моей вольные мысли\\
И людям я светлой мечты не сказал,\\
Нахмурились думы, как тучи нависли,\\
Затмили желанный святой идеал,\\
А грудь моя ноет и стих мой угрюмый\\
Томит мое сердце, застывшее льдом;\\
Явлюсь ли к народу я с грустною думой ---\\
Презренные взгляды я вижу кругом,\\
Один без участья друзей и совета\\
С тоскою встречаю шипы вялых роз\\
И стонет больная душа без привета\\
Не выливши горя, не выплакав слез.
\end{altverse}
\end{verse}
1904



\poemtitle{Не поется песня}
\settowidth{\versewidth}{Мой голос смолк с печальным вздохом}
\begin{verse}[\versewidth]
Не льется песнь живым потоком\\
Среди родимой стороны,\\
Мой голос смолк с печальным вздохом\\
И грустно замер звон струны.\\!

Замолк аккорд созвучи строных\\
И стынет жар в груди моей,\\
Душа болит, ей нет покойных\\
Отрадных дум и мирных дней,\\!

Бушуют с злым коварным роком\\
Порывы добрые любви;\\
Зачем не мог я быть пророком,\\
Зачем заглохло все в груди?..\\!

О, скоро ль светлый луч проглянет,\\
Весна воскреснет предо мной,\\
Священный мир в душе настанет\\
И голос вновь раздастся мой\\!

За всех судьбою угнетенных,\\
Сирот, страдающих в глуши,\\
Я вылью в звуках вдохновенных\\
Печальны гнев своей души.

\end{verse}
1904

%130
\poemtitle{Грусть по воле}
\settowidth{\versewidth}{\vinДни счастливые миновалися,}
\begin{verse}[\versewidth]
Затуманилась зорька алая,\\
Туча темная наклонилася,\\
Голова моя бесталанная\\
Тяжело на грудь опустилася.\\!
\begingroup
\leftskip1em
\rightskip\leftskip
    Силы крепкия  надорвалися,\\
    Сердце бедное истомилося,\\
    Дни счастливые миновалися,\\
    Злое горюшко появилося.\\!
\endgroup

%Где-ж ты делася, воля-вольная,\\
%Где-ж ты с радостью затерялася?\\
Где ж ты делася, воля-вольная,\\
Где ж ты с радостью затерялася?\\
Знать, на век в удел жизнь бездольная,\\
На борьбу с нуждой мне досталася.
\end{verse}

\poemtitle{В осенний дождь}
\settowidth{\versewidth}{Осенней пасмурной порою}
\begin{verse}[\versewidth]
Осенней пасмурной порою\\
Холодный дождик моросил,\\
К большой деревне под горою\\
Старик с мальчишкой подходил.\\!

Поблекший взор, горбатый, хилый,\\
С сумою рваной на плечах\\
Старик запел и с ним унылый\\
Дитя с слезами на глазах.\\!

Остался мальчик сиротою,\\
Не помнит матери, отца,\\
И вот удел его --- с сумою\\
Водить несчастного слепца.
\end{verse}
1904

\poemtitle{Осень}
\settowidth{\versewidth}{Вот и осень, все созрело}
\begin{verse}[\versewidth]
\indentpattern{01011}
\begin{patverse*}
Вот и осень, все созрело\\
В огородах и полях,\\
Гусей стадо пролетело,\\
Слышны крики в облаках\\
И настала тишь в лесах.\\
На березах пожелтели\\
И посыпались листы,\\
Паутины полетели.\\
Прежде где цвели цветы ---\\
Луга мертвы и пусты.
\end{patverse*}
\end{verse}
1904



\poemtitle{}
\settowidth{\versewidth}{Довольно спать, рассвет мерцает,}
\begin{verse}[\versewidth]
Довольно спать, рассвет мерцает,\\
Зажглась желанная заря;\\
Надежда дружбы расцветает\\
Любовью пламенно горя.\\!

Сомненье прочь! давно мы знали\\
Друг друга вместе за ярмом,\\
В одних цепях нас с детства гнали,\\
Одним безжалостным кнутом.\\!

Вставай, друзья, пора за дело,\\
На помощь правду призовем,\\
Стеною станем дружно, смело\\
Рука с рукой вперед пойдем.\\!

Пора, друзья, плотней народу
Сродниться внутренним крестом,\\
Стряхнем ярмо, вернем свободы,\\
С любовью данную Христом.
\end{verse}
1904


\poemtitle{Горемычная доля}
\settowidth{\versewidth}{Эх ты долюшка}
\begin{verse}[\versewidth]
\begin{altverse}
Эх ты долюшка\\
Горемычная,\\
Что судьба-ль моя\\
Безталанная,\\
Как былиночку\\
Ты подрезала.\\
Глупо сделали\\
Отец с матерью ---\\
За немилого\\
Замуж выдали,\\
За ревнивого\\
Мужа старого.\\
Ох, тоска змея\\
Подколодная,\\
Точно камнем грудь\\
Давить белую;\\
Не услышу я,\\
Молодешенька,\\
Речи ласковой\\
И приветливой,\\
Чем подруженьки\\
Мои хвалятся,\\
Живут в бедности,\\
Наслаждаются.\\
Ох, как грустно мне\\
Без мила дружка;\\
Схоронила я\\
Свою молодость\\
В куче золота\\
У богатого;\\
Он прелстил моих\\
Отца с матерью\\
Золотой казной,\\
Своей славою.
\end{altverse}
\end{verse}
1904



\poemtitle[<<Утопил ты свою долю...>>]{***}
\settowidth{\versewidth}{Утопил ты свою долю}
\begin{verse}[\versewidth]
Утопил ты свою долю\\
В чарке пьяного вина,\\
Стали в тягость и неволю\\
Тебе дети и жена,\\
Бросил бедных ты, покинул\\
Через пьянство в нищете,\\
И армяк последний скинул ---\\
Пропил в грязном кабаке,\\
Получил взамен худое\\
Ты за прежни свой наряд,\\
Лицо желтое, больное,\\
И тупой нахальный взгляд.
\end{verse}
1904




\poemtitle{За правду}
\settowidth{\versewidth}{Чтоб был не страшен червь могильный}
\begin{verse}[\versewidth]
\begin{altverse}
О, подлость ты людская злая,\\
Зачем правду отравляешь?\\
Блажена цель ее святая,\\
А ты гонишь, презираешь.\\
Но нет! Не будешь ты здесь вечно\\
Над миром злобствовать, язвить,\\
Одна лишь правда бесконечно\\
Она в людях должна лишь жить,\\
Ее дал тот, кто духом сильны,\\
Чтоб мир и братство здесь узреть,\\
Чтоб был не страшен червь могильный,\\
Завет дал в правде умереть.
\end{altverse}
\end{verse}
1904


\poemtitle{Страдная пора}
\settowidth{\versewidth}{Вокруг уже спелая рожь золотится}
\begin{verse}[\versewidth]
Вокруг уже спелая рожь золотится\\
И страдная в поле настала пора,\\
Вот вечер и солнце за горы садится,\\
Пахнуло прохладой и спала жара.\\!

Леса и долины покрылись росою\\
И, влажные, дремлют родные поля;\\
Крестьянин усталый идет полосою,\\
Суровую долю угрюмо кляня.\\!

За ним торопливо девчонка шагает ---\\
На худеньком теле лохмотья одни,\\
Она, как и старшие, счастья не знает,\\
У ней, как у старших, суровые дни.\\!

Отцу, изнуренному трудной работой,\\
Уж некогда стало ее приласкать,\\
Иною встревожен тяжелой заботой ---\\
Лежит дома хворая девочки мать.\\!

Согнула работа, свалилась подруга,\\
С кем бедность лихую делить он привык,\\
Страдалица вечно не знала досуга ---\\
Нужда, ребятишки, их стоны да крик;\\!

И хлеб черствый пища, а лучшей не знали,\\
Но вместе не страшен был голод с нуждой,\\
С зарей на работе косили и жали,\\
Покорно мирились с коварной судьбой,\\!

И в горьком раздумье мужик исхудалый\\
Шел, голову мрачно склонивши на грудь,\\
Уж еле тащится ребенок усталый,\\
Темней и темней становится путь.\\!

Чу, полночь глухую в селе прозвонили,\\
Раздалось уныло в ушах мужика,\\
А где-то за лесом лишь волки завыли,\\
И ночь опустилась темна, глубока.
\end{verse}
1904


\poemtitle{Не губи же свою долю}
\settowidth{\versewidth}{Ах, зачем, зачем стремишься}
\begin{verse}[\versewidth]
Ах, зачем, зачем стремишься\\
К вину, корню ты порока ---\\
Не минуешь злого рока,\\
И в разврат ты погрузишься.\\!

Будешь жертвой приговора\\
Ты постыдных увлечений\\
И болезней и мучений\\
И публичного позора\\!

Эх, не все еще пропало,\\
Не губи же свою долю,\\
Вспомни ты святую волю,\\
Вырви с корнем это жало.
\end{verse}
1904


\poemtitle{За чужою волей}
\settowidth{\versewidth}{Целый век в заботе}
\begin{verse}[\versewidth]
\begin{altverse}
Жизнь мастеровая ---\\
Целый век в заботе,\\
За чужою волей,\\
В тягостной работа;\\
Отравленный воздух\\
Потом адской муки\\
Ты глотаешь бедный\\
И мозолишь руки.\\
Не имеешь мысли,\\
Не живешь душою,\\
А поник уныло\\
Рабской головою.\\
Плечи сводишь вместе,\\
Гнешь дугою спину,\\
И в груди ты носишь\\
Горькую кручину.\\
А твое знакомство\\
С тяжкою нуждою\\
Знать за труд в награду\\
Послано судьбою.\\
И за все под старость ---\\
Голод и скитанье,\\
Силы нет работать,\\
Просишь подаянье.\\
Век так бродишь тенью,\\
Слез накопишь море,\\
Да заплакать стыдно, ---\\
Не поймут ведь горе.
\end{altverse}
\end{verse}
1904



\poemtitle{Пробудися, душа}
\settowidth{\versewidth}{Из-под снежных оков уж ручьи загремели}
\begin{verse}[\versewidth]
\indentpattern{00101}
\begin{patverse*}
Из-под снежных оков уж рчьи загремели,\\
Высоко в облаках жаворонки запели;\\
Вот краса и любовь и весенние дни.\\
Веселее деревья в лесах зашумели,\\
Только мрачны, как ночь, мои думы одни.\\
Не воскрес только я обветшалой душою,\\
Вьется черная тень над моей головою,\\
Днем и ночью томит, застилает глаза.\\
За позором страстей, как за тьмою густою,\\
Меркнет радости луч, гаснет взор в небеса,\\
А мир полон любви, полон жизни весною,\\
Пробудися, душа, этой чудной порою,\\
Распахни же ты грудь и объятья свои.\\
И свободней вздохни обновленной душою,\\
Нам привет за рекою поют соловьи.
\end{patverse*}
\end{verse}
1904


\poemtitle{Бремя давит}
\settowidth{\versewidth}{Весело летом в деревне живется}
\begin{verse}[\versewidth]
\indentpattern{0011}
\begin{patverse*}
Весело летом в деревне живется,\\
Песня и говор до ночи несется;\\
С мягких лугов и родимых полей,\\
Громко им вторит в лесу соловей.\\
В зелени тонут веселые звуки\\
Машут косою усталые руки\\
С горькой нуждою лихого борца,\\
Сколько отваги в движеньи лица,\\
Веры глубокой в семейное счастье,\\
Будет ли доля святого участья\\
Труженик бедный на помощь тебе?\\
Или сгниешь покорно в борьбе,\\
С горечью хлеба кусок добывая,\\
Молодость, силу в груди надрывая.\\
Выльешь ли в песне весело порой\\
Тяжкое бремя, что давит горой.
\end{patverse*}
\end{verse}
1904


\poemtitle{В сознании}
\settowidth{\versewidth}{Опять тяжелый бред похмелья}
\begin{verse}[\versewidth]
\begin{altverse}
Опять тяжелый бред похмелья\\
Терзает злобною тоской,\\
На миг безумного веселья\\
Сменял здоровье и покой.\\
Дрожу, лохмотьями покрыты,\\
На грязной лавке под окном,\\
Графин уж пуст, стакан разбитый,\\
Ночник погас, темно кругом;\\
Тоска в избе, объятой мглою,\\
Чу, плач и вздохи на полу.\\
Скандал знать был опять с женою,\\
Опять, да, вспомниль, там в углу.\\
Зачем? За что? Небось упреки:\\
Я изверг, пьяница, злодей!\\
А там долги, уплаты, сроки,\\
Совсем души нет у людей...\\
Хорош и я! Да сыты ль дети?\\
Да был ли хлеб вчера у них?\\
Я их хотел лишить на свете\\
И ласк родимых дорогих,\\
И той любви, чем жизнь согрета,\\
Чем можно цвесть им и расти.\\
Нет, кончил пить! Я жду привета,\\
Жена, голубушка, прости...
\end{altverse}
\end{verse}
1904

\poemtitle{В Страстную седмицу}
\settowidth{\versewidth}{Пред ликом светлым я взываю}
\begin{verse}[\versewidth]
\begin{altverse}
К Тебе, о Боже, прибегаю,\\
Пошли мне милости своей,\\
Пред ликом светлым я взываю,\\
Душой страдающей моей,\\
Простри ты крепкую десницу,\\
Мой дух мятежный усмири,\\
О Боже, в Страстную седмицу\\
Святою мыслью осени;\\
Вонми Ты грешному моленью,\\
Своим покровом огради,\\
Предай, что мерзостно забвенью,\\
Дух целомудрия пошли.
\end{altverse}
\end{verse}
1904


\poemtitle{Желанный луч}
\settowidth{\versewidth}{Прошло тоскливое ненастье}
\begin{verse}[\versewidth]
Прошло тоскливое ненастье,\\
Глянуло солнце из-за гор,\\
И сердцу бедному о счастье\\
Опять твердит усталый взор.\\!

Воскресла жизнь, воскресни радость,\\
Раздались трели из кустов,\\
Уж в грудь с росою льется сладость,\\
Медвяных, бархатных цветов.\\!

Про миг любви, красу природы\\
Лепечет тихо стройный лес,\\
Давно желанный лучь свободы\\
Блестнул мне радостно с небес.
\end{verse}
1904

\poemtitle{}
\settowidth{\versewidth}{}
\begin{verse}[\versewidth]
\indentpattern{}
\begin{patverse*}
\end{patverse*}
\end{verse}
1904


%273

\poemtitle{Просьба}
\settowidth{\versewidth}{Боже правый! Дай мне силы}
\begin{verse}[\versewidth]
\begin{altverse}
Боже правый! Дай мне силы\\
    Бремя тяжкое нести,\\
Дай мне время до могилы\\
    Душу правдою спасти;\\!

Разреши мое сомненье,\\
    Дух мятежный обнови\\
И святое вдохновенье\\
    На добро благослови, ---\\!

Чтоб согреть любовью брата,\\
    Мрак невежды разогнать,\\
И за зло добром в отплату,\\
    Руку помощи подать.
\end{altverse}
\end{verse}
Впервые в 1904 г.


%%%%%%%%%%%%%%%%%%%%%%%%%%%%%%%%%
%%%%%%%%%%%%%%%%%%%%%%%%%%%%%%%%%
%% Книга 1910 года <<Из тьмы>> %%
%%%%%%%%%%%%%%%%%%%%%%%%%%%%%%%%%
%%%%%%%%%%%%%%%%%%%%%%%%%%%%%%%%%


%108
\poemtitle{Муза}
\begin{verse}[\versewidth]
\indentpattern{10}
\begin{patverse*}
\vinО, Муза скорбная моя! ---\\
Взмахни могучими крылами!\\
Лети в родимые поля,\\
И пой над кровлей и плетнями, ---\\
%Где семьижалких бедняков\\
Где семьи жалких бедняков\\
В лачугах рабски изнывают,\\
В ярме заржавленных оков\\
Свой век тяжелый проживают\ldotse\\
Судьбы позорной произвол\\
Для них связал искусно сети,\\
Чтоб в них бедняк смиренно шел,\\
И шли за ним покорно дети;\\
Чтоб кровью сытого питал,\\
%А сам, --- и дети голодали\ldotst\\
А сам и дети голодали\ldotst\\
%И чтоб без ропота страдал,\\
И, чтоб без ропота страдал,\\
%Блаженства рая обащали\ldotst\\
Блаженства рая обещали\ldotst\\
Тот рай, что мертвым отдают\\
Жрецы в минуту погребенья,\\
А сами с бедного дерут\\
Последний грош без сожаленья;\\
Чтоб сладко жить, спокойно спать,\\
Толпа льстецов, душой отсталых,\\
Готова крест последний взять\\
С рабов голодных и усталых\ldotst\\
Как стая злобных псов подчас,\\
Чтоб удержать свои основы,\\
Святое все они зараз\\
С лица земли стереть готовы\ldotst\\
О, Муза скорбная моя!\\
Взмахни бесшумными крылами, ---\\
Пусть песня горькая твоя\\
Звучит над кровными буграми\ldotse
\end{patverse*}
\end{verse}
1910



\poemtitle{Мужик}
\settowidth{\versewidth}{С детства к терпенью покорно привык.}
\begin{verse}[\versewidth]
\indentpattern{10}
\begin{patverse*}
\vinВечно суровую, рабскую долю\\
Тянет забитый, голодный мужик;\\
 С детства узнал он нужду и неволю,\\
С детства к терпенью покорно привык.\\
 Вечно идет он тяжелой дорогой,\\
Потом он землю чужую поит\ldotst\\
 Труженик честный, обиды он много\\
С болью на сердце на время таит\ldotst\\
 Грудью стоит он за землю родную, \\
Первая жертва в кровавом бою.\\
 Он ли не выстрадал долю иную,\\
Право на землю родную свою\ldotsq
\end{patverse*}
\end{verse}
1910


% 104
\poemtitle{Не ходи}
\settowidth{\versewidth}{Где не надо жизнью ложной}
\begin{verse}[\versewidth]
\begin{altverse}
Не ходи, нужда, за мною,\\
    По моим следам ---\\
Снег растает, я весною\\
    Все тебе отдам:\\
Закопченную лачугу\\
    И чердак пустой,\\
Где зимою слушал вьюгу\\
    За трубой худой.\\
Все возьми --- мои онучи,\\
    Рваный мой озям\ldotst\\
Я пойду туда, за кручи,\\
    К солнцу, к небесам\ldotst\\
Где свободно будет можно\\
    Воздухом дышать,\\
Где не надо жизнью ложной\\
    Душу омрачать\ldotst\\
\end{altverse}
\end{verse}
1910

\poemtitle{Борцу}
\settowidth{\versewidth}{Тяжел твой путь, пройденный путь}
\begin{verse}[\versewidth]
\indentpattern{10}
\begin{patverse*}
\vinТяжел твой путь, пройденный путь...\\
И дальше темная дорога\\
В поту кровавом хлеб добудь!\\
Лачуга тесна и убога ---\\
Землицы клок пусто лежит\\
Под зноем бурной непогоды...\\
А даль еще страшней грозит,\\
Суля больной душе невзгоды...\\
А там и старость... Злой кошмар\\
Зажмет свирепо грудь тисками,\\
В крови потушит юный жар,\\
Покроет кудри сединами...\\
И скорбь оставит яркий след,\\
Страданья след на лбу высоком,\\
Брьбы упорной прежних лет\\
С тоскою в сердце одиноком...\\
Один ты был тогда, один,\\
Когда твои слагались думы...\\
Как страж, как грозный палладин,\\
Ты в темноте стоял угрюмо...\\
И рвался с яростью разбить\\
Оковы ржавые насилья...\\
И думал муки позабыть,\\
Но были слабы все усилья...\\
От мук не мог ты убежать\\
Ни в город шумный, ни в пустыню,\\
И, больше прежнего страдать\\
Пришлось вчера, страдать и ныне...\\
И, может, целый ряд ночей,\\
Терзаться будешь в адской муке\\
Среди коварных палачей\\
С свободой милою в разлуке...
\end{patverse*}
\end{verse}
1910

\poemtitle{Что же дало им терпенье?}
\settowidth{\versewidth}{Низко березы к земле наклоняет}
\begin{verse}[\versewidth]
\begin{altverse}
Низко березы к земле наклоняет\\
Ветер, сурово срывая листы...\\
Душу больную тоска надрывает,\\
Давят тяжелым кошмаром мечты...\\
Словно унылое, мрачное море,\\
Жизнь замирает родимых полей...\\
С холодом снова голодное горе\\
Крадется в избы забитых людей...\\
Бедные, сколько труда положили,\\
Сил  закопали на пашне родной!\\
Все, что собрали, в оброк заплатили,\\
Все до последней копейки одной...\\
Где же награда за тяжкие муки?..\\
Только спина изогнулась дугой!..\\
Что же добыли мозольные руки,\\
Ради любимой семьи дорогой?..\\
С верой глубокой просили у неба\\
Ведро в ненастье и дождик в жары...\\
Хлеба нам, только насущного хлеба,\\
Чтобы хватило до вешней поры!..\\
Что же дало им святое терпенье,\\
Что же им сулит слепая судьба?..\\
Даст ли их горю когда утешенье,\\
Или же вечно с нуждою борьба?..
\end{altverse}
\end{verse}
1910


\poemtitle{Тяжело бедняку}

% 8
%(стр 102)
\settowidth{\versewidth}{А с суровой нуждой тяжелее бороться!}
\begin{verse}[\versewidth]
\indentpattern{10}
\begin{patverse*}
\vinТяжело бедняку за сохою ходить.\\
А с суровой нуждой тяжелее бороться!\\
От зари до зари в поле хлеб молотить,\\
А детей накормить и одеть --- как придется.\\!

Что и даст урожай, --- за долги все идет,\\
И с семьею бедняк остается голодный,\\
И загложет тоска, --- больно сердце сожмет, ---\\
Как подует зимой с поля ветер холодный...\\!

В нетопленой избе заискрится мороз,\\
Захохочет метель над раскрытой поветью,\\
Ребятишки дрожат, изнывают от слез,\\
Злобный голод, как зверь, воцарится над клетью.\\!

Сколько ж сил тогда надо ему мужику,\\
Чтобы вынести горе семьи и мученье?..\\
И придут ли на помощь к нему, бедняку,\\
Приласкает ли кто за святое мученье?..
\end{patverse*}
\end{verse}


\poemtitle{Думы}
%(стр. 116)
\settowidth{\versewidth}{Думушки – думы! Вы радость и мука,}
\begin{verse}[\versewidth]
\begin{altverse}
Думушки-думы! Вы радость и мука,\\
Днем лучезарным и в долгую ночь,\\
В тьме непроглядной вы – свет и наука, ---\\
Тучи суровые гоните прочь...\\!

С вами, о, думы, я крепко сдружился,\\
С вами я горе и радость делю,\\
С вами я песням родным научился, ---\\
В сладком забвенье отрадно пою...\\!

Резвую ль юность когда вспоминаю,\\
Всюду вы, светлые думы, со мной!\\
Или в тоске безутешно страдаю, ---\\
Вы, только вы мне даете покой...
\end{altverse}
\end{verse}
1910


% Page 109
\poemtitle{Нет, не хочу\ldotst}
\settowidth{\versewidth}{Правда ль, что рабство дано мне в удел?}
\begin{verse}[\versewidth]
\begin{altverse}
%Надо-ль коварной судьбе покорятся\ldotsq\\
Надо ль коварной судьбе покоряться\ldotsq\\
    Правда ль, что рабство дано мне в удел?\\
Нет\ldotse Не хочу больше бури бояться,\\
    %Вырву тот страх, чем я с детства болел. ---\\!
	Вырву тот страх, чем я с детства болел.\\!
Высушил грудь, и усталые руки\ldotst\\
    Жизнь не хочу я в цепях потерять,\\
Вынес я много обмана и муки, ---\\
    Ложь вековую пора мне понять\ldotst\\!

Сбросить пора мне оковы насилья,\\
    Слепо мириться теперь не хочу,\\
Я разверну угнетенные крылья,\\
    Вольною птицею ввысь полечу\ldotst
\end{altverse}
\end{verse}
1910


%112
% Проверено по книге
\poemtitle{Больной век}
\settowidth{\versewidth}{В струях желанный дар свободы}
\begin{verse}[\versewidth]
\begin{altverse}
Пройдет зима, и шумно воды\\
    С полей родимых потекут,\\
В струях желанный дар свободы\\
    Лугам и рощам принесут;\\
Спадут оковы снеговые\\
    С ручьев гремучих, с быстрых рек.\\
И снова долы вековые\\
    В красе увидит человек\ldotst\\
Но, чувство робкое подскажет:\\
%    --- <<Ах! --- Светлый мир, не для меня.>>\\
    <<Ах! --- Светлый мир, не для меня.>>\\
Душа в тоске на зло укажет,\\
    Суровый век больной кляня\ldotst\\
Цветет природа молодая,\\
    И блещет солнце в небесах\ldotst\\
%А, мы живем, изнемогая,\\
А мы живем, изнемогая,\\
    В нужде, проклятьях, и слезах\ldotst\\
Рабами нас рабы родили,\\
    Без духа пламенной борьбы,\\
И с детства ложный страх вселили\\
    Перед величием судьбы\ldotst\\
В цепях неволи век томиться,\\
    Всю жизнь безропотно страдать,\\
%Слепыми быть, и лжи молиться,\\
Слепыми быть и лжи молиться,\\
    И, не расцветши --- увядать\ldotst
\end{altverse}
\end{verse}
1910


\poemtitle{Сон}
\indentpattern{001101011010110101101}

\begin{verse}
\begin{patverse}
Мне снилось чудо из чудес:\\
Среди прозревшего народа\\
Христос воистину воскрес,\\
И звукам радостным с небес\\
Внимает чуткая природа...\\
Повеял всюду тихий мир,\\
Где прежде бурными волнами\\
Кипел войны кровавый пир,\\
Лежит разбитый зла кумир\\
В грязи потоптанный ногами...\\
А в небе светлая заря\\
Сполохом розовым играет...\\
Блестят румяные поля,\\
Проснувшись, тучная земля,\\
В цветах любви благоухает...\\
Не слышен больше тяжкий стон,\\
Народ свободными устами\\
Проклял цепей ужасных звон...\\
--- Ах, Боже, как был сладок сон,\\
Душе навеянный мечтами...
\end{patverse}
\end{verse}
1910


%%%%%%%%%%%%%%%%%%%%%%%%%%%%%%%%%%%%%
%%%%%%%%%%%%%%%%%%%%%%%%%%%%%%%%%%%%%
%%%%%%%%%%%%%%%%%%%%%%%%%%%%%%%%%%%%%


\poemtitle{Полет мысли}
\settowidth{\versewidth}{В живой любви тепла и света}
\begin{verse}[\versewidth]
\begin{altverse}
В живой любви тепла и света\\
Предела нет, и нет границ\\
Тоске и радости поэта, --- \\
Они на тысячах страниц\ldots\\
Слезами вписаны, пропеты, \\
При блеске розовых зарниц\ldots\\
% Правописание надо б
% http://www.evartist.narod.ru/text1/38.htm
Желаньям надо б утомиться,\\
Но жажда все еще живет,\\
И мысль летит, --- все дальше мчится\\
Под бурный шум, в круговорот,\\
К разгадке смелая стремится\\
% Нужна ли здесь запятая?
И --- неразгаданной умрет\ldots
\end{altverse}
\end{verse}


\newpage
%\vspace*{0cm}

\poemtitle{Палисадник}
\settowidth{\versewidth}{Мой любимый полисадник}
\begin{verse}[\versewidth]
\begin{altverse}
Мой любимый палисадник\\
Светом солнечным залит,\\
Частокол его старинный\\
Алым хмелем перевит;\\
Распустилася малина,\\
Вишни, яблони цветут,\\
И смородина меж ними\\
Разметалась там и тут;\\
Пчелы вьются над кустами,\\
Собирая сладкий мед,\\
И пестреет над лужайкой\\
%Мотыльковый хоровод,\\
Мотыльковый хоровод.\\
А под вербой, ближе к тыну, ---\\
Заповедная скамья,\\
Где в тени, в часы досуга,\\
Отдыхает вся семья\ldots\\
Здесь надежды воскресают\\
И безмолвствует вражда,\\
Забываются обиды,\\
Гнет, и горе, и нужда.
\end{altverse}
\end{verse}

\newpage
\vspace*{0cm}


% Сверено с оригиналом. Егор, Ольга 19/12/15 


\poemtitle{В погоне за юностью}
\settowidth{\versewidth}{О, юность, юность, погоди\ldotse}
\begin{verse}[\versewidth]
О, юность, юность, погоди\ldotse\\
Моя звезда не закатилась,\\
Она позднее засветилась, ---\\
% Строка поправлена по книге с авторскими правками 
Надежда-радость впереди\ldotse\\
% Строка поправлена по книге с авторскими правками 
Еще цветы не отцвели,\\
Мороз не выжал аромата,\\
И красотою жизнь богата, ---\\
Лишь только жажду утоли\ldotst\\
Умерь свой бег, и дай скорей\\
Хоть миг желанного привета\ldotst\\
% По правке 
И поспешим с тобою к свету\\
% По правке 
Весенней молнии быстрей\ldotst
\end{verse}


\newpage
\vspace*{0cm}

\newpage
\vspace*{-0.5cm}
 
\poemtitle{Размышление}
\settowidth{\versewidth}{А ты? Ты жалкий, ты ничтожный,}
\begin{verse}[\versewidth]
\begin{altverse}
К чему о правде рассужденья,\\ 
%   Было:
%   Пророком что-ли хочешь быть\ldotse\\
Пророком что ли хочешь быть\ldotse\\
Оставь напрасные волненья, ---\\
Спокойно лучше в мире жить.\\!

Не трогай нервы, равнодушно\\
Смотри, где льется пот и кровь\ldotst\\
Тебе ли быть великодушным?!\\
Их ждет борьба, а не любовь.\\!

А ты? Ты жалкий, ты ничтожный,\\
Тебе ль страданье превозмочь?!\\
Тебя пугает шум тревожный,\\
Гони мечты, гони их прочь!\\!
\end{altverse}
\vin\ldots\ldots\ldots\ldots\ldots\ldots\ldots\ldots\ldots\ldots\\!
\begin{altverse}
Нет, лжешь позорно, подлость злая,\\
И тем волнуешь ум сильней,\\
В неистовстве изнемогая,\\
Я с правдой выступлю смелей.\\!

Презренье брошу я кумиру,\\
И с ним упитанным жрецам;\\
Иль разобью во гневе лиру,\\
И, как червяк, --- погибну сам.
\end{altverse}
\end{verse}
Впервые напечатано в 1904 г., в 1917 с небольшими изменениями.

\newpage
\vspace*{0cm}



\poemtitle{Березы}
\settowidth{\versewidth}{Как был ты встречен с теплой лаской}
\begin{verse}[\versewidth]
\begin{altverse}
Люблю я белые березы,\\
Как близких, ласковых друзей\ldotst\\
На них росинки, словно слезы,\\
Чаруют прелестью своей\ldotst\\
%Дрожат листы и нежный шепот\\
Дрожат листы, и нежный шепот\\
О тайне вещей говорит:\\
<<Оставь, безумец, дерзкий ропот,\\
Твоя звезда еще горит;\\
Еще огни не угасали\\
Там, где витает мысль твоя,\\
От преждевременной печали\\
Уйди в небесные края\ldotst\\
И окрыленною мечтою\\
Простор далекий облети,\\
Отдайся грезам и покою\\
В волшебном, облачном пути.\\
% Сейчас делаем ветку с авторской пунктуацией, затем сделаем ветку под корректуру. Корректура нужна, потому как есть спорные пунктуационные моменты и нужен профессиональный корректор.
И, отдохнув душою, снова\\
Вернись к друзьям с запасом сил,\\
С любовью пламенного слова\\
Скажи, что в тайне пережил;\\
Утешь их вечной чудной сказкой\\
Лазурных солнечных высот,\\
Как был ты встречен с теплой лаской\\
%Было (в оригинале)
%Среди божественных красот>>..
%Стало (исправил) - Г.К.
Среди божественных красот...>>
\end{altverse}
\end{verse}


\newpage
\vspace*{0cm}


% Как думаешь, как сделать поиск в новой книге?
% Может, нужен индекс? - Список стихов по алфавиту.
% - Возможно ли установить даты написания стихов и отсортировать их по дате написания или по году написания? 


% Старое название - ``Весна проснулась'' - c 182
\poemtitle{Веснянка}
\settowidth{\versewidth}{Проснулась жизнь, любовью дышит,}
\begin{verse}[\versewidth]
\begin{altverse}
Проснулась жизнь, любовью дышит,\\
    По жилам кровь быстрей бежит\ldotst\\
Зеленый лес листвой колышет,\\
    Под пенье птиц вокруг шумит\ldotst\\
Стада пасутся\ldotst Вниз долины\\
    Ручей извилистый гремит,\\
Таясь в тени густой калины,\\
% Вот здесь вопрос, что поставить вместо этого авторского выбора пунктуации ..,
%    Веснянка юная стоит..,\\
     Веснянка юная стоит\ldotst\\
Прекрасный взгляд ее, лукавый,\\
    Скользит с улыбкой по ручью\ldotst\\
% Здесь дефис стоит после запятой. Ставить его или нет? А вдруг нет? Место мало, вот и поставил. Запятая с тире вполне себе употребляется в языке. Другой вопрос, уместно ли это в данном случае? - Это к редактору книги. Кто редактор книги. Не я. А кто? Это надо найти редактора и корректора. 
%Вдали, --- рыбак взошел на лавы,\\
Вдали --- рыбак взошел на лавы,\\
    И чинит легкую ладью\ldotst\\
На миг мелькнула грудью белой\\
    Краса веснянки в камышах,\\
%И, поплыла как лебедь, смело,\\
И поплыла, как лебедь, смело,\\
    Играя брызгами в волнах;\\
И лишь пестреет золотистый\\
    Наряд красавицы в лугах\ldotst\\
Склонился ландыш серебристый\\
    К фиалке бархатной в кустах\ldotst\\
Шепча любовно лепестками,\\
    Фиалка ловит страстный взгляд\ldotst\\
Сплелися нежно стебельками,\\
    И льют душистый аромат\ldotst
\end{altverse}
\end{verse}



\newpage
\vspace*{0cm}


\poemtitle{Зима}
\settowidth{\versewidth}{Злобно пылью снежной}
\begin{verse}[\versewidth]
\begin{altverse}
Все вокруг в природе\\
Сном могильным спит,\\
В стройном хороводе\\
Лес седой стоит,\\
На деревьях иней\\
Серебром блестит,\\
А вдали свод синий\\
Над землей висит\ldotst\\
И река под снегом\\
Замерла в тиши;\\
Лишь метель, набегом,\\
Прошумит в глуши,\\
Засвистит в прибрежных\\
Камышах сухих,\\
Злобно пылью снежной\\
Обдавая их.\\
И растут все выше\\
Берега кругом, --- \\
Расписные крыши\\
И за домом дом\ldotst
\end{altverse}
\end{verse}

\newpage
\vspace*{0cm}


% Страница 54
\poemtitle{Заколдованный лес}
\settowidth{\versewidth}{Лес поредел, валятся листья...}
\begin{verse}[\versewidth]
\begin{altverse}
Лес поредел, валятся листья,\\
Деревья хмурые стоят, ---\\
И жутко темными ветвями\\
В сыром тумане шелестят\ldotst\\
Семья лишайников желтеет,\\
Зубцы раскинув по земле;\\
Осины прелой пряный запах\\
Висит в дрожащей полумгле.\\
Заснуло все\ldotst Лишь дятел где-то\\
Порою, стукнет по сосне,\\
И снова тишь\ldotst И тяжко дышит\\
Лес заколдованный во сне\ldotst
\end{altverse}
\end{verse}

\newpage
\vspace*{0cm}


\newpage
\vspace*{0cm}

% P. 132
% В редакции 1904 года последнее четверостишье сделано первым
\poemtitle{Возвращение}
\settowidth{\versewidth}{Опять я вижу парк старинный}
\begin{verse}[\versewidth]
\begin{altverse}
Душа наполнилась мольбою\\
    И вера вспыхнула в груди,\\
Как будто реет предо мною\\
    Святое счастье впереди.\\!

Опять я вижу парк старинный\\
    И слышу лип кудрявых шум,\\
Листвы зеленой шепот дивный\\
    Волнует мой усталый ум.\\!

Гляжу я в даль, где речка вьется\\
    Между лощин по скату гор,\\
И чую --- сердце жадно рвется\\
    Туда --- на волю, на простор.
\end{altverse}
\end{verse}

\newpage
\vspace*{0cm}



\newpage
\vspace*{0cm}

% P 191
\poemtitle{Другу}
\settowidth{\versewidth}{\vinЧто пою про скорбь да муки\ldotst}
\begin{verse}[\versewidth]
\begin{altverse}
Не вини меня, друг милый,\\
    Что пою про скорбь да муки\ldotst\\
Про того мой стих унылый,\\
    У кого в мозолях руки\ldotst\\!

Стал бы петь, как плещет море,\\
    Ходят волны голубые,\\
Да повсюду стонет горе,\\
    Гибнут близкие, родные\ldotst\\!

%О лесах бы спел дремучих\\
О лесах бы спел дремучих,\\
    О красе лугов душистых,\\
Да вокруг так много жгучих\\
    Горьких слез, невинных, чистых\ldotst\\!

Спел бы я про вечер сонный,\\
    О метелях в зимний холод,\\
Сердце рвет неугомонный,\\
    Словно зверь, народный голод.\\!

Я бы спел про взгляд прекрасный,\\
    Чем подруга подарила,\\
Да сдавил кошмар ужасный,\\
    Грудь тоска мне защемила\ldotst
\end{altverse}
\end{verse}

\newpage
\vspace*{0cm}


\newpage
%\vspace*{0}

%162
\poemtitle{Береза}
\settowidth{\versewidth}{\vinСиротой стоишь;}
\begin{verse}[\versewidth]
\begin{altverse}
Ты березынька,\\
    Ты кудрявая,\\
Одинокая,\\
    Сиротой стоишь;\\
А вокруг тебя,\\
    В речке плавая,\\
Шепчет ласково\\
    Молодой камыш:\\
<<Где укроюсь я\\
    В непогожий день,\\
Если бурная\\
    Речка вспенится\ldotse\\
Кто возьмет меня\\
    Под густую тень,\\
Когда солнышком\\
    Буря сменится?>>\\
Вдруг березынька\\
    Белоствольная\\
%Расшумелася\\
Расшумелася,\\
    Раскачалася:\\
<<Ах, когда бы мне\\
    Воля вольная, ---\\
Всю бы жизнь с тобой\\
    Я шепталася.\\
Распустила бы\\
    Сень ветвистую, ---\\
Пусть над реченькой\\
    Дремлет гладь и тишь!\\
Целовала б я\\
    Золотистую\\
Красоту твою,\\
    Молодой камыш!..>>
\end{altverse}
\end{verse}

\newpage
\vspace*{0cm}

%136
\poemtitle{Люблю}
\settowidth{\versewidth}{\vinГлухой тропой, в степи бесплодной...}
\begin{verse}[\versewidth]
\indentpattern{1010}
\begin{patverse*}
\vinЛюблю я истину,\\
Люблю в добро святую веру,\\
    И не поддамся я\\
В обман ханже и лицемеру.\\!

    Да будет крест на век\\
Моей звездою путеводной!\\
    С ним, не страшась, пойду\\
Глухой тропой, в степи бесплодной\ldotst\\!

    В душе своей создам\\
Я светлый рай --- жилище Бога,\\
    И будет счастьем мне\\
В лучах любви к добру дорога\ldotst
\end{patverse*}
\end{verse}

\newpage
\vspace*{0cm}

%211
\poemtitle{В дороге}
\settowidth{\versewidth}{\vin\vinЧерез мост с дровами воз}
\begin{verse}[\versewidth]
\indentpattern{01202020202020202020}
\begin{patverse}
Холод\ldotst Пасмурно в лесу\ldotst\\
   Лист увядший на дороге\\
      %Прилипает к колесу\\
	  Прилипает к колесу,\\
И гремят по корням дроги\ldotst\\
      Через мост с дровами воз\\
Еле тянет лошаденка,\\
      Как и прежде, так же бос,\\
Мужичок в худой шубенке\\
      Тихо за возом идет,\\
Бледный, с впалыми щеками,\\
      На лице все тот-же гнет,\\
Не смываемый веками\ldotst\\
      С ним и голод, с ним и страх\\
Идут рядом неразлучно\ldotst\\
      Яд у бедного в устах,\\
А на сердце скучно, скучно\ldotst\\
      Темнота\ldotst и, нем язык\\
За себя сказать не может,\\
      Мужичок терпеть привык,\\
А судьба-змея все гложет.
\end{patverse}
\end{verse}

\newpage
\vspace*{0cm}


%%% Откуда это??????? 
\poemtitle[<<Не унывай, душа моя...>>]{***}
\settowidth{\versewidth}{\vinЧто тепен путь, трудна дорога!}
\begin{verse}[\versewidth]
\begin{altverse}
Не унывай, душа моя,\\
 Что темен путь, трудна дорога!\\
Стремись к добру, огнем горя…\\
 Любить кто хочет свет и Бога –\\
Пред тем во тьме горит заря.
\end{altverse}
\end{verse}

\newpage
\vspace*{0cm}


%173
\poemtitle{Служи добру}
\settowidth{\versewidth}{Взгляни вокруг, --- примеров много:}
\begin{verse}[\versewidth]
Служи добру, люби природу,\\
И в ней учись искать свободу,\\
Отбрось позорный, рабский страх, ---\\
Ведь ты родился не в цепях\ldotst\\!

Взгляни вокруг, --- примеров много:\\
Идут тернистою дорогой\\
Другие люди с юных лет;\\
У них в душе один завет ---\\!

Добыть себе иную долю,\\
%Иль гибнуть всем в борьбе за волю\\
Иль гибнуть всем в борьбе за волю,\\
И память тяжких черных дней\\
Украсить жертвою своей\ldotst
\end{verse}

\newpage
\vspace*{0cm}


%194
\poemtitle{Святая ночь в деревне}
\settowidth{\versewidth}{<<Христос воскрес!>> --- вещают миру}
\begin{verse}[\versewidth]
\begin{altverse}
<<Христос воскрес!>> --- вещают миру\\
    %Святая ночь и, небеса,\\
	Святая ночь и небеса,\\
Вокруг, в проталинах пестрея,\\
    Поля, и темные леса\ldotst\\
<<Воскрес Христос!>> --- порвав оковы,\\
    Лепечет вольная река,\\
И в темь глубокую --- свободу\\
    Несет с собой издалека\ldotst\\
Бодрит весенний воздух свежий,\\
    Дрожит пасхальный перезвон\ldotst\\
Толпами шумными крестьяне\\
    Во храм идут со всех сторон;\\
Спешат услышать голос правды,\\
    Что мир настал, --- Христос воскрес\ldotst\\
Любовь повеяла на землю ---\\
    Весной душистою с небес\ldotst\\
\end{altverse}
\end{verse}


\newpage
\vspace*{0cm}


%262
\poemtitle{Любовь}
\settowidth{\versewidth}{Любовь!.. Как жгучи эти звуки!}
\begin{verse}[\versewidth]
\begin{altverse}
Любовь!.. Как жгучи эти звуки!\\
    От них огонь горит в груди,\\
И утихают злые муки,\\
    И светит счастье впереди...\\
Любовь, как солнышко в лазури\\
    По небу синему плывет,\\
Порой среди страстей и бури\\
    С улыбкой тихо подойдет...\\
Обнимет крепко, безмятежно,\\
    Истомой сладкой обожжет...\\
И в поцелуе долгом, нежном,\\
    Нектаром сердце обольет...\\
\end{altverse}
\end{verse}

\newpage
\vspace*{0cm}


%275
\poemtitle{Под голубым небом}
\settowidth{\versewidth}{<<Зацветем под жгучим солнцем}
\begin{verse}[\versewidth]
\begin{altverse}
Тихо шепчется фиалка\\
    С василечком молодым,\\
Он сулит любовь и ласку\\
    Ей под небом голубым:\\
<<Зацветем под жгучим солнцем\\
    Краше прежнего, мой друг,\\
Пусть над нами раздаются\\
    Песни вольные вокруг>>...\\
А лучи тепла и света\\
    Льются, льются с высоты,\\
Опьяняют и ласкают\\
    Дивным блеском красоты...\\
Распылалася фиалка\\
    Страстью нежной к васильку,\\
И склонилась с поцелуем\\
    К молодому стебельку...\\
\end{altverse}
\end{verse}


\newpage
\vspace*{-2.5cm}


%209
\poemtitle{Ты не со мной}
\settowidth{\versewidth}{\vinБезумной ревностью любовь\ldotsq}
\begin{verse}[\versewidth]

\begin{altverse}
Мой друг, зачем ты омрачила\\
     Безумной ревностью любовь\ldotsq\\
Во мне огонь, и страсти сила\\
     Лишь для тебя волнует кровь\ldotst\\!

Ложусь в постель, твой образ снится,\\
     До утра реет надо мной,\\
И грудь моя в тоске томится, ---\\
     Зачем я, друг мой, не с тобой\ldotst\\!

Пойду ль вечернею порою\\
      За речку слушать соловья,\\
Туман ложится полосою,\\
      И в нем твой призрак вижу я\ldotst\\!

Спешу дрожащими руками,\\
    Тебя обнять, поцеловать,\\
Но призрак тает пред глазами,\\
      Порыв заставив проклинать\ldotst\\!

Один с глубокою  тоскою,\\
     Застыв в безмолвной тишине,\\
Стою в раздумье над рекою,\\
     Мечтою полон о тебе\ldotst\\!

Вокруг глухая ночь темнеет,\\
       Вдали чуть слышен соловей,\\
Больное сердце холодеет\\
        В груди истерзанной моей\ldotst\\!

Стою и --- грусть в душе лелею\\
     К тебе, мой ангел дорогой,\\
И об одном лишь я жалею,\\
  Что нет тебя, мой друг, со мной\ldotst
\end{altverse}
\end{verse}

\newpage

\vspace*{-2cm}


%251
\poemtitle{Не забуду}
\begin{verse}[\versewidth]
\indentpattern{10}
\begin{patverse*}
\vinТы мне сказала: <<Позабудь!>>\\
Но я, мой ангел, не забуду\ldotst\\
     При первой встрече где-нибудь\\
В твоих объятьях снова буду\ldotst\\
     %А --- нет, то в грезах золотых\\
	  А нет --- то в грезах золотых\\
Тебя увижу ночью темной,\\
      О счастье дней пережитых\\
Мечтать я буду с грустью томной.\\
     С тобою жизнь моя цвела\ldotst\\
Я помню ветхое оконце,\\
     Где занавесочка была\\
Тобой повешена от солнца\ldotst\\
     Как были ночи хороши,\\
Когда туда ты провожала.\\
     И в заколдованной тиши\\
Обнявши крепко, целовала\ldotse\\
     Поутру, раннею зарей,\\
Будить ходила на рассвете\ldotst\\
     Ах\ldotst Сколько радости порой\\
% Нужен ли тут вопросительный знак? Или восклицательный?
Я видел в ласковом привете\ldotse\\
На грудь с душевной простотой\\
Спешишь ко мне ты наклониться...\\
Даешь, любуясь красотой,\\
Желаньем сладостным упиться...\\
Прошла пора весны моей,\\
С тоскою осень наступает,\\
А память летних теплых дней\\
В душе моей не умирает...\\
Могу ли солнца луч забыть,\\
Который в холод сердце греет,\\
И милый призрак разлюбить,\\
Что нежно так душа лелеет\ldotsq
\end{patverse*}
\end{verse}

\newpage
\vspace*{0cm}

%94
\poemtitle{Расписной положок}
\settowidth{\versewidth}{\vinМне тебя приласкать\ldotsq}
\begin{verse}[\versewidth]
\begin{altverse}
%Ах, придется-ль когда\\
Ах, придется ль когда\\
   Мне тебя приласкать\ldotsq\\
Мне к другому теперь\\
   Тяжело привыкать...\\!

Помнишь, --- как горячо\\
    В положке обняла\ldotsq\\
С тех пор душу тебе,\\
    Милый друг, отдала\ldotst\\!

Ни запить, ни заесть\\
    Поцелуи твои, ---\\
Глубоко у меня\\
    Они в сердце легли\ldotst\\!

Примирилась с другим,\\
    Покорилась судьбе,\\
Но желанья мои\\
    Улетают к тебе, ---\\!

В расписной положок,\\
    На кроватку твою,\\
% нужна ли запятая? 
% Где ты, в сладостном сне\\ 
Где ты в сладостном сне\\
    Нежишь душу мою...
\end{altverse}
\end{verse}

\newpage
\vspace*{0cm}

% 222
\poemtitle{Она ждала}
\settowidth{\versewidth}{Она ждала и в даль глядела}
\begin{verse}[\versewidth]
\begin{altverse}
Она ждала и в даль глядела\\
    За синь туманную лесов,\\
И многое сказать хотела,\\
% В оригинале:
% Но, сердце мучилось без слов
    Но сердце мучилось без слов\ldotst\\
Она ждала, --- и без надежды\\
    Минуту счастья увидать ---\\
Ей заколдованные вежды\\
    Нельзя свободно поднимать\ldotst\\
% В оригинале:
% Но, в тайниках души несмелой
Но в тайниках души несмелой\\
    Горела пламенная страсть ---\\
Любви запретной, онемелой\\
    Бесследно миг не мог пропасть\ldotst\\
Она ждала\ldotst Порой мятежной\\
    Готова друга догонять, ---\\
% В отчаяньи --- ?
В отчаянье --- с улыбкой нежной\\
    Скрывая страх, его обнять\ldotst\\
%Она ждала --- смотря в оконце ---\\
Она ждала, смотря в оконце,\\
    Без сна, по целым по ночам,\\
Ждала с зарей, --- при свете солнца,\\
    Доверясь ласковым лучам.\\
Она ждала. Но --- ведьма злая\\
    К ней путь-дорогу замела,\\
%И вместо радостей, --- немая\\
И, вместо радостей, немая\\
    Тоска на сердце залегла.
\end{altverse}
\end{verse}

\newpage
\vspace*{0cm}


\poemtitle{Листок}
\settowidth{\versewidth}{Как много вдруг напомнил мне}
\begin{verse}[\versewidth]
\begin{altverse}
Как много вдруг напомнил мне\\
     Листок бумаги белой!\\
Здесь были вписаны слова \\
     Ее рукой несмелой\ldotst\\
Здесь скрыта нежная любовь\\
     Ее глубокой страсти, ---\\
Водила милая пером\\
     Скрываясь от напасти\ldotst\\
% В оригинале Что-б
Чтоб зависть злобная людей\\
     Любви не омрачила, ---\\
Горячей искренной слезой\\
     Листочек омочила\ldotst\\
И, как молчания печать,\\
     К листку уста прижала\ldotst\\
С печалью томною в глазах ---\\
     Прощаясь, уезжала\ldotst\\
И тайну чуткую теперь\\
     Хранит листочек белый\\
Следы, что к сердцу провела\\
     Она рукой несмелой\ldotst\\
\end{altverse}
\end{verse}

\newpage
\vspace*{0cm}


% 83
\poemtitle[<<Ниже, ниже наклоняйтесь...>>]{***}
\settowidth{\versewidth}{Чтобы с милым на свиданье}
\begin{verse}[\versewidth]
\begin{altverse}
Ниже, ниже наклоняйтесь\\
Серебристые березы,\\
Чтобы люди не видали\\
На моих ресницах слезы;\\
Чтобы с милым про свиданье\\
Я одна бы только знала\ldotst\\
%Утони же, глубже, тайна,\\
Утони же глубже тайна,\\
Там, где вишня расцветала,\\
Где цветы ее под нами\\
Белым снегом рассыпались,\\
И листы зеленой груши\\
Тихо, ласково шептались.\\
Зарастай моя дорожка,\\
Где я с милым проходила,\\
Поломалася малина,\\
Что весною посадила.\\
Оклевали птицы вишню\\
И черемуху густую\ldotst\\
Грустно мне глядеть под вязы\\
На скамеечку пустую\ldotse\\
Солнце спряталось за тучку,\\
Мглой мой садик одевает,\\
И холодный, резкий ветер\\
Листья желтые срывает\ldotst\\
\end{altverse}
\end{verse}


\newpage
\vspace*{-2.5cm}


%227
\poemtitle{Замужем}
\settowidth{\versewidth}{Радость, счастие --- далеко,}
\begin{verse}[\versewidth]
\begin{altverse}
Радость, счастие --- далеко,\\
    Больше их не увидать\ldotst\\
Так вздыхаючи глубоко\\
    Буду вечно вспоминать\\
О девичей вольной доле,\\
    Как у матушки родной\\
Я росла в любви и холе,\\
    Земляничкою лесной\ldotst\\
% Здесь в коце либо точка, либо двоеточие... в двух книгах
Я в засеночках белела:\\
    В новом браном пологу,\\
И на солнышке алела\\
    Я, резвяся на лугу\ldotst\\
Русу косыньку чесала\\
    Нежно матушка моя,\\
На задворках запевала\\
    В хороводах первой --- я\ldotst\\
Добры-молодцы гналися\\
    За моею красотой\ldotst\\
Но, те годы пронеслися\\
%%%% Тире так в оригинале
%    --- Безвозратною мечтой\ldotst\\
    Безвозратною мечтой\ldotst\\
Непробудно спит в могиле\\
%    Мать родимая моя.\\
    Мать родимая моя,\\
И досталось не по силе\\
    Мне коварная семья:\\
Злою ревностью изводит\\
    Муж, как старый домовой\ldotst\\
А свекровь, косырясь, ходит,\\
    Смотрит тучей грозовой\ldotst\\
За красу мою золовки\\
    Ненавидят и корят,\\
Провинятся в чем плутовки ---\\
    На меня наговорят\ldotst\\
%Деверья, --- те защипают\ldotst\\
Деверья --- те защипают\ldotst\\
    И не смею оттолкнуть.\\
Свекор с бабушкой пытают,\\
    Целясь метко упрекнуть\ldotst\\
Не с кем горюшко размыкать,\\
    Мне без матушки родной\ldotst\\
%%%% В оригинале через дефис
%Как-бы хуже не накликать\\
Как бы хуже не накликать\\
    В горьких думушках одной!\\
\end{altverse}
\end{verse}

\newpage
\vspace*{0cm}


%9
\poemtitle{Не мне}
\settowidth{\versewidth}{Что --- <<море лишь одно свободно,}
\begin{verse}[\versewidth]
\begin{altverse}
Что --- <<море лишь одно свободно,\\
%А люди, --- жалкие рабы,>> ---\\
А люди --- жалкие рабы,>> ---\\
Не мне, не мне она сказала\\
%В пылу желаний и мольбы\\!
В пылу желаний и мольбы!\\!
%В оригинале стоит !.
%Нет! я --- не раб, я --- рыцарь духа!\\
Нет! Я --- не раб, я --- рыцарь духа!\\
Пускай разит меня удар!\\
Я закаленный крепче стали,\\
В моей груди --- страстей пожар\ldotse\\!

Страдай, томись больное сердце,\\
%Терзайся, - рвись моя душа,\\
Терзайся, рвись моя душа,\\
За то, что с ней одна минута\\
%Была безумно - хороша!\\!
Была безумно хороша!\\!

За то, что я --- в порыве счастья ---\\
Зажег святой огонь в крови,\\
Готов пойти на муки ада,\\
Сгорая в пламенной любви\ldotst\\!

Нет! Я --- не трус\ldotse Я точно море\ldotse\\
Вскипает бурею душа\\
%При мысли: --- с ней была когда-то\\
При мысли: с ней была когда-то\\
Минута в жизни хороша\ldotst
\end{altverse}
\end{verse}

\newpage
\vspace*{-2.5cm}


%258
\poemtitle{Гуляка}
%\center{(из отдела юмористики)}
\settowidth{\versewidth}{Вчера в саду с гетерами}
\begin{verse}[\versewidth]
\begin{altverse}
Вчера в саду с гетерами\\
     Гульнуть я захотел\ldotst\\
С веселыми манерами\\
     Я пил, плясал и пел\ldotst\\
Гетеры грудь высокую\\
     Я страстно обнимал\\
И в губы черноокую\\
     Безумно целовал\ldotst\\
Забыл семью родимую,\\
     Я, --- с пьяной головой,\\
И звал неумолимую\\
     Гетеру за собой\ldotst\\
Она с улыбкой слушала,\\
     Что: <<Я, мол, поняла»\ldotst\\
Пила и много кушала.\\
     Бумажник забрала\ldotst\\
Как стал пустой бумажник мой,\\
     Гетеры взгляд потух.\\
%Пора, мой друг, тебе домой! ---\\
<<Пора, мой друг, тебе домой!>> ---\\
%      Смеясь, --- сказала вдруг\ldotst\\
      Смеясь, сказала вдруг\ldotst\\
%Нечестно! --- крикнул с злобою,\\
<<Нечестно!>> --- крикнул с злобою,\\
     Рванувшись я вперед\ldotst\\
Но, пред моей особою\\
     В момент явился счет:\\
%Духи\ldotst Помада\ldotst Сволочи! ---\\
Духи\ldotst Помада\ldotst <<Сволочи!>> ---\\
     Я крикнуть им хотел,\\
И сам от злобной горечи\\
     Как будто онемел\ldotst\\
И нагло так обобранный,\\
     Шатаясь, вышел я\ldotst\\
Глазам моим оборванной\\
     Представилась семья\ldotst\\
%Пришел домой и, нежно --- я\\
Пришел домой и нежно я\\
     Жену свою обнял,\\
И в щечку белоснежную\\
     Бедняжку целовал\ldotst\\
Утешил речью сладкою\\
     За милый, нежный взгляд,\\
%И, плакал я украдкою\\
И плакал я украдкою,\\
%     Почуяв --- в сердце яд\ldotst
     Почуяв в сердце яд\ldotst
\end{altverse}
\end{verse}


\newpage
\vspace*{0cm}

%257
\poemtitle{Искренний ответ}
%\center{(из отдела юмористики)}
\settowidth{\versewidth}{\vinБудто бы юность разделишь со мной?}
\begin{verse}[\versewidth]
\begin{altverse}
%--- Правда-ли то, что вчера мне сказала,\\
%    Будто бы юность разделишь со мной?\\
<<Правда ли то, что вчера мне сказала,\\
    Будто бы юность разделишь со мной?\\
%Милая, верно-ль себя проверяла?\\
Милая, верно ль себя проверяла?\\
    Может страданья те были виной,\\
Что в неудачной любви испытала\\
    Чуткой душою, в обмане с другим?\\
Бедная, верно ль меня ты узнала.\\
%    С сердцем знакома-ль разбитым моим?\\
    С сердцем знакома ль разбитым моим?\\
Знаешь ли ты, --- что все те же страданья\\
    Встретишь со мною в пути роковом?\\
%Искренно друг мой скажи на признанье,\\
Искренно, друг мой, скажи на признанье,\\
    Чтобы я понял в ответе твоем;\\
Будешь ли сладко, с кипучею кровью,\\
    Кудри седые мои целовать?\\
Можешь ли пламенно, с нежной любовью,\\
%    В страстном порыве меня ты ласкать\ldotsq\\
    В страстном порыве меня ты ласкать\ldotsq>>\\
%--- <<Буду! Лукаво она отвечала, ---\\
%--- Буду и утром, и ночью в тиши\ldotst\\
%--- Буду! Глядя ему в очи шептала, ---\\
%--- Только именьице мне подпиши\ldotst
<<Буду! --- лукаво она отвечала, ---\\
Буду и утром, и ночью в тиши\ldotst\\
Буду! --- глядя ему в очи шептала, ---\\
Только именьице мне подпиши\ldotst>>
\end{altverse}
\end{verse}

\newpage

\vspace*{-2.5cm}


%64
\poemtitle{В подвале}
\settowidth{\versewidth}{\vinЧто над смиренницей женой}
\begin{verse}[\versewidth]
\indentpattern{10}
\begin{patverse*}
\vinОн был и пьяница, и мот\ldotst\\
Играл, кутил и похвалялся,\\
 Что над смиренницей женой\\
Своею властью издевался\ldotse\\
 Она, <<законная>> раба,\\
С святым терпеньем все сносила,\\
 И вдруг --- коварная судьба\\
На зло <<владыке>> подшутила\ldotse\\
Какой-то купчик молодой\\
Смутил бедняжку для потехи...\\
Свозил на тройке в ресторан\\
И дал немножко <<на орехи>>, ---\\
И, после сладких, лестных слов,\\
Оставил гостью на квартире.\\
Спознавшись с купчиком, она\\
Души не чаяла в кумире\ldotst\\
Забыв о муже, расцвела\\
Опять, как лилия весною,\\
%Живая, стройная, под час, ---\\
Живая, стройная, подчас\\
С ума сводила всех красою\ldotst\\
Муж долго, злобно ревновал,\\
Грозил убить, иль сам убиться,\\
%И кончил тем, --- пришлось совсем,\\
И кончил тем, --- пришлось совсем\\
% Как лучше тут?
%В хмелю кутил с кругу спиться\ldotst\\!
В хмелю кутиле с кругу спиться\ldotst\\
Слонялся точно дикий пес,\\
Голодный, грязный, без призора,\\
И ночью часто засыпал\\
В навозной куче у забора\ldotst\\
Прошли цветущие года,\\
И ей вдруг счастье изменило:\\
Другую купчик полюбил,\\
И снова стала жизнь постыла\ldotst\\
Она не вынесла борьбы\\
Своей озлобленной душою,\\
% Нужен ли дефис?
%И, так-же, как погибший муж,\\
И так же, как погибший муж,\\
Привыкла к горькому запою\ldotst\\
Шаталась так же, как и он,\\
Кляня позорное изгнанье,\\
В грязи, в лохмотьях, босиком,\\
Прося у встречных подаянья\ldotst\\
Их с мужем вновь судьба свела\\
В углу вонючего подвала\ldotse\\
Взглянули ей в беззубый рот\\
Глаза голодного шакала\ldotst\\
Но страх пропал минувших лет,\\
И в ней теперь дышала злоба,\\
Никто на шаг не отступал,\\
Как псы, --- готовы грызться оба;\\
В припадке бешеном отмстить\\
Хоть здесь за жгучую обиду\ldotst\\
И, вмиг смирившись, муж не снес\\
Жены истерзанного вида:\\
Сознал вину он в первый раз,\\
Прочтя в ее глазах презренье,\\
%Сказав: <<прости, я виноват>>\ldotst\\
Сказав: <<Прости, я виноват>>\ldotst\\
И --- подал руку примиренья\ldotst
\end{patverse*}
\end{verse}

\newpage
\vspace*{0cm}


%153
\poemtitle{В кабаке}
\settowidth{\versewidth}{Лучше было б век мне в девушках сидеть,}
\begin{verse}[\versewidth]
%\vinНе бушуй, уймись гуляка молодой,\\
\vinНе бушуй, уймись, гуляка молодой,\\
Пожалей, не рви зипун себе худой\ldotst\\
Вон, --- жена твоя в лохмотьях, босиком, ---\\
%Со слезами просит хлеба под окном\ldotst\\
Со слезами просит хлеба под окном\ldotse\\
%Ведь, детишки-то голодные сидят, ---\\
Ведь детишки-то голодные сидят\\
И в худые окна жалобно глядят\ldotst\\
Поджидают мать желанную с сумой,\\
%Ты-ж, --- бушуешь с беззаботной головой, ---\\
Ты ж --- бушуешь с беззаботной головой, ---\\
Будто горы золотые ты нашел\ldotst\\
%И, с похмелья на работу не пошел\ldotst\\
И с похмелья на работу не пошел\ldotst\\
Каждый день идешь кабак ты навещать,\\
Как же с голоду детишкам не кричать\ldotsq\\
Ох, --- не скоро тебя вызовешь домой ---\\
Хоть бы мать пришла желанная с сумой\ldotst\\
Истомилась с ребятишками она\\
Каждый день тобой в печаль погружена;\\
Чуть плетется нездоровая домой\\
%К ребятишкам, обездоленным, с сумой\ldotst\\
К ребятишкам обездоленным с сумой\ldotst\\
%Грустный взор ее слезинками блестит, ---\\
Грустный взор ее слезинками блестит,\\
%Кровью кашляет, бедняжка, говорит, ---\\
Кровью кашляет, бедняжка, говорит:\\
<<Ни жилица я на свете молода,\\
Красоту мою состарила нужда\ldotst\\
Лучше было б век мне в девушках сидеть,\\
Чем ему в глаза бесстыжие глядеть>>\ldotst
\end{verse}
Первая публикация в 1904, затем в 1917 было переделано. Печатается по переделанному.

\newpage
\vspace*{0cm}


\newpage
\vspace*{0cm}




\newpage
\vspace*{0cm}

%122
\poemtitle{Не один я}
\settowidth{\versewidth}{Это --- поле, да луг, лес зеленый за речкой,}
\begin{verse}[\versewidth]
\indentpattern{10}
\begin{patverse*}
\vinЯ теперь не один --- у меня есть друзья,\\
Это --- поле, да луг, лес зеленый за речкой,\\
%    А холодной зимой дума гостья моя ---\\
    А холодной зимой дума --- гостья моя,\\
Во всю долгую ночь с ней сижу я за свечкой,\\
    Мы поем, говорим и поплачем, когда\\
Грусть за сердце возьмет и защемит невольно\ldotst\\
    А забрезжит заря, оживаем тогда ---\\
С нею радость придет, снова сердце довольно\ldotst\\
    И ждем светлого дня, ждем мы теплых лучей,\\
Солнце в каплях росы по лугам засверкает ---\\
    Позабудем тогда холод зимних ночей,\\
%И, смелей запоем, о чем сердце желает\ldotst
И смелей запоем, о чем сердце желает\ldotst
\end{patverse*}
\end{verse}

\newpage
\vspace*{0cm}


% 265
\poemtitle{Продолжение биографии}
% В старом варианте ``Думы за работой''
\settowidth{\versewidth}{И та же грусть, и те страданья,}
\begin{verse}[\versewidth]
\begin{altverse}
Мои стихи --- в страду работы\\
    Порой слагаются в тиши,\\
И часто --- с горя и заботы\\
    Со стоном рвутся из души\ldotst\\
И та же грусть, и те страданья,\\
    Что много лет тому назад\\
В глуши, впотьмах, без света-знанья\\
    %Мне вносят жизненный разлад,\\
	Мне вносят жизненный разлад.\\
Тяжелый труд, нужда и муки ---\\
    В борьбе за счастье и простор,\\
Согнулся стан, в мозолях руки,\\
    Уныло гаснет ясный взор\ldotst\\
Любовь и радость изменяют\ldotst\\
    Мрачней полуночи мечты\ldotst\\
За тучей тучу нагоняют, ---\\
    Взамен поблекшей красоты\ldotst\\
Грозят и скорби, и невзгоды ---\\
    Мой челн ударом раздробить\ldotse\\
Но, кто был силен в непогоду,\\
    И, бурю может полюбить\ldotst
\end{altverse}
\end{verse}

\newpage
\vspace*{0cm}


\newpage
\vspace*{0cm}



\newpage
\vspace*{0cm}


%272
\poemtitle{Песня солдата}
\settowidth{\versewidth}{За свободу родимого края\ldotst}
\begin{verse}[\versewidth]
\indentpattern{110}
\begin{patverse*}
\vinМоя родина мать!\\
%    Я иду умирать\\
    Я иду умирать,\\
Братьям счастья и света желая;\\
    Все готов перенесть\\
    Я за славу и честь,\\
За свободу родимого края\ldotst\\
    В грозном, страшном бою\\
    Кровь не раз я пролью,\\
%Но, сломивши --- врага супостата, ---\\
Но, сломивши врага-супостата,\\
    Я хочу за тебя\\
    Умереть, --- всех любя,\\
С верой в подвиг родного солдата.
\end{patverse*}
\end{verse}

\newpage
\vspace*{0cm}


%279
\poemtitle{Моя звезда}
\settowidth{\versewidth}{Средь жуткой тишины с улыбкой неземной}
\begin{verse}[\versewidth]
Один\ldotst Один вокруг, --- куда ни погляди\\
Лишь стены серые грозятся мне тоскою,\\
%И долгожданный луч надежды впереди\\
И долгожданный луч надежды впереди,\\
%Он вряд-ли подарит желанный миг покоя.\\
Он вряд ли подарит желанный миг покоя.\\
%Туман, седой туман глядит ко мне в окно\\
Туман, седой туман глядит ко мне в окно,\\
И вместе с темнотой за страхом страх вползает,\\
%И также на душе мучительно темно\\
И так же на душе мучительно темно,\\
Как будто жизнь сама сгорая, угасает\ldotst\\
%Но, это что\ldotsq Звезда моя опять со мной\\
Но, это что\ldotsq Звезда моя опять со мной,\\
Опять меня волшебным светом озарила,\\
Средь жуткой тишины с улыбкой неземной\\
Любовь и теплоту вокруг меня разлила\ldotst\\
%Затмить мою звезду не мог туман седой\\
Затмить мою звезду не мог туман седой,\\
%Окутать в мрак, --- была бессильна злоба ночи\\
Окутать в мрак, --- была бессильна злоба ночи,\\
%И вдруг я, --- увидал всю радость пред собой\\
И вдруг я, увидал всю радость пред собой,\\
%Глядят мне в душу вновь божественные очи\\
Глядят мне в душу вновь божественные очи,\\
%И, сердце трепетать заставила краса\ldotst\\
И сердце трепетать заставила краса\ldotst\\
%И, молодость моя --- как будто возвратилась\ldotst\\
И молодость моя, --- как будто возвратилась\ldotst\\
%О, дайте мне налюбоваться небеса\\
О, дайте мне налюбоваться, небеса,\\
Пока за горизонт она не закатилась.
\end{verse}

\newpage
\vspace*{0cm}


% 261
\poemtitle{Во власти поэзии}
\settowidth{\versewidth}{Мне жизни больше, жизни дайте\ldotse}
\begin{verse}[\versewidth]
\indentpattern{0010012}
\begin{patverse*}
Мне жизни больше, жизни дайте\ldotse\\
Сильней вокруг благоухайте\\
    Росой медвяною цветы\ldotse\\
%О розы, розы, расцветайте\\
%И сладким хмелем опьяняйте;\\
%    Меня волшебные мечты\ldotst\\
О розы, розы, расцветайте!\\
И сладким хмелем опьяняйте\\
    Меня, волшебные мечты\ldotst\\
		Со мною --- ты\ldotst\\
Моя любовь, моя царица,\\
Стройна, как ель и белолица,\\
    Мила, как бархатный цветок,\\
Твои глаза --- лучи зарницы,\\
А голос --- дивный звук цевницы,\\
    Речей твоих живой поток\\
        Меня увлек.\\
И стал я твой, как раб под властью,\\
Живу, дышу твоею страстью,\\
    В любви сгораю без огня\ldotst\\
Иду, спешу навстречу счастью\\
С тобой отдаться сладострастью, ---\\
    Под шепот лип, при свете дня\ldotst\\
        Зажги меня\ldotst
\end{patverse*}
\end{verse}

\newpage
\vspace*{-2.5cm}


% 263
\poemtitle{Раздумье}
\settowidth{\versewidth}{Много есть печали}
\begin{verse}[\versewidth]
\begin{altverse}
Много есть печали\\
В сердце у меня,\\
Да запас хранится\\
Силы и огня\ldotst\\
Думаю, что слажу\\
Я с неволей злой,\\
Не отдамся в руки\\
И нужде лихой\ldotst\\
Встану я с зарею,\\
Полосу пахать,\\
Под лучами солнца\\
%Зерны разсевать\ldotst\\
Зерны рассевать\ldotst\\
Буду ждать я всходов\\
На полях родных,\\
Да беречь посевы\\
От морозов злых\ldotst\\
%И дождусь цветущей, ---\\
И дождусь цветущей,\\
Колосистой ржи,\\
Намечу я копен\\
Много у межи;\\
Смолочу, вывею,\\
Зернышек продам;\\
Снова дома счастья\\
Ребятишкам дам,\\
Заживу с семьею,\\
Как большой богач\ldotst\\
%Погоди-же сердце,\\
Погоди же сердце,\\
Погоди, не плачь\ldotst
\end{altverse}
\end{verse}

\newpage
\vspace*{0cm}


%89
\poemtitle{Пленник}
\settowidth{\versewidth}{\vinНа свободе летать,}
\begin{verse}[\versewidth]
\begin{altverse}
Хорошо снегирю\\
На свободе летать,\\
Буйный ветер полей\\
В камышах догонять;\\
Вить с подружкой гнездо\\
Теплой ранней весной,\\
И по зорям свистать\\
На опушке лесной\ldotst\\
Но не здесь одному,\\
В теремке золотом,\\
Изнывая в плену\\
За хрустальным окном,\\
С неизменной тоской\\
За решеткой сидеть,\\
И на зелень лугов\\
Сиротливо глядеть\ldotst\\
Ни отборным зерном,\\
Ни водой ключевой,\\
Не заменят ему\\
Колосок полевой\ldotst\\
Там отрада его ---\\
Меж лесов и меж гор,\\
Где синеет река\\
И далекий простор\ldotst
\end{altverse}
\end{verse}

\newpage
\vspace*{0cm}


%124
\poemtitle{Океан}

%\small{Посвящается Л. Н. Толстому в день его 80-летней годовщины!}
%Исправил - Г.К.
%\small{Посвящается Л. Н. Толстому в день его 80-летней годовщины.}
% Точка не нужна в посвящении:
%\small{Посвящается Л. Н. Толстому в день его 80-летней годовщины}

\settowidth{\versewidth}{\vinТы вечно диких сил гроза!}
\begin{verse}[\versewidth]
\indentpattern{10010}
\begin{patverse*}
\vinТы вечно диких сил гроза!\\
%Смываешь темныя твердыни\ldotst\\
Смываешь темные твердыни\ldotst\\
В твою бездонную пустыню\\
Невольно падает слеза\\
%Народа страждущаго ныне\ldotst\\
Народа страждущего ныне\ldotst\\
Опять ты выйди из тиши\\
Под гнетом властного напора,\\
И в даль желанного простора\\
В порыве радостном спеши\\
%Чрез скалы серыя и горы\ldotst\\
Чрез скалы серые и горы\ldotst\\
И, с бурей, --- богатырь седой,\\
Омой враждебную природу\ldotst\\
И необъятную свободу\\
Весны цветущей, золотой\\
Неси родимому народу\ldotst
\end{patverse*}
\end{verse}

\newpage
\vspace*{-2cm}




\newpage
\vspace*{0cm}

%36 
\poemtitle{Одинокое сердце}
\begin{verse}[\versewidth]
Как солнце тускло за горой\\
Леса глухие освещает,\\
%Так радость, редкою порой,\\
Так радость редкою порой\\
Больное сердце навещает\ldotst\\
Знать, ей в груди уж места нет!\\
Тоска гнездо себе свивает,\\
%Вползет змеей, и тяжкий след\\
Вползет змеей и тяжкий след\\
На сердце бедном оставляет.\\
И гложет грусть\ldotst Глядишь в окно, ---\\
Когда вдали заря займется\ldotsq\\
%За то, --- как счастливо оно\\
Зато --- как счастливо оно\\
При светлой радости забьется\ldotse\\
Вся кровь в нем вспенится ключом!\\
И, встретив день весны лазурной,\\
Готово таять под лучом\\
В живой любви и страсти бурной.
\end{verse}


\newpage
\vspace*{0cm}

\poemtitle{Не скажу}
\settowidth{\versewidth}{Я в бесплодной тоске ожиданья}
\begin{verse}[\versewidth]
\begin{altverse}
Я в бесплодной тоске ожиданья\\
     Одинок --- в полумраке сижу\ldotst\\
С нетерпеньем клокочут рыданья\ldotst\\
     Но о том --- никогда не скажу, ---\\!

Что на сердце глубоко таится\ldotst\\
     Что невольно так мучит меня!\\
Что коварства людского боится,\\
     И пылает сильнее огня\ldotst\\!

%Тихой лаской – порой согревает\\
Тихой лаской порой согревает,\\
     Поцелуем касаясь седин\ldotst\\
%И, что душу тоской надрывает! ---\\
И что душу тоской надрывает! ---\\
     %Тайну эту – я знаю один\ldotst
	 Тайну эту я знаю один\ldotst
\end{altverse}
\end{verse}

\newpage
\vspace*{0cm}


\poemtitle{Мысли}
\settowidth{\versewidth}{Пусть мысли также будут чисты,}
\begin{verse}[\versewidth]
\begin{altverse}
Пусть мысли также будут чисты,\\
          Как снег белеет за окном,\\
%Резвы как молния, лучисты\\
Резвы как молния, лучисты,\\
          Как грань алмаза пред огнем.\\!

Пускай оденутся в порфиру\\
          Весны цветущей золотой.\\
В лучах на славу всему миру\\
          Заблещут светлой красотой.\\!

Летят с седыми облаками\\
          В лазурь далекую небес,\\
Весной с зелеными лесами\\
          Плетут узорчатый навес.\\!

Плывут за легкими ладьями,\\
          Кружась и пеняся в реке.\\
И пусть над звонкими ручьями\\
          Как лебедь вьются вдалеке.\\!

Пусть реют в пламенном эфире,\\
          Гоня усталость темных вежд,\\
Чтоб с новой силой в грешном мире\\
Искать спасительных надежд\ldotst
\end{altverse}
\end{verse}
* В ночной глуши

Глушь немая... не колышет
Ветер темные кусты,
И повисли в сладкой неге
Задремавшие листы.

В небе, тихо угасая,
Зорька алая горит.
Луг, обрызганный росою,
Бриллиантами блестит.

Над зарею золотятся
Кружевные облака
Их, купая, обнимает,
В светлом зеркале река.

С ними в струи вод глядится,
Опрокинувшись, камыш
И пьянит, благоухая,
Замирающая тишь.

В ольхах свистнул зимородок,
Встрепенувшись перед сном,
И спустилася на землю
Ночь спокойная кругом…


* Тоска

Как хмур и темен стал мой сад!.
Куда ни кинь усталый взгляд, -
Везде следы холодных дней
Ведут печальный разговор
С душой тоскующей моей…

Валятся желтые листы…
Мне жаль весенней красоты
С огнем и ласкою в глазах -
Она здесь пела и цвела
Царицей в солнечных лучах.

Теперь здесь пусто без цветов...
Наряд осыпался с кустов,
Смешался с пылью на земле;
Деревья голые стоят,
Дрожа в осенней полумгле.

И на душе темно, темно...
Знать, счастье светлое давно
Ушло с весенней красотой...
Один теперь остался я
С своею грустною мечтой!


* Ель

Затуманилися дали...
Поседели облака...
И блестят яснее стали
Пруд и темная река...

Потихоньку опускаясь,
Словно пух, летит снежок
И, по травке расстилаясь,
Серебрит вокруг лужок;

Виснет клочьями повсюду:
На деревьях и кустах,
Липнет к дремлющему пруду,
Замирая в берегах...

Голый лес поник угрюмо,
Чуя скорую метель...
И тревожат только думы
Зеленеющую ель...

Ель стоит, как прежде, стройно,
С приподнятой головой,
Улыбаяся спокойно,
Вся в одежде парчевой...


* У родимой матушки

Купаясь в сером мареве,
В дали темнеет лес
И облака завесили
Лазурный свод небес...
Суровый ветер носится,
С дождем в окно стучит,
В избе нужда угрюмая
Предательски молчит...
Покуда гостья черная
Не всем еще страшна,
Живая, песня чудная
Из хижины слышна...
Поет, не зная горюшка,
Красотка за столом,
Спокойно грудь высокую
Склонивши над шитьем...
За матерью родимою
И труд не угнетет,
И дочь как зорька ясная,
Алеет и цветет...
В нужде родилась, выросла,
Уж ей не привыкать
В страде порою летнею
Пахать, косить и жать...
Зато как любо, весело
С задорным огоньком
Играть, резвясь на улице,
Весенним мотыльком!..
Ах, нет дороже волюшки,
Когда бушует кровь
И красит жизнь не золото,
А ласка, и любовь!


* У моря

Мне любо слышать рокот моря,
Глядеть во след бегущих волн,
Мечтою реять на просторе,
Как быстрокрылый легкий челн...

В размахе бешеном пучину
Дробить алмазами огней
И безотвязную кручину
Топить в дни пасмурные в ней...

В объятьях волн прибрежных тонет,
Вздымаясь шпицами, гранит,
А море громче, злее стонет,
Дрожа и пеняся кипит...

И гордо в схватке остывает -
Ложатся волны словно сталь...
И сердце снова оживает,
Стремясь в неведомую даль...


* В глухих стенах

В глухих стенах грущу одна...
И с милым редкие свиданья
Вселяют в душу лишь страданья
И заставляют пить в разлуке
Отраву горькую до дна...

Была весна тепла, ясна,
Вокруг цвело и благоухало...
Я вместе с другом ликовала...
И вдруг исчез тот миг отрадный,
Как сладость утреннего сна...

Угнала милого судьба
С полей родимых в край далекий...
Безлюдный север, снег глубокий
Я вижу в окна одиноко,
А в сердце - грустная борьба...

И вновь с мучительной тоской
Растет желанье встретить счастье,
В волнах житейского ненастья
Смелей свой легкий бег направить
К свободе, к дали голубой...


* Не мне

Что «море лишь одно свободно,
А люди — жалкие рабы» —
Не мне, не мне она сказала
В пылу желаний и борьбы...

Нет! Я не раб, я — рыцарь духа!
Пускай разит меня удар!
Я, закаленный крепче стали,
В моей груди — страстей пожар!..

Страдай, томись больное сердце,
Терзайся, рвись моя душа,
За то, что с ней одна минута
Была безумно хороша!

За то, что я в порыве счастья
Зажег святой огонь в крови,
Готов пойти на муки ада,
Сгорая в пламенной любви...

Нет! Я не трус!.. Я точно море!..
Вскипает бурею душа
При мысли: с ней была когда-то
Минута в жизни хороша...


* Камелия

Камелия, мой чудный цвет,
Цветов тебя милее нет!
Ты вечно борешься с природой,
Цветешь в суровую погоду,
Зимою сочный твой листок
И розо-бархотный цветок
Весенний день напоминают,
Мечты о многом вызывают...
Мечты летят к подруге той,
Что белоснежною рукой
Тебя садила-поливала,
В жару увянуть не давала...
Камелия, мой чудный цвет!
Милее той подруги нет,
Живет она в тяжелой доле,
Судьбой забитая, в неволе,
Тоска голубушку гнетет...
Но, все же, милая, цветет,
Как ты, невольно презирая,
И ласкова, как дева рая...


* Неволя

Теперь одни мечты со мною
А прежних радостей уж нет!
Один с суровую судьбою
Борюсь в глуши под градом бед...

Под хохот бурь и шум метели
Влачится жизнь, как ночь темна,
И слышно — стонут, гнутся ели,
Касаясь мерзлого окна...

В каморке пусто и уныло,
Ночник мерцает на стене...
И тают силы... Все постыло...
И сердце сдавлено во мне;

Простора нет ему, как прежде,
Отрадно биться и любить…
И меркнут светлые надежды,
Свободным соколом не быть…


* Друзьям

Отрадней было прежде нам…
То время помните ль, друзья,
Когда был с вами вместе я?
Смелее рвались мы душой
К родным, желанным берегам!
И где мы шли, рыдала мгла…
И, чар томительных полна,
Плыла задумчиво луна,
Среди берез и лозняка
Узоры чудные ткала…
Шептали ласково кусты
Про жизнь свободную в стране,
И взгляд тонул наш в вышине,
Где звезды светлые горят,
Как очи вечной красоты.
Сливались наши голоса
В желанный гимн любви святой;
Теперь давно нет дружбы той!
И взор усталый наш слепят
Туман и едкая роса…


* ***

Ах, как весной отрадно жить!
Она мне душу обновляет,
На честный труд благословляет,
И ум мой глубже вдохновляет
Заветы правды полюбить…

Как нежный друг, зовет весна,
С надеждой, полной упованья,
Смелей идти путем страданья,
К добру и светлому сознанью,
Будить родимый край от сна.


* Жертва труда

Клубится дым внутри овины,
Народ в поту и запылен…
Гудит привод… Хрустя, машина
Жует зубцами жесткий лен.
Мужик, как гном, стоит при свете
В углу у грязных фонарей,
Кричит: «скорей, скорее дети
Бросайте лен в валы, скорей!»
Сестер, девчонка, обгоняя,
Сменить отца бегом бежит,
Тресту проворно отрывая,
В зубцы валов вложить спешит.
Но знать, девчонку ожидало
Несчастье злое в этот миг:
Неверный взмах и — руку сжало
В зубцах отточенных стальных.
Раздался вопль; зловеще блещет
Фонарь мерцающий в углу…
С рукой оторванной трепещет
Девчонка, падая во мглу.
Отец в безумьи, брат рыдает,
В слезах подростки голосят…
Гудит привод, не утихает,
В испуге лошади храпят.
Перенесет ли мать больная?
Вернется ль снова радость в дом?
Привод гудит, костру взметая,
И жуткий стон стоит кругом.


* Набат

Гуди, набат, гуди смелее,
Звени металл, звени сильнее
Во мраке млеющей ночи,
Над сытой праздностью в бездельи,
Что вечно кружится в похмельи,
Металл — кричи!

С ударом ярче разгорайся,
Над злобой громче раздавайся,
И бурей грозною гуди,
Где тонут люди в тьме пороков,
Забывши Бога и пророков…
Их сон — буди!

Я верю, смелый зов набата
В душе разбудит совесть брата,
Проложить к правде верный путь,
Подружить скоро с светом знанья,
Вселить ему к добру желанья
С любовью в грудь…

На милость сменят гнев тираны,
Любовь у всех залечит раны,
Друзей создаст из палачей,
Согреются сердца народа,
Полюбят все тогда свободу,
Как даль ручей.


* Орел в плену

Орел в плену, лишенный воли,
Поник под тяжестью оков;
Не ждет себе он лучшей доли
И чует: не летать уж боле
Среди свободных облаков…

Не вить гнезда с подругой милой
Среди ущелий и теснин,
И не равняться с ветром силой…
Ему все кажется могилой —
Не видно солнца и долин!

Крепка тюрьма… Но гордо очи
Глядят вокруг на палачей,
И кажется, во мраке ночи
Орел встряхнется, что есть мочи,
Порвет позорный гнет цепей!


* Батрак

Знать, слезами меня умывала
В колыбели, родимая мать,
И тяжелый мне путь указала:
Век с семьею в нужде коротать;

На ногах грошевые лаптишки
По грязи волочить за сохой,
И дрожать по зимам в зипунишке,
Побираяся с хлебом, с сумой…

Не досталось мне, горе-Емельке,
Ни небесных даров, ни земных,
Захватили родную земельку
Стая коршунов – хищных и злых…

Хоть клочок бы землицы родимой!
Я б руками ее раскопал,
И с своею семью любимой
Без лошадки вспахал, взборновал;

Да не дали наследства мне деда,
И полоска родная моя
Очутилась у Прокла соседа,
А моя вся батрачит семья.

С малолетства в объятьях неволи,
Как меня, ее сжала судьба!..
А родимое, дедово поле,
Заменили теперь отруба..


* Юноше

Не верь страстям, оно есть мука,
Отрава лучших юных дней;
От них померкнет свет-наука,
Отрадный луч красы твоей;
Обман их в сердце направляет
Стрелой чарующей любви,
И яд растленья оставляет
В твоей клокочущей крови…
Встряхнись, - безумством омраченный,
Скорей беги от них вперед, -
Туда, где скорбью удрученный
К свободе движется народ…


* Уйди, тоска

Уйди тоска, уймитесь слезы!..
Не вечно-ж вьюги да морозы,-
Пора весне теплом дохнуть …
Ее живительные грозы
Навеяли мне в сердце грезы,
Свободней хочется вдохнуть!
Зимы холодной сбросив маску,
Трепещет лес в веселой пляске,
И, страстью нежною дыша,
Весна чарует теплой лаской
И шепчет тихо-тихо сказку,
Как жизнь отрадно хороша…
В моей красе утехи много!..
Иду родною всем дорогой,
Несу трудящимся привет!..
Наказан кто судьбою строго,
Тому несу я веру в бога,
Живущим в тьме несу рассвет…
А тем кто в горе и печали
Сирот и бедных утешали,
Несу безмерную любовь,
Чтоб чувств святых не угашали,
Я дам им новые скрижали
И свежих сил волью их кровь…


* Переселенцы

Здесь вера твердая и мощь
Нашли желанную опору,
Страна свободного труда
Предстала пламенному взору.

И смело плуг забороздил
По глади девственного поля;
С тех пор пришельцев у реки
Лелеет ласковая доля.

Пригрело солнышко семью
В глуши от родины далекой,
И труд тяжелый их расцвел
Вокруг избушки одинокой…



* Без друга

Сколько грусти одинокой
Я в то время пережил,
Как суровый край далекий
Нас с тобою разлучил…
Без тебя вокруг увяли
Рано травы и цветы,
На деревьях полиняли
Шелковистые листы…
Что заботливо ласкала
Друга нежная рука,
Здесь, со мною, оковала
Безысходная тоска…
Только призраки остались
Посреди родных полей,
Что, порой, переплетались
С думой смелою моей…
Лишь они хранят ревниво
Друга образ дорогой,
По излучинам и нивам
Тихо следуя за мной…


* Прислуга

У храма, на каменных плитах,
Девчонка стоит у притвора.
В лохмотьях и горем убита,
Закрылась худою ручонкой,
Не кажет унылого взора.
Лихая неволя загнала
Сюда за грошами бедняжку,
У девочки мать захворала…
Сегодня последний остаток -
Проели на хлебе рубашку.
Отца с колыбели не знает,
А – знатный торговец купчина…
Голодная мать умирает,
Бросает сироткой малютку
Такая, знать, в жизни причина!..
До ней мать жила в услуженьи,
По бедности все испытала,
Терпела порой униженьи
Хозяина брань, и невольно,
Бесстыдные ласки нахала…
Причина и тут объявилась;
Больною прислугу признали,
На грех и девчонка родилась…
Несчастную мать вдруг с ребенком,
Позорно из дому прогнали.
С тех пор уже торной дорогой
Доходит она до могилы…
Лишений так видела много,
Что слезы ей выели очи,
И вытекли каплями силы.
Купчина же, с глаз прогоняя,
Терпел ли хоть долю укора?
И, глядя, как, еле живая,
Тащилась с ребенком прислуга,
Жалел ли увядшего взора?
Нет, чужды знать, сытым страданья!
Забыл о прислуге купчина,
Не бросил гроша подаянья…
И ныне прошел он, довольный,
Не видя малютки кручины…


* Нужда

Ах, знать, в покое не оставит
Меня суровая нужда
И радость светлую отравит.
И, как покорного раба,
Нести позорный гнет заставит,
Среди лишений и труда!

Откуда, как она явилась?
Какою мрачною порой?
Со мною вместе, знать, родилась,
В страданьях матери слезой
И незаметно опустилась
На плечи влажною росой…

Со мной росла… И - вдруг змеею,
Мне грудь и шею обвила
И, крепко сжав с улыбкой злою,
Путем тернистым повела,
Крапивой жгучей и хвоею
Язвила тело мне и жгла…

Я, с ней в борьбе расправив руки,
Старался с силою смахнуть
Долой нужду и злые муки, -
Больную спину разогнуть…
Да нет, не в силах, знать уж, злюки
Теперь с усталых плеч стряхнуть!

Сильнее, крепче обнимает,
Как брата старшая сестра,
И желчью горькою питает
Меня до вечера с утра,
Коварно шепчет и внушает,
Что примириться с ней пора…

Ох, знать, в покое не оставит
Меня суровая нужда…
Я чую, - сердце мне отравит,
И, как покорного раба,
Нести позорный гнет заставит
Среди лишений и труда!


* Неведомый друг

Счастливец тот, кто в сладкий миг
С тобой желанье разделяет,
И грудь высокую твою
В безумной страсти обнимает…

Ему с блаженством суждено
Владеть тобою без запрета
Но, чем-то тихою порой,
Твоя душа теплей согрета…

Ты что-то робко от людей
В душе таинственно вскрываешь,
Как будто в грезах золотых
С другим прелестней расцветаешь?..

С улыбкой нежной на устах
Стремишься с резвою мечтою –
Туда, где темный лес шумит,
С твоей беседуя душою...

Под шепот ласковый ветвей
Ему ты думы поверяешь
И, как свободная волна,
Рокочешь смело и рыдаешь…


* Грезы

В сердце холодно и пусто,
Стынет жар в моей крови.
Одинок я, как в могиле,
Без огня былой любви.

Лишь, порою, чаровницы,
Грезы шепчут по ночам,
Про широкую свободу,
О полете к небесам.

И зовут далеко думы
Заглянуть за облака,
Где в стране полночной блещет
Звездно-млечная река.

Где купается в тумане
Рой бесчисленных богинь,
Им не ведомы печали,
Вкруг их воздух свеж и синь…

Бездны тешат их громами,
Нежит страстью молний взгляд,
Грозы в шуме бурь мятежных
О просторе говорят…

Эльфы светлые пред ними
Пляшут с арфами во мгле,
И летят их звуки тихо,
Замирая на земле…


* В погоне за весною

Целует ветер и ласкает
В лесу кудрявую сосну;
В порыве страстном догоняет,
В ветвях березовых весну.

Гремит кленовыми листами
И гнет осинку до земли,
То, вдруг, застонет над кустами,
Там, где черешни расцвели…

Гоняясь всюду за весною,
Срывает лозы на пути,
Шумит над липой молодою
И шепчет грустно «не найти»!..

«Куда красавица девалась
С теплом и ласкою в очах?
Вот здесь сейчас она смеялась,
Играя в солнечных лучах.

Глазами светлыми манила
И золотистою косой,
Из кроны веток изумрудных
Кропила свежею росой…»

И вновь, за гостьей шаловливой
Спешит под музыку листов,
Где слышен смех ее игривый
Среди ракитовых кустов…


* Памяти матери

Тебя, кормилица родная,
В любви пылающая, мать,
Я, скорбь твою переживая,
Во веки буду вспоминать!..
Всегда краса твоя и сила,
Горели ярким огоньком;
Меня ты с радостью кормила,
Со мной порхала мотыльком.
Порой тебе не спались ночи,
Росли твои заботы вновь…
И стыла кровь, и гасли очи,
А все тепла была любовь…
Она, как солнце, согревала
В метель холодную, зимой,
И никогда не угасала,
Повсюду следуя за мной…


* Не видно солнца

Не видно солнца красного,
Темнеют облака,
И дремлет, крепко скована,
Глубокая река…

Умчались птицы певчие
За теплые моря…
И в серых тучах прячется
Румяная заря…

Угрюмый лес задумался
В венце седых кудрей,
И веет ветер холодом
С поблекнувших полей…

Шуршит сухим кустарником,
К меже былинку гнет,
Уныло свищет в изгородь,
И грусть мне сердце жжет…


* Одинокая

Невыносимые страданья
Кипят на дне моей души –
Затем, как милые лобзанья
С любовью жгли меня в тиши…
На миг один мне в злой разлуке
Тоска покоя не дает,
Ревниво сердце в адской муке
Объятий крепких снова ждет…
Но нет теперь их - друг далеко…
И, жизнь бессильна без него
Мне счастья дать, я – одинока
Без друга сердца моего…
Какие чары он имеет,
Того во век не разгадать…
Но, лишь он взглянет, - сердце млеет, -
Готово все ему отдать…
Увижу ль ласковые глазки?
Прильну ль опять к его устам?
Хотя б во сне, в той чудной сказке –
В волшебных грезах по ночам

* Осенью

Не грусти, что листья под шумом облетели
В злую непогоду осени сырой…
Не тужи, что рано кудри поседели,
Что пришлось проститься с юною порой…

У зимы суровой есть свои красоты, -
Серебром покроет землю и леса…
В старости угрюмой – славные заботы:
Надо жизнь проверить, в чем ее краса…

Заровнять, загладить на пути ухабы, -
Чтоб другим за нами посмелей идти…
И с любовью братьям угнетенным, слабым,
Помогать желанный счастья луч найти…

Не грусти, что силы скоро ослабеют, -
Дух бодрее, крепнет и в груди любовь…
Пусть мороз грозится, руки леденеют,
Лишь бы не остыла в добром сердце кровь…


* Из воспоминаний

Ты помнишь ли, когда с тобой
Случайно встретился под ивой,
Где искрометною волной
Внизу ручей журчал игривый,
И можжевельника кусты,
Пьяня душистою смолою,
Сулили счастье на земле
И радость жизни нам с тобою.
Горел румянец на лице,
Зарею утренней алея,
И шепот нежный замирал…
Ты вся дрожала, пламенея…
И вдруг безумно ты меня
В порыве жгучем, охватила,
Любовь и радостный испуг
Во взоре томном затаила…
Вокруг родные берега
Ковром пушистым зеленели,
И капли светлые росы
В цветах алмазами блестели.
Хотел бы вечно я продлить
Тот миг желанного блаженства;
Казалось солнышко с небес
В лучах разлило совершенство…
Увы!.. прошла пора весны,
И быстро лето увядает;
Срывая желтые листы
Холодный ветер завывает…
Один теперь на берегу
Я с горькой думою своею,
Грущу в разлуке о тебе
И радость прежнюю лелею…


* Порыв

Я весь – порыв! Огонь и слезы
Ключом в груди моей кипят...
И песни гнева и угрозы
Пред силой темною звучат…

В борьбе неравной я упорно
Под тяжким гнетом устоял
И не дал загрязнить позорно
В душе заветный идеал…

Бессильны были все мученья
Мечты желанные затмить:
Успел я в лучшие мгновенья
Их в ум, и в сердце воплотить…

Теперь не страшна темень ночи,
Я верю в будущий рассвет,
Гляжу судьбе смелее в очи,
И страха призрачного нет.

Заглохли мрачные сомненья…
Иду с отвагою вперед;
Огонь святого вдохновенья
Несу в запуганный народ…


* В разлуке

Не для меня твоя краса…
С другим, с другим ты обоймешься,
И в поцелуе с ним сольешься,
Когда ночь скроет небеса…

Хотела ты меня любить,
И волю дать своим желаньям,
Но, лишь одним воспоминаньем
Судьба велит в разлуке жить…

Ты помнишь – ли тот ясный день,
Когда со мною повстречалась, -
В любви заветной мне призналась, -
Как скрыла нас под дубом тень?..

Каким огнем горела ты! –
С мольбой на грудь ко мне припала
В порыве страстном, и шептала
Свои заветные мечты.

Я отдал все, чтоб заглушить
Твои безумные рыданья…
И сам до жгучего страданья
Тебя стал пламенно любить…

Теперь давно в разлуке мы, -
Судьба безжалостно карает…
Но, мне любовь напоминает
Тебя, как луч среди тюрьмы…

Твой образ из моей души
Теперь прогнать ничто не может…
Мне сердце призрак твой тревожит…
Со мной он день и ночь в тиши.

Хотела ты меня любить
И волю дать своим желаньям,
Но, грустным лишь воспоминаньем
Судьба велит в разлуке жить…


* Верная подруга

Он был невинно осужден,
В любви к свободе обличен,
И точно вор сведен в тюрьму;
Лишен был воздуха и света,
И в пищу дан лишь хлеб ему…

Он, словно в склеп, туда вошел…
Никто из близких не пришел
Его в несчастье навестить,
Назвать себя его родными –
Им не хотелось допустить…

И вдруг, - решилася одна,
Ему не дочь и не жена,
И не сестра ему , ни мать,
Толпы злословья презирая,
Придти в тюрьму и приласкать…

Несносна ей его печаль,
В бедняге друга было жаль
Душе отзывчивой, святой…
Пришла, надеждой подкрепила
В борьбе с жестокою судьбой…

И снова дух воспрянул в нем!..
И сердце вспыхнуло огнем!..
А через год – свободным стал…
И, дав любовь за дружбу другу
В семью презревших не попал…

Та ближе стала из родных
К нему, что в скорби чувств святых
Могла в больную душу влить;
Она заставила страдальца
Надеждой в будущее жить!..


* На кладбище

Тихо шепчутся березы
Над могилою родной…
Плачут ивы, наклонившись
Над тяжелою плитой…
Помню, как старушку маму
Здесь давно я схоронил,
И горючими слезами
Холм могильный оросил…
И теперь, с разбитым сердцем,
Снова плакать я пришел, -
Без тебя, моя родная,
Друга в жизни не нашел…
Хоть, порю, закипает
Страсть безумная в крови
Но – бесследно затихает
Без огня святой любви…
И томится, ноет сердце
Одиноко у меня,
Как былиночка трепещет
В холод пасмурного дня…
На могиле здесь отрадней:
Солнце греет грудь мою,
Здесь смелее, чем в народе
Песни грустные пою…
В них с любовью воскресает
Память ласковой души,
И несутся, тают звуки
В замирающей тиши…


* Одинокое сердце

Как солнце тускло за горой
Леса глухие освещает,
Так радость, редкою порой,
Больное сердце навещает…
Знать, ей в груди уж места нет!
Тоска гнездо себе свивает,
Вползает змей, и тяжкий след
На сердце бедном оставляет.
И гложет грусть… Глядишь в окно, -
Когда вдали заря займется?..
За то, - как счастливо оно
При светлой радости забьется!..
Вся кровь в нем вспенится ключом!
И встретив день весны лазурной,
Готово таять под лучом
В живой любви и страсти бурной.


* Весною

Да, мой друг, теперь я ожил, -
Легче дышит грудь моя,
Словно солнышко, согрела –
Ласка теплая твоя…
Снова нежными листами
Лес кудрявый зашумел…
Резвый жаворонок в поле
Песню звонкую запел…
И кукушка зарыдала
В темной чаще на горе,
Где мы в прошлый год слюбились
В летний вечер, на заре…
Все мне здесь напоминает
Твой веселый, милый взгляд,
Затаенную улыбку,
Светло-палевый наряд…
Наши волосы сплетались
В поцелуях на груди…
Улыбалися цветочки
Наклоняясь впереди…
Сердцу сладкую истому
И теперь они сулят,
Дружно шепчут меж собою, -
На лужок к себе манят;
Под кудрявую березу
В заколдованной тиши…
Как и прошлый год, уютно
Здесь, в родной моей глуши…


* Не забывай

Не забывай меня, мой милый,
В своей родимой стороне…
Быть может, ты живешь счастливо? –
Ты вспомни, вспомни обо мне.

Сходи на тот знакомый берег,
К ольхе кудрявой молодой,
Где мы с тобою любовались
Зеленым лесом и рекой;

Где с грустью слушали кукушку,
И ждали песен соловья,
Быть может – там, под шепот веток, -
Любовь припомнится моя…

И, вспомнив, как пылала страстью
К тебе я, голубь сизый мой, -
Пошли ко мне привет сердечный
По речке с быстрою волной…


* Одно утешение

Тогда лишь жизнь во мне проснулась,
Когда ты встретился со мной,
И грудь отрадно всколыхнулась
С твоей любовью неземной.
Твой милый взгляд стрелой певучей
Мне сердце юное пронзил, -
Он страсть зажег в крови кипучей,
И этой страстью опьянил.
Я расцвела с тобою снова,
Как мак душистый полевой,
И все тебе отдать готова
В минуту ласки огневой…
Когда грущу в часы ненастья
И слезы лью в цепях скорбей, -
Один ты мне приносишь счастье,
Даешь покой душе моей.


* Грезы пленника

Лишь ночью темною в забвеньи
Сплетают грезы радость мне…
В минуту грустного томленья
Тебя увидел я во сне:
Склонилась ты – и в миг лобзанья
Твои коснулися меня,
И кроткий взгляд… твои рыданья –
Просили ласки и огня…
Ты мне сказала: «Друг мой милый,
Теперь на веки я твоя»…
Тебе взаимно до могилы
В безмерном счастье клялся я…
А ночь дышала нежной лаской
На вежды сонные мои,
И долго, долго в сладкой сказке
Шептала ты слова любви…
В обьятьях страсти и свободы
Моляся звездным небесам,
Забыв оковы, тюрем своды,
Мы пили жизненный бальзам…
Алел рассвет зари багряной,
Когда растаял сладкий сон…
И, вдруг – один я с тяжкой раной
В тоску немую погружен…
Исчезла ты, виденье рая
С любовью светлой, неземной…
И вторит сердце, замирая,
Что, не со мной ты, - не со мной.


* Сумерки в деревне

Тихо сумерки ложатся…
За окном густеет тень…
Словно крепость, вырастает
За малинником плетень…

За деревней, по оврагу
С рожью стелятся поля…
И, обрызнута росою,
Дремлет тучная земля…

Люди сладко отдыхают
После тяжкого труда…
И на время позабыта
Непосильная нужда…

Шумно бегают подростки…
Птички жмутся у застрех…
Слышно - песни раздаются,
Молодой, задорный смех…


* В погоне за юностью

О, юность, юность, погоди!
Моя звезда не закатилась,
Она сейчас лишь засветилась,
Надежды много впереди!
Еще цветы не отцвели,
Мороз не выжал аромата,
И красотою жизнь богата, -
Лишь только жажду утоли…
Умерь свой бег, и дай скорей
Прильнуть к душистому фиалу…
И поспешим мы к идеалу
Весенней молнии быстрей…


* Невольница

При дороге елочка
Гнулася, шаталася,
В непогоду осенью
На зоре – сломалася…
Сиротинкой девица
Одиноко выросла,
И борьбы, красавица,
С злой судьбой не вынесла…
И досталась старому
Мужу бородатому,
Злому да ревнивому
Богачу пузатому…
И томится бедная,
По ночам вздыхаючи,
Издалека милого
Друга поджидаючи…
Да не быть ей, пленнице,
Снова вольной птицею,
Не летать по рощицам
Быстрою синицею…
Не обняться с молодцем,
С тихой, теплой ласкою,
Все теперь ей прошлое
Только будет сказкою…
И сердечко пылкое
Догорит, состарится;
Лишь тоска останется..


* Без воли

Какая радость быть любимым
Тобою, чистая душа,
Сливаться вместе с сердцем милым,
Одними чувствами дыша!..
Играть волнистою косою,
Ласкать и крепко целовать,
Любуясь пышною красою
Своей подругой называть.
Я чую, - вновь родилась сила
В груди истерзанной молей…
Мой друг, мне все с тобою мило,
И страсть, и гнев души твоей,
И нежный шепот милой речи
В устах медовых у тебя…
Ах!.. Если б воля нашей встречи, -
В объятьях сжег бы я, любя!..
Испил бы я нектар целебный
Красы цветущей, молодой,
И вечно пел бы гимн хвалебный
Душе отзывчивой, святой…


* Родные поля

Милее нет родных полей!..
Как будто солнышко светлей
Лучи над ними проливает,
И ветер, ласковей шепча,
Густую ниву колыхает.

Заросший тиной дремлет пруд…
Вокруг шумит веселый люд,
Сверкая острыми косами…
Несется запах от лугов –
Травою сочною с цветами…

И каждый здесь, куда ни глянь, -
Берет с земли не даром дань:
Он землю потом поливает,
Под зноем солнечным весь день,
Лишь поздней ночью отдыхает…

И стонет труженик всю ночь…
Да, видно, бедному не в мочь
В полях добыть богатство с бою!?
Все также голоден и слеп
Силач, замученный нуждою…

Да, лучше нет родных полей, -
Земля состарит здесь скорей,
Согнет, и в борозду уложит…
Уйдет суровая нужда,
И сон никто не потревожит…


* Любовь прошлому

Возможно-ль прошлое забыть,
О чем давно страдаю?
Луною солнце заменить,
Иль мрачной осенью весну,
Возможно-ль? Я – не знаю…

И чтоб в отчаянии, порой,
В слезах не утопиться, -
Иду послушно за судьбой,
Как лист поблекший липну к ней,
Стараясь позабыться…

Так жизнь влача суровых дней,
Ищу я развлеченья,
Но, - все обман, все холодней…
Тоска вливает в сердце мне
Смертельный яд сомненья…

Пусть горечь в душу тем вношу,
Пусть грудь щемит больнее,
Я прошлого не погашу,
К нему стремится страсть моя
Безумней и сильнее…

Ему – в груди моей огонь!..
А чем живу – постыло…
Лишь думы тайные затронь,
И в миг во сне все оживет,
Что прежде мило было…

Опять мечты взволнуют кровь,
И сердце встрепенется,
И снова к прошлому любовь
Пышнее розы расцветет,
И грустью отзовется…


* Обездоленный

В тревоге сердце замирает
Я с грустью места не найду…
Мне все тебя напоминает
Куда я, друг мой не пойду…

Теснится робкое сознанье
Зачем я смел любить тебя?..
И я в душе таю страданье,
Еще мучительней любя…

Где ты, мое очарованье?
Скажи мне тайну, темный лес!..
Но лес вокруг хранит молчанье,
Лишь слышен гром вдали небес…

И, мнится, - больше не увижу
Твоих любимых карих глаз,
И песен чудных не услышу,
Что здесь весной встречали нас…

Исчезла ты… И все сменилось…
Листы бледнеют на кустах,
С тобой разлука отразилась
Беззвучным шепотом в устах…

Брожу по лесу опьяненный
Красой волшебною твоей…
И призрак твой завороженный
Ношу, как дар, в душе твоей…


* Ночь в лесу

Лес темнеет… Тишь… И только
Ветка хрустнет по ногой…
И навис шатер угрюмый
Над моею головой…
Дальше, в глушь, дорога уже, -
Спотыкаясь по корням,
Я иду совсем на память…
Зная край к деревне там…
Что-то белое мелькает
За осиной впереди…
Я на миг остановился
С сердцем бьющимся в груди: -
Не русаки ль, хороводом
Идут по лугу гулять?..
Что, - как вздумают для смеха
Здесь меня защекотать?
Нет! Березы забелели,
Выдвигая стройный ряд,
Словно призрачные феи
Наклонившися, стоят…
Вдруг луна с небес взглянула,
Тени бледные легли…
Всполохнувшись, закричали
За оврагом журавли…
Словно музыка, листами
Темный лес зашелестел…
И, в испуге, серый филин
Надо мною пролетел…
Жутко мне… А ночи тайна
Приковала на пути, -
С заколдованного места
Мне не хочется уйти…
И опять чаруют взоры
Блики светлые луны,
И в мечты мои вплетают,
Сказки чудной старины…


* Мольба

С тобой воскреснет вдохновенье,
И бури яростный порыв,
Любовь и сладкие волненья,
И в бодром сердце дерзновенный
К свободе пламенный призыв…
С тобою, светлая богиня,
Растет запас духовных сил,
Ты – мой оплот, моя твердыня, -
С тобой не раз в тени древесной
Я радость тихую делил.
Сойди опять ко мне, родная,
В величье дивной красоты!..
Тоскует вновь душа больная,
К тебе с мольбой на встречу рвется,
Таит горячие мечты…
Взмахни же легкими крылами,
Ко мне, мой ангел, наклонись,
Прильни ты жгучими устами,
Дохни в лицо, чтоб звуки песен
В полях родимых понеслись…


* Повтори

Ах, друг, умерь мои страданья,
Залей пожар в моей крови,
Раскрой в душе воспоминанья
Скажи мне сказку о любви…
Ты с нею солнцем засветилась
Среди ликующего дня,
И – вдруг как в тучах, - робко скрылась
В тоске оставила меня…
Скажи о прошлом, как встречали
С тобой восторженно зарю, -
Всему, что мы переживали
Я чувства лучшие дарю…
Такая ль страсть и сладость муки
Кипят безумно у тебя?..
Лелеешь ли, в часы разлуки
Со мною, прошлое – любя?..
Живешь ли тем же обаяньем,
Как прежде - в грезах золотых?
Томишься ль властным ожиданьем
Свиданий радостно-немых?..
О всем, что сердцем пережито
В волшебной сказке говори,
И, если счастье не забыто? –
«Люблю» - скорей мне повтори…


* Юному другу

Скоро, друг, настанут годы
Жгучей страсти и борьбы,
Закипят вокруг невзгоды —
Силы мачехи судьбы!
Приготовься к битве с ними
Добрым сердцем и душой!
Вознесись, мой друг, над ними
С чистой, пламенной мечтой!
И в борьбе, могучей волей
Злой судьбы разбей кумир!
Не мирись с лихою долей,
Пред тобою — светлый мир!
Пусть пугает жизни бремя!
Не тумань слезами взор;
Золотое будет время,
Верь лишь в счастье и простор!
Зацветет красой природа,
И прибавит сил в груди,
И желанная свобода
Свет откроет впереди…
Ой, не гните ветры

Ой, не гните ветры
Под окном рябину!
Унесите дальше
Вы мою кручину.
При дороге, в поле
Там ее развейте
Бурною волною
Из реки залейте,
Присушите зноем
К берегам суровым!

Нашумевшись вволю
Прилетайте снова;
Тихо всколыхните
Красную рябину,
Что б не потревожить
Вдалеке кручину…
Пусть там спит глубоко,
Может, не простенся,
И опять с весельем
Радость в грудь вернется.


* Мужик

Родная деревня,
Знакомый мне пруд.
Смотрите, как наши
Здесь братья живут.
Что шаг, то наука
Мудреная вам,
Изнеженным в холе
Судьбы господам!
С зари до зари тут
Работать спешат,
Ни грязь, ни мозоли
Людей не страшат.
Здесь труженик гору
Подвинет плечом!
Просящий не встретит
Отказа ни в чем.
Забитый нуждою,
Крестьянин бедняк,
Отдаст, по привычке,
Последний пятак!
В мороз обогреет в избе,
Во время напасти
Поможет в борьбе.
Кругом его силы
На деле видны:
В трудах и защите
Родимой страны…
Где гнет и невзгоды —
Повсюду мужик!
Должно быть он честен,
Могуч и велик?


* Заколдованный лес

Лес поредел, валятся листья,
Деревья хмурые стоят,
И оголенными ветвями
Они тихонько шелестят…
Семья лишайников желтеет,
Зубцы раскинув по земле;
Осины прелой пряный запах
Висит в дрожащей полумгле.
Заснуло все… Лишь дятел где-то
Порою, стукнет по сосне,
И снова тишь… И тяжко дышит
Лесь, заколдованный во сне…


* Зима

Все вокруг в природе
Сном могильным спит,
В стройном хороводе
Лес седой стоит.

На деревьях иней
Серебром блестит,
А вдали свод синий
Над землей висит…

И река под снегом
Замерла в тиши;
Лишь метель, набегом,
Прошумит в глуши,

Засвистит в прибрежных
Камышах сухих,
Злобно пылью снежной
Обдавая их.

И растут все выше
Берега кругом,
Расписные крыши
И за домом дом…
Родная картина


* ***

Царит давно глухая осень…
И через лес глядится просинь
Вдали задумчивых небес,
Где сосны красные сплелися,
Раскинув траурный венец…

Снежинки резвыми толпами
Летят, кружася над полями,
И на реке синеет лед,
И ветер свищет над межою,
Траву сухую низко гнет…

В родимой сердцу деревушке
Спешат укутывать избушки.
Несется плач издалека:
В солдаты парня провожают,
И снова холод и тоска…


* Дружба

Елочка зеленая
Стройною тычинкою
Выросла и сблизилась
С горькою осинкою.

Жмутся вместе ласково,
Крепко обнимаются
И парчей серебряной
В холод прокрываются.

Запаслись нарядною,
Теплою обновою,
Смотрят, улыбаются
На метель суровую:

Как она по кустикам
С воем злится, носится,
Как под ветви темные
Приютится просится.


* Апостол

Всю жизнь он с чистою душою
Всегда без страха шел вперед,
И речью пламенной, живою
В родной стране будил народ…

Как солнца лучь под небесами,
Горел его отважный взор.
Он, непреклонный пред врагами,
Терпел насмешки и укор.

Толпа каменьями бросала,
Гоня апостола порой,
И вдруг безумно ликовала,
От речи смелой, громовой…

Он все сносил: толпы глумленье
Делил с восторгом пополам,
И, с тяжким вздохом сожаленья,
За мир молился небесам…

Как светлый праведник, глубоко
За всех врагов своих страдал,
В тоске томился одиноко
И крепость духа призывал,

Чтоб сил хватило и терпенья
Пройти тернистый темный путь.
И для родного поклоненья
Добыть святое что-нибудь,

В родном краю ослабить муки
Толпы забитой, трудовой,
Чтоб не страдали тяжко внуки
От тьмы холодной, вековой.

/Посвящается невольнику/


* Источник

Затих источник говорливый,
Струей кристальной не гремит,
Под слоем льда он крепко спит…
К нему с тоскою гнутся ивы,
И грустно ветер шаловливый
Сухой осокой шелестит…
Давно ли их он, словно друга,
Баюкал нежно по ночам…
Теперь над ним по целым дням
Зимы-затейницы подруга:
В безумной пляске свищет вьюга,
Мосты мостит по берегам…
Пригнул сугроб осоку долу,
И все мороз в хрусталь кует,
Под сень холодную зовет…
И дремлют ивы, снега горы
Вокруг раскинули узоры,
Метель все ниже ивы гнет,
И снится им: утихла буря,
С небес повеяло теплом,
Блеснул источник серебром
И брызнул искрами лазури.
И ивы ветви разогнули,
И слышен говор струй кругом.
Широко волны голубые
Залили снежные поля
И ветер, тихо шевеля,
Целует ветви молодые,
Цветы пестреют золотые,
И в бархат рядится земля.


* Он

Ко всем царила в нем любовь
И вдруг в злодействе обличили…
Пролить его святую кровь
В позорной казни присудили.

Но вера так была полна
Живой любви к святому делу,
Что страшный приговор суда
Спокойно принял он и смело.

Когда взошел на эшафот,
Глаза отвагою горели:
Он знал, что гибнет за народ,
Который вел к заветной цели.

В свой тяжкий век, короткий век,
Он сделал все, что в силах было,
Чтоб был свободен человек…
Сам — ранней жертвой лег в могилу…

Но не могли с ним растоптать
Правдивый гнев живого слова…
Народам будет всем звучать
Оно, могущее, сурово…


* Я увидел весну

Я увидел весну
В золотистых кудрях,
Там за нею бегут
Ручейки на полях.

Собираясь к реке,
Неумолчно гремят
О просторе полей,
О красе говорят.

Всколыхнулась река
От напора их вод,
И под взглядом весны
Разрушается лед…

А вдали, за рекой,
Птичек хоры звенят,
Одевается лес
В свой зеленый наряд.

Дунул ветер с лугов,
В небе тучка прошла,
И под теплым дождем
Озимь вдруг ожила…

А весна все идет
По широким полям,
И, бросая цветы,
Сулит радость людям.


* Памяти С. Я. Надсона

Не умер он! Душа святая
Века в народе будет жить:
Сердца глаголом прожигая
И совесть в людях пробуждая,
Научит правду полюбить!

Придет пора, слова пророка
На почву добрую падут:
Исчезнет зло и власть порока,
И люди братски, без упрека
Друг другу руки подадут.

И мир воскреснет к жизни новой…
Народ с угрюмого чела
Тогда сорвет венец терновый!
Другой сплетет венок лавровый,
Чтоб воля славная была.

Чтоб всем отверглись двери рая,
И радость тихая пришла,
Лучами солнца согревая,
Чтоб юдоль—плача, жизнь земная
Была покойна и светла!


* Светлый гость

Прокрался луч в мою каморку
И властным взором осветил.
Во всем назойливую бедность
Он мимолетно находил…
Убогий стол, сырую плесень,
Лохмотья, мусор на полу…
Бледнеет луч, досады полон,
И вдруг коснулся книг в углу!
Взглянув в открытые страницы,
Прочел с любовью неземной,
И в миг он ласково разлился,
Играя нежной белизной…


* В подвале

Он был и пьяница, и мот.
Играл, кутил и похвалялся,
Что над смиренницей женой
Своею влстью издевался!

Она, «законная» раба,
С святым терпеньем все сносила,
И вдруг коварная судьба
На зло «владыке» подшутила!

Какой-то купчик молодой
Смутил бедняжку для потехи:
Свозил на тройке в ресторан
И дал немножко «на орехи».

И, после сладких, лестных слов,
Оставил гостью на квартире.
Спознавшись с купчиком, она
Души не чаяла в кумире…

Забыв о муже, расцвела
Опять, как лилия весною,
Живая, стройная под час,
С ума сводила всех красою…

Муж долго, злобно ревновал,
Грозил убить, иль сам убиться,
И кончил тем — пришлось совсем
В хмелю кутил с кругу спиться…

Слонялся точно дикий пес,
Голодный, грязный, без призора,
И ночью часто засыпал
В навозной куче у забора…

Прошли цветущие года,
И ей вдруг счастье изменило:
Другую купчик полюбил,
И снова стала жизнь постыла.

Она не вынесла борьбы
Своей озлобленной душою,
И, так же, как погибший муж,
Привыкла к горькому запою…

И так же бродила везде,
Кляня позорное изгнанье,
В грязи, в лохмотьях, босиком,
Прося у встречных подаянья…

И с мужем вновь судьба свела
В углу вонючего подвала!
Взглянули ей в беззубый рот
Глаза голодного шакала…

Но страх пропал минувших лет,
И в ней теперь дышала злоба,
Никто на шаг не отступал,
Как псы, готовы были грызться оба.

В припадке бешеном отмстить
Хоть здесь за жгучую обиду…
И вмиг, смирившись, муж не снес
Жены истерзанного вида:

Сознал вину он в первый раз,
Прочтя в ее глазах презренье,
Сказав: «прости, я виноват»…
И подал руку примеренья.


* Умирающий снег

Леса оделися ковром,
Луга цветут красою мая,
Блестя на солнце серебром,
Ручей гремит, волной играя.
И снег чуть виден за холмом,
На дне оврага умирая.

Забавы жаль ему своей:
Зимы с морозом, прежней силы…
Как в дикой пляске средь полей,
Всему грозил он сном могилы…
Дохнула вдруг весна теплей,
С тех пор в тени лежит он, хилый…

По нем текут потоки слез,
И жадно ржавчина съедает…
Зима ушла, пропал мороз…
И он, забытый, плачет, тает…
Над ним, под шум весенних гроз
Шиповник алый расцветает…


* Голос из тюрьмы

Мой милый ангел черноокий,
Зачем тебя я полюбил?
В любви безумной рок жестокий
Меня к страданью осудил…

Ты миг один пробыть со мною
Теперь возможным не найдешь!
И горьких дум моих с тоскою
В неволе тяжкой не уймешь.

Во сне ль, когда в часы полночи,
В сияньи ль солнечного дня,
О, друг, твои, я вижу, очи
С любовью смотрят на меня…

И кто же нам суровой доли
Клеймо позорное сотрет?
И кто теперь замок неволи
В тюремной двери отопрет

И даст свободу повстречаться
С тобою, добрый ангел, мне
И теплой лаской обменяться
В живой любви, наедине?


* Чует ли сердце

Чует ли сердце твое молодое,
Где-то там сердце таится родное,
Бьется и рвется из мощной груди,
Веря, что счастье его впереди?
Где вы друг друга, когда повстречали?
Грезы ль волшебные, сказка ль, мечта ли?
Бредит красавец тобой наяву:
«Любу когда я своей назову…»
Солнце ли всходит, заря ль догорает,
Все на речном берегу он вздыхает,
Призрак старается твой уловить,
Хочет лелеять и в тайне любить…
Дунет ли ветер, трава ль всколыхнется,
Вдруг его сердце замлеет, займется
Нежною страстью в кипучей крови...
Тихо он шепчет: «я жажду любви.»


* Желанная гостья

Из-за теплых морей
Возвратилась весна,
В золотистых кудрях,
Словно зорька ясна…

Все проснулось вокруг
С благодатной весной:
Где царицей прошла
По опушке лесной,

Всюду птиц голоса
Неумолчно звенят
И деревья в лесах
Ароматы струят.

Зеленеют луга
Точно пышный ковер,
Страстной негой манят
Зачарованный взор…

И, бросая цветы,
Гостья тихо идет,
И в усталую грудь
Счастье-радость несет.
По небу тихо
Плывут облака
И дремлет уныло
У леса река.

Хмуро дымится
Туманная даль,
И в грудь заползает
Невольно печаль.

Вспомнилось время
Далекое мне:
Объятья и ласки
В ночной тишине;

Жгучие взгляды,
Безумный порыв,
И пламенный трепет,
И страсти прилив…

Помню, как радость
Мне сердце зажгла,
Румяной зарею
На миг расцвела.

В душу мне искра
Запала от ней
И вдруг утонула
В печали моей.

Чую лишь, где-то
Таятся на дне
Ревнивые думы
В моей голове.

А светлая радость
Далеко в глуши
Других утешений,
В безмолвной тиши…

Меня же теперь
Сжимает тоска,
Глубока и мрачна,
Как эта река.


* Полдень

Изумрудных полей
Распахнулася даль,
И игривый ручей
Заглушает печаль.
Голубая волна
Льется, неги полна,
Обнимая гранит,
Несмолкая гремит..
И шумит ветерок
В золотых камышах,
И сверкает поток
Серебром в берегах…
А вдали, у ракит,
Заводь тихая спит,
В ясном зеркале вод
Потонул небосвод.
В лучезарном огне
Млеет купол небес,
За ручьем, в стороне
Залит солнышком лес;
Он поет и звенит,
Песней в чащу манит,
Под зеленую сень,
Где прохлада и тень.


* К родной могиле

Не знаю сам, какие силы
Меня влекут на те места,
Где зеленеет холм могилы,
Вблизи ольхового куста?..

Где соловей поет так нежно
В ветвях над холмиком родным,
Каким – то роком неизбжным
И он знать, крепко связан с ним? –

Быть может, много раз летали
Сюда с подружкою вдвоем,
Счастливо лето проживали
В уютном гнездышке своем? –

Теперь, как я, в стране далекой,
Быть может, он осиротел,
О милой спет в тоске глубокой, СПЕЛ?
Один на холмик прилетел,

И мне напомнил про былое,
Что я давно похоронил,
И чувство грустное, больное
Опять он в сердце разбудил…



* Увядшие цветы

Ох, как жестоко отплатила
Ты мне за ласку и любовь,
Коварным взором охладила
В груди пылающую кровь!..

Тебе хотелося страданье
В душе затронуть у меня,
И вызвать тяжкие рыданья
При встрече памятного дня…

Поверь мне – клясть сама ты станешь
Свое безумство в черный день,
Невольно в горестях вспомянешь
Мою угаснувшую тень;

Придешь к могиле приклониться,
Сорвешь увядшие цветы…
Но, счастью вновь не возвратиться:
Его со мной схоронишь ты…


* Утешительница

Много дум тяжелых
На душе таилось,
И из глаз навеки
Радость схоронилась…

Рвется, ноет сердце
От тоски и горя,
И в груди скопилось
Слез кипучих море;

Да весна подкралась
С лучезарным светом,
И теплом дохнула
С лаской и приветом;

Каждую былинку
Ветер тихо будит
И лучами солнце
Греет и голубит;

У реки покрылись
Берега цветами,
Всколыхнулись рощи
И шумят ветвями;

Зеленеет озимь
На полях широких;
Небо отразилось
В омутах глубоких;

Звонко раздается
Жаворонков пенье,
И песни веет
В сердце утешеньем.

И в лицо мне свежим
Ветерком пахнуло,
И былую силу
В грудь мою вдохнуло…

Притаилось горе
И обсохли слезы,
И манят и нежат
Ароматом розы



* ***

Ах, зачем ты уродилась
Так добра и хороша?..
По тебе, мой друг, в разлуке
Исстрадалася душа…

Не дождусь я тайной встречи,
Чтоб прильнуть к твоей груди…
Под кусточек, - где слюбились,
На минуточку приди… -

И в последний раз, голубка,
Обойми и приласкай!.. (ОБНИМИ?)
Успокой… С тяжелым сердцем
От себя не отпускай…


* Мы шли

Едва заметною тропою
Мы шли тогда, рука с рукою,
И сердце млело у меня,
И дружба тесная с тобою
Просила ласки и огня…

Я помню этот полдень знойный:
Желанья рвались беспокойно
Со мною радость разделить,
Но, сердце робость сжала больно, -
Готова пламя погасить…

А мир свободно дышит, манит
Туда, где солнцем берег залит,
В кусах у вьющейся реки…
И кровь мне голову туманит…
Я пальцы жму твоей руки.

Мое смущенье замечая,
И ты, на ласку отвечая –
Тихонько шепчешь: - «Милый мой, -
Смотри, как ветками качая,
Играет ветер гулевой».

И, вспыхнув вся, ко мне склонилась,
И сердце радостно забилось
В груди высокой у тебя…
И в поцелуе жгучем слилось
Со мною ты, - меня любя.


* Мгновенье

Бывают тяжкие мгновенья:
В немой тоске готов сгореть,
В безумном страшном исступленьи
Закрыть глаза и умереть…

Тогда бежишь, как зачумленный,
Чтоб скрыться дальше от людей,
И шепчешь, точно опьяненный,
Ведь не злодей я, не злодей…

Но – поглядишь – вокруг приволье: -
Родной простор, кудрявый лес,
Луга цветут, краса, раздолье…
Лазурь реки и даль небес…

И в миг пройдет тоска немая,
И легче, радостнее вздохнешь,
Природу взором обнимая,
Душою снова оживешь…

А песня жаворонков льется,
Звенит в небесной высоте…
И жажда вдруг опять проснется
К любви, к добру и красоте…


* Весна

Вот идет она царица,
Благовестница весна!
Величава, светлолица,
Синеока и ясна…
Гонит снег последний хилый,
С ним тлетворный злой недуг,
Людям в грудь вливает силы,
Поднимает павший дух…
Время радостей настало, -
Все трепещет и живет…
Солнце ярче заиграло
И лучи отрадней льет…
Зазвенела песнь в природе,
Славословя дар Творца,
Из глуши будя к свободе
Усыпленные сердца…


* Совет

Когда прославить хочешь друга,
Старайся душу в нем узнать…
В минуты общего недуга
Чтоб ложь за правду не сказать…

И тех друзей цени дороже
Между житейских бурь и гроз,
Глазами кто заглянет строже
В пучину зла народных слез.

В ком сердце вылито из стали
Для мщенья подлости людской,
Кто братьев ведает печали,
И в душу им несет покой…

В таких друзьях, любовь иная,
Добро их вечно не умрет…
Они отворят двери рая,
Они спасут родной народ…


* Перед пленом

Когда ты, друг мой, трепетала
В объятьях пылких у меня,
То счастье светлое пропало,
Погас последний отблеск дня…

Теперь вдали я, лишь желаньем,
Мечтой лечу, мой друг, к тебе,
И шлю, измученный страданьем,
Проклятья горькие судьбе…

Зачем злодейка разлучила
И яд тоски влила мне кровь?
Гнетет, как тесная могила,
Страшит тяжелой мукой вновь…

Угнав от ласк, красот природы
Она спешит меня лишить: -
Отнять простор, отнять свободу
И луч надежды погасить…

Не жизнь сулит она с родными,
А серый, тесный каземат,
За то, что свой был меж чужими,
И друг, и недруг был мне брат…

Судьба заставила склониться
И пасть безропотно в борьбе,
Под гнетом дум душой томиться,
Вдали страдая о тебе…


* ***

У омута, в глуши лесной,
Нависла старая плотина,
Склонившись к ней, - переплелись
Ивняк и серая крушина…

Они ревниво сторожат
Минувших лет седую тайну:...
Как, будто б, жгучий поцелуй
Вчера подслушали случайно!..

Когда–то мельница была
Здесь под крутыми берегами, -
Старик – кудесник ворожбой
Сводил красоток с женихами…

Теперь зеленые кусты
Свиданья их напоминают,
Шепча о страсти молодой
В душе былое пробуждают…


* Безвременье

Урожай родимой нивы
Повилика обвила…
Ложь коварная в народе
Черной немочью прошла:
Яд разлит, порывы глохнут,
Затаенна месть в груди,
На любовь, добро и правду
Нет надежды впереди…
Тучи темные нависли,
Солнце спряталось во мгле,
Радость светлая погасла
Утопая в черном зле;
Все корысть, людская жадность
Поглощают на пути,
С добрым сердцем человека –
Диво - дивное найти!..
Честь за деньги продается,
Стыд в подполье убежал,
И кичится всюду ложью
Лишь отъявленный нахал…


* У дороги

Словно сестры здесь сошлися
Две березы и сплелися,
Широко раскинув сень,
И зовут, шепча ветвями,
Под свою густую тень.

Много бедности убогой
Мимо их несут дорогой,
Выбиваяся из сил,
И прохожий, в жаркий полдень,
Здесь приют свой находил.

Иногда, под шепот тайный
Ихней повести печальной,
Путник сладко засыпал,-
В золотых свободных грезах
Гнет и горе забывал.

И березы оживали:
Нежно, ласково шептали
Наклоняяся над ним,
И беседовали тихо
Словно с другом дорогим…


* ***

Ниже, ниже наклоняйтесь
Серебристые березы,
Чтобы люди не видали
На моих ресницах слезы;
Чтобы с милым про свиданье
Я одна бы только знала…
Утони же, глубже, тайна
Там, где вишня расцветала,
Где цветы ее под нами
Белым снегом рассыпались,
И листы зеленой груши
Тихо, ласково шептались.
Зарастай моя дорожка,
Где я с милым проходила,
Поломалася малина,
Что весною посадила,
Оклевали птицы вишню
И черемуху густую…
Грустно мне взглянуть под вязы
На скамеечку пустую!..
Солнце спряталось за тучку,
Мглой мой садик одевает,
И холодный, резкий ветер
Листья желтые срывает…


* С тобою

Как жизнь свою тебя люблю,
Подруга мне сказала,
И знай, в тебе не разлюблю
Того, о чем мечтала…

С твоей отзывчивой душой
Я с детства породнилась,
Мечтою реяла с тобой,
И искренно молилась.

За то, чтоб тяжкая судьба
Удар не наносила,
И чтоб жестокая борьба
Удар не наносила,
И чтоб жестокая борьба
Любви не угасила…

Твоя любовь, - вот мой оплот,
Мое святое счастье!..
Я в ней забылась от невзгод,
Сурового ненастья…

С тобою, ласковый мой друг,
Идя путем свободы,
Я поборола все вокруг:
И горе, и невзгоды;

Они не страшны больше мне,
Среди житейской бури,
С тобой, мой друг, наедине
Я вижу даль лазури…


* Грустная дружба

Белоствольная березка,
С нею тополь серебристый,
Наклонившись к старой хате
Заплели узор тенистый.
Сквозь побеги молодые
Солнца луч им улыбался,
И их чудный, нежный шепот
В хате бедной раздавался…
Но, прошли весна и лето,
Непогода вдруг подула,
Неприветливо, сердито
Ветки гибкие рванула…
Гнется к тополю березка,
Друга крепко обнимает,
И на крышу ветхой хаты
Капли слез она роняет;
Мнится ей, что скоро, скоро
Холод утренний осенний
Налетит с морозом жгучим,
Оборвет наряд весенний…
И поникнет стройный тополь
Головою горделивой,
И застынет, в жутком страхе,
С думой грустной, молчаливой…
Не шуметь и ей отрадно
Оголенными ветвями,
Не баюкать в хате бедной
Деток радужными снами…

* ВАСЯ

Все ребята на лужайке
           У сарая, за двором,
Только Вася засиделся
           В душной хате с букварем.
Очень хочется мальчишке
           С ребятишками играть,
Да привык вперед уроки
           Поутру он прочитать.
«Кончил дело, гуляй смело»,-
            Видит Вася в букваре
И, убрав на полку книгу,
            Мигом скрылся во дворе.
Из-за верб густых на Васю
            Утро ласково глядит,
Невидимка-жаворонок
             В поле песенкой манит.
Услыхав привет сынишке
             В звонких детских голосах,
Из калитки выбегает
             Мама с радостью в глазах.
 К грамотею со всей мочи
             Сестры, братья понеслись,
Их ручонки в шаловливом
            Хороводе заплелись:
За собою тянут Васю
            По лужайке пробежать,
А с небес высоких солнце
            Изливает благодать.


* ГОЛУБИ

В тесной клетке голубок
                 Жалобно воркует,
А голубка сиротой
                  Над гнездом  тоскует.
Он искал своим птенцам
                  Зерен в чистом поле,
Вдруг запутался в силках
                  И сидит в неволе.
А она с детьми о нем,
                  Бедная, страдает.
Кормит птенчиков одна,
                  Холит, утешает:
«Вырастайте поскорей,
                  Улетайте в поле
И найдите там отца
                   Взаперти, в неволе!»
Оперились голубки,
                   Весело вспорхнули,
В воздух ринулись и вдруг
                   В небе утонули.
Увидав их, мальчуган
                   Живо догадался:
«Приманю их к голубю,
                   Что в силки попался!»
Терем с голубем на двор
                    Выставил и манит...
Терем с дверцей отпертой
                    Солнцем ярким залит,
В нем рассыпано пшено,
                    Чашечка с водою...
Только не расстаться птицам
                     С волей золотою!
Пленник взвился к сыновьям
                     Вмиг из клетки тесной,
И взлетела вся семья
                     К синеве небесной!


* К БОЛЬНОЙ

Расшумись ты, ветер буйный!
        Заглуши мою печаль!
Ты, свободный и разгульный,
        Улети с печалью вдаль,
Через реки, через горы,
        В город шумный и большой,
Заведи там разговоры
        С одинокою больной.
Всколыхни чуть занавеску
        Над подушкою, любя,
И Она с улыбкой детской
        Встретит ласково тебя.
Поцелуй больные вежды
        И отраду в грудь вдохни,
Чтобы светлые надежды
        Снова дали счастья дни.
Вера все преодолеет
         И терпеньем наградит,
Радость снова грудь согреет,
         Злую немочь победит


* МИЛЫЙ ДРУГ

                         Вдали от шума и страстей,
       В ночи, туманной, молчаливой,
                          Опять я с музою своей
       Сам-друг беседую под ивой.
                          Ничто желанных, чутких дум
        В глухой тиши не нарушает:
                          Чарует муза скорбный ум
        И нежной песней утешает.
                           И млеет сердце от любви
        И чувств святого вдохновенья.
                           Пожар затих в моей крови
        В блаженный час уединенья.
                           Дошла до Господа мольба
         Души тоскующей средь ночи:
                           Забыта страстная борьба
         Забыты пламенные очи,
                           Что так мучительно зажгли
          И полонили сердце страстью.
                           Да! Дни безумные прошли
        И новый путь нашел я к счастью!
                           В напеве чудном звонких струн


* Опять один я... (стр 92)

Хотел бы знать, что дальше будет,
Чтоб грозный рок свой разгадать!
Быть может лучший друг забудет?..
Быть может, счастья не видать?
Опять один я. Тьма глухая
Меня окутала кругом…
Бушует, свищет вьюга злая,
Резвясь, на воле за окном…
И я, как узник кандалами,
Окован мрачною тоской:
В разлуке с милыми друзьями
Страдаю сердцем и душой…
Увяла прелесть увлеченья,
Как в осень поздние цветы,
И мысли, точно в заточеньи, -
Мертвы без дел и красоты…
Гнетут со всех сторон невзгоды,
Теснят больную грудь мою…
Поклонник правды и свободы, -
Давно им песен не пою…
Нужда и волю заковала,
Святую радость отняла…
Ты любишь честь – она сказала
И, как змея, вдруг обняла…
Кружась, метель в окно стучится,
Вползает инеем мороз,
И холод на сердце ложится;
Глаза слипаются от слез…
Хоронит волю вьюга злая,
Хоронит счастие мое…
И, глаз усталых не смыкая,
С тоскою слушаю ее.


* Воля (стр 93)
(завет другу)

Чуешь, как здесь, в поле,
Дышет ровно грудь?
То-то, брат, о воле,
Помни, не забудь!..
Ей не ставь преграды
На пути своем,
Лучшей ты награды
Не найдешь ни в чем…
В жизни с ложью черной
В сделку не вступай,
В душу так позорно
Грязью не бросай…
Совесть, крепче стали
С детства закаляй,
С нею без печали
Где пришлось, гуляй…
Чтобы сердце чисто
Было у тебя, -
Правду режь речисто,
А молись – любя.
Подойди с советом
В горе к бедняку,
И согрей приветом
Душу старику…
Добрых дел на воле
Непочатый край,
С волей в горькой доле
Можно сделать рай!..

С волей все возможно!
Не жалей лишь сил…
Если жил тревожно, -
Знай, не даром жил!


* Расписной положок (стр 94)

Ах, придется ль когда
Мне тебя приласкать?..
Мне к другому теперь
Тяжело привыкать.

Помнишь,  как горячо
В положке обняла?..
Свою душу тебе
С той поры отдала…

Ни запить, ни заесть
Поцелуи твои,
Глубоко у меня
Они в сердце легли…

Примирилась с другим,
Покорилась судьбе,
Но, желанья мои
Улетают к тебе!

В расписной положок,
На подушку твою,
Где ты в сладостном сне,
Нежишь душу мою…


* В родную глушь (стр 95)

На чуждой стороне
В тоске томится грудь…
Бывает жутко мне
И в душу заглянуть…
Одни проклятья в ней
С невольным злом растут,
Следы суровых дней
Усталый ум гнетут…
От тягостных работ
Порой спина трещит,
Но жизненных забот
Ничто не облегчит.
Здесь негде отдохнуть,
Душе здесь – вечный плен…
И сгинешь, где-нибудь,
Среди фабричных стен…
Живу я лишь мечтой:
Когда заря блеснет
Уйти в свой край родной, -
Там легче грудь вздохнет.
И смело взявши плуг
На пашне, за рекой,
Скорей найдешь вокруг
Свободу и покой.


* ОДИНОКИЙ

Кручина грудь мою терзала,
Когда к другому ты ушла,
И что так свято сберегала,
Ему невольно отдала...

А мне оставила страданье
В больной, истерзанной груди,
Среди позорного изгнанья,
Без веры светлой впереди...

Я уличен в глазах народа
За то, что волю полюбил,-
Свободный сын родной природы
Судом людским не дорожил...

Попал в тюрьму я, одинокий,
За правду смелую свою!..
Знать, жребий выпал мне жестокий
Погибнуть здесь, в чужом краю...

Отец твой, жадностью объятый,
Запродал, бедную, тебя;
И тешится супруг богатый,
Красу девичью погубя...


* МАТЬ

Один твой сын учил детей,
Другой был резвым для затей.
Таких в округе поискать!..
И вдруг – судьба толкнула их
Под пули стать врагов лихих
И сиротой оставить мать...


Один был взят во цвете лет,
Другой хотел увидеть свет.
Совсем мальчонкой молодым...
Захвачен страшною войной,
Ушел страдать за край родной,
Туда, где смерть и адский дым.

«Не знаю, что и как сказать,
Усопшими их поминать?..
О, Боже мой, Святой Отец!..
От горести изныла грудь...
Хотя бы слух какой-нибудь
Узнала я... Один конец...

Помилуй их, спаси, Христос!
Ты также крест тяжелый нес,
На нем и умер за людей...
О, Боже правый, дай мне сил!
Мой старший сын добру учил.
И младший был – не лиходей...

Но... Вдруг громкий клич: Война, война!
Объята пламенем страна...
Ушли сыны громить врагов.
Осталась я теперь одна,
Душа мольбы и слез полна
За кровных ясных соколов.

И ночь мне не закроет глаз:
Что ждет в страданье горьких нас?
Змеей гнездится боль в груди.
Молю усердно, чтобы Бог
Скорей сломить врагов помог;
Быть может, радость впереди?..


* стр 99  МУЖИК

Вечно суровую, рабскую долю
Тянет забитый, голодный мужик.
С детства узнал он нужду и неволю,
С детства к терпенью покорно привык.
Вечно идет он тяжелой дорогой,
Потом он пашню чужую поит.
Труженик честный, обиды он много
Сколько уж лет в своем сердце таит
Грудью стоит он за землю родную,
Первая жертва в кровавом бою,-
Он ли не выстрадал долю иную,
Право на землю




* Не ходи… (стр. 104)

Не ходи, нужда, за мною,
По моим следам –
Снег растает, я весною
Все тебе отдам:
Закопченную лачугу
И чердак пустой,
Где зимой я слушал вьюгу
За трубой худой.
Все возьми: мои онучи,
Рваный мой озям…
Я пойду туда, за кручи,
К солнцу, к небесам…
Где бедняге будет можно
Воздухом дышать,
Где не надо жизнью ложной
Душу омрачать…


* Муза (стр 108)

О, Муза скорбная моя! –
Взмахни усталыми крылами,
Лети в родимые поля,
И пой над милыми холмами,
Где семьи жалких бедняков
В лачугах рабски изнывают,
В ярме заржавленных оков
Свой век тяжелый проживают!..
Судьбы позорной произвол
Для них связал искусно сети,
Чтобы бедняк смиренно шел,
И шли за ним покорно дети;
И чтоб он сытого питал,
А дети вечно б голодали…
И чтоб без ропота страдал!
Блаженства рая обещали
(Тот рай, что мертвым отдают
Жрецы в минуту погребенья),
А сами с бедного берут
Последний грош без сожаленья.
Чтоб сладко жить, спокойно спать,
Толпа льстецов, душой отсталых,
Готова крест последний взять
С рабов голодных и усталых…
О, Муза скорбная моя!
Взмахни бесшумными крылами, -
Пусть песня горькая твоя
Звучит над милыми холмами!..


* Нет, не хочу… (стр. 109)

Надо ль коварной судьбе покоряться?..
Разве лишь рабство дано мне в удел?
- Нет!.. не хочу больше бури бояться, -
Вырву тот страх, чем я с детства болел

Высушил грудь и усталые руки, …
Жизнь не хочу я в цепях потерять,
Вынес я много обмана и муки, -
Ложь вековую пора мне понять…

Сбросить поря мне оковы насилья,
С ними мириться теперь не хочу.
И, развернув свои сильные крылья,
Вольною птицей я в высь полечу…


* Бедняк (стр. 110)

Вот дождался бедняк и весны золотой,
А в душе у него так же все непокойно:
Надо хлеба купить по цене дорогой,
Но в мошне ни гроша: - затоскуешь невольно.

Не услышит он птиц, что на воле поют,
По осинкам в лесу и березкам порхая…
Крик и рев на селе, - скот последний берут
У крестьян, за долги, со двора угоняя…

Всюду злая нужда скалит зубы свои,
Мрачной тучей гроза яркий свет закрывает…
Только слышит бедняк брань да стоны семьи,
Слез в груди у него океан закипает…

Убежать? Да куда от неволи уйдет?
Не уйдут от нужды исхудалые ноги…
А свободы бедняк вряд ли где и найдет.
Хоть не ешь и не пей а, плати, брат, налоги…


* Больной век (стр. 112)

Пройдет зима, и шумно воды
С полей родимых потекут,
В струях желанный дар свободы
Лугам и рощам принесут.
Спадут оковы снеговые
С ручьев гремучих, с быстрых рек.
И снова долы вековые
В красе увидит человек…
Но, чувство робкое подскажет:
 «То мир чудес не для меня»…
Душа в тоске на зло укажет,
Больной, суровый век кляня…
Цветет природа молодая,
Суля отраду впереди, -
А мы живем, изнемогая,
С тоскою жгучею в груди…
Рабами нас рабы родили,
Без духа пламенной борьбы,
И с детства ложный страх вселили
Перед величием судьбы…
В цепях неволи век томиться,
Всю  жизнь безропотно страдать,
Слепыми быть, слепцам молиться,
И не расцветши – увядать…


* Родная скорбь (стр. 115)

Родная скорбь, родные слезы,
Туман осенний и морозы
Все также, хмурые, висят,
И зиму лютую с метелью
Унылой родине сулят.
Бредет за клячею худою
Угрюмый пахарь бороздою,
Бедняк, покорный раб труда.
Ему – все прежняя неволя,
И подать – первая нужда…




Далее стихи из переписки РОСЛИТАРХИВА


* Новая пашня

Из газеты «Красный пахарь» 29 октября 1928 года

По деревне разметались
Девки резвою гурьбой
С криком… «Эвон, запахали,
За Ивановой избой…»

В самом деле – трактор пашет
В два отвала позади…
Тракторист веселый Саша
Лихо правит впереди…

По всему мирскому «кону»
Межи роет поперек
Под весенний шум и звоны…
Ну, и выдался денек.

Прибежали бабы в поле
На диковинку взглянуть…
Так и хочется на воле
Хором песню затянуть!..

С словом выступил Григорий,
Поводя худым плечом…
- И Никола и Егорий,
Все поверья ни при чем!..
С решета крапить не надо,
Значит, общего коня, -
В поле ж – радовать нам взгляды
Будут вместе зеленя!

Вместе скосим, обмолотим…
Может станет «по душам»
Дом общественный «сколотим»!,
Жизнь в коммуне хороша! –

Если все с умом да с ладом,
НЕ введем себя в изъян! –
И обвел довольным взглядом,
Чутко слушавших крестьян.

С голубого неба льются
Разомлевшие лучи…
Над сырою пашней вьются
С дружным говором грачи…

А вдали глубоко пашет
Неустанно конь стальной
Комсомолец правит, - Саша
Делу общему родной!..


* ВЕНОК НА МОГИЛУ крестьянского писателя С.Т. Семенова


Схоронили друзья без попов...
Там, где озимь его трудовая
Будет нежно шептаться без слов,
Тучный колос к могиле склоняя.

Налетит ли шалун-ветерок,
Или туча с грозой пронесется,
Снова в тихий к нему уголок
Взглянет солнце и вдруг - улыбнется.

Точно знамя - что пахарь любил
Труд и нивы полоску родную,
Выбиваясь на пашне из сил,-
И согреет могилу сырую.

Крестьянин Е.К.Кузьмичев


* ВСХОДЫ


Пасмурная осень
     Все вокруг немеет
И полей широких
     Красота бледнеет.

Потянуло грустью,
     Злобно ветры веют,
Только всходы шелком
     Ярче зеленеют.

Им не страшен холод
     Зимней вьюжной бури.
В юных светлых грезах
     Видят край лазури.

Обогреет солнце
     И весна разбудит.
Хорошо расти им
     Под лучами будет:

Колоситься, рдется
     Наливаться тучно,
Жаворонка слушать
     С трелью сладкозвучной.

Егор Кузьмич Кузьмичев – крестьянский поэт


* Веснянка
Проснулась жизнь, любовью дышит,
    По жилам кровь быстрей бежит…
Зеленый лес листвой колышет,
   Под пенье птиц вокруг шумит…
Стада пасутся… И в долине (вниз долины)
    Ручей извилистый гремит,
Таясь в тени густой калины,
    Веснянка юная стоит..,
Прекрасный взгляд ее, лукавый,
    Скользит с улыбкой по ручью…
Вдали рыбак взошел на лавы,
    И чинит легкую ладью…
На миг мелькнула грудью белой
    Краса веснянки в камышах,
И, поплыла как лебедь, смело,
    Играя брызгами в волнах;
И лишь пестреет золотистый
    Наряд красавицы в лугах…
Склонился ландыш серебристый
    К фиалке бархатной в кустах…
Шепча любовно лепестками,
    Фиалка ловит страстный взгляд…
Сплелися нежно стебельками,
    И льют душистый  аромат…
(до 1917 года)

Егор Кузьмич Кузьмичев


* Любовь (1899 г)

Любовь!.. Как жгучи эти звуки!
От них огонь горит в груди,
И утихают злые муки,
И светит счастье впереди…

Любовь как солнышко в лазури
По небу синему плывет,
Порой среди страстей и бури
С улыбкой тихо подойдет.

Обнимет крепко, безмятежно,
Истомой сладкой обожжет,
И в поцелуе, долгом, нежном,
Нектаром сердце обольет.

Александр Павлович Кузьмичев
Любовь (конец 20 века)

Опять любовь, опять надежды!
Я целовал святые вежды,
И озаренный потолок
Мне помогал, как только мог.

Она ласкать себя давала,
Но… тихо кулачком толкала,
Толкала тихо кулачком,
Пока не вытолкала вон.

И в этом нахожу я сходство
С высоким свойством естества -
Такая степень благородства,
Такая сила волшебства!

Да! Целовать ее – целую,
А мне кажется, - ворую –
Листки льда с ланит сорвал,
Но, сдается мне, - украл.

Губы в губы я кладу –
Нет сомнения – краду.


* МЫ

Мы душою пламенеем,
Алой радостью горим,
От восторга мы пьянеем,
Что свободно говорим.

Что свободно, вольно дышим
В вихре солнечных лучей,
Крика дерзкого не слышим
Кровожадных палачей.

Цепи ржавые разбиты,
В прах поверженная власть.
Мощь народная разлита.
Как вулкан бушует страсть!

Прочь, трусливые, с дороги!
Время смелых впереди!
Места нет больной тревоге
С верой пламенной в груди.

С нею все преодолеем,
Мир коварный победим,
Правду светлую взлелеем,
Царство братства создадим!


* МЫСЛИ

Пусть мысли также будут чисты,
          Как снег белеет за окном,
Резвы как молния, лучисты
          Как грань алмаза пред огнем.

Пускай оденутся в порфиру
          Весны цветущей золотой.
В лучах на славу всему миру
          Заблещут светлой красотой.

Летят с седыми облаками
          В лазурь далекую небес,
Весной с зелеными лесами
          Плетут узорчатый навес.

Плывут за легкими ладьями,
          Кружась и пеняся в реке.
И пусть над звонкими ручьями
          Как лебедь вьются вдалеке.

Пусть реют в пламенном эфире,
          Гоня усталость темных вежд,
Чтоб с новой силой в грешном мире
          Искать спасительных надежд...


* НА СМЕРТЬ С.Т. СЕМЕНОВА

В пылу бушующего зверства
     У темных сил пощады нет!
В хаосе диком изуверства
     Семенов пал во цвете лет.

Коварный враг добра и света -
     Клевет незримая змея -
Не пощадила в нем поэта -
     Нашлась преступная семья.

С винтовкой бросилась, с наганом!
И честный гражданин угас,
Как солнце вдруг, — лучом багряным
Вдали блеснув последний раз...

Но после жертв — сильнее всходы!
Ведь ночь не вечна, тьма пройдет.
Уже горит заря свободы,
И солнце красное взойдет!

Повсюду светлый дух воспрянет,
И с ним Семенов будет жив!
А злых убийц народ проклянет,
Позором черным заклеймив!..


* РАБОТНИЦА ЗА ЛЬНОМ

Тяжела со льном забота,
А еще трудней работа
          Выбирать, стелить и мять.
Потемнеешь вся от пота
          День и ночь его трепать.

По зорям спешишь, молотишь,
Горы пыли с ним проглотишь
          В руки тысячи заноз
И мозолей наколотишь,
          Молотя за возом воз.

Что не взмах, то пыли боле,
И закашлешь поневоле,
          Изнурясь в шальном чаду.
Но забыть о лучшей доле
          Не забудешь и в аду.

Вспомнишь, есть на свете люди -
Точно яблоко на блюде,
          Холят белое плечо,
И сильней забъется в груди,
          Вспыхнув сердце горячо!

У людей тех роза в щечках,
Им напомнит ли сорочка
          О мучительном труде?-
Как за льном не спится ночка
          Нам, работницам, в нужде...

Егор Кузьмич Кузьмичев – крестьянский поэт


* Я увидел весну

Я увидел весну
     В золотистых кудрях,
Там за нею бегут
     Ручейки на полях;
Собираясь к реке,
     Неумолчно гремят,
О просторе полей,
     О красе говорят.
Всколыхнулась река
     От напора воды,
И под взглядом весны
     Разрушаются льды…
А вдали, за рекой,
     Птичек хоры звенят;
Одевается лес
     В свой зеленый наряд.
Дунул ветер с лугов,
     В небе тучка прошла,
И под теплым дождем
     Озимь вдруг ожила…
А весна все идет все идет
     По широким полям,
И, бросая цветы,
     Сулит радость людям.

(до 1917 года)


* Красноармейская

Рукопись
Из семейного архива Кузьмичевых

Уже множество лет
Ленинизма завет
    Гимн поем
    Интернационала!
С революцией в шаг,
Средь заводов и шахт
    Мировая звезда
    Запылала!..
На под'еме вперед
С нами Сталин - оплот
    Красной армии -
    Страна Советов...
С нами вождь ВКП-е
Боевой комитет!
    Крепче спайки такой
    Нигде нету...
Не боясь ничего,
Мы умрем за немго!
    Уничтожив
    Кулацкую свору,
Не страшна смерть в боях...
На колхозных полях
    Не забудут
    Стальную опору...
Слава наша растет!
Точно мак стяг цветет
    Красной Армии,
    Социализма...
Все о нас говорит!
Сердце наше горит
    Мировою звездой
    Большевизма!..


* ***

Рукопись из домашнего архива Кузьмичевых

    Преследующим из-за угла
    клеветникам-подкулачникам

Нет! Я не с вами в серой стае,
    Друзья мне там, где красный флаг
В руках рабочих полыхает
    И их гигантский виден шаг,
Где дружны с фабрикой колхозы,
    Горит над Лениным звезда,
Скрипят полозьями обозы,
    Несутся с хлебом поезда...
А вам позор и вашим сплетням!
    Взгляните в старый черновик -
Мой труд подскажет многолетний,
    Что я природный большевик!
За большевизм страдал я многолетний
    Царизмом яростно гоним...
С тех пор, как смело проклял бога,
    Я вами стал неуязвим!..
В борьбе почуяв сил избыток
    Мой голос громче закричит
На вас в злой миг душевных пыток, -
    Вы слуги белых, - палачи!
Вредители вы, лицемеры!
    Вы тайно рветесь заглшить
Энтузиазм глубокой веры,
    Чем пыл селькора хочет жить!..
Ваш яд клевет изсякнет скоро,
    Заставят злой язык прижать...
Когда сильней перо селькора
    Коварство будет обнажать,
Затихните как свуки лая
    Трусливых псов из-за угла!
Дорога для меня прямая
    К социализму пролегла.
Я за вторую пятилетку,
    За темпы стройки, за колхоз!
Врагам под'ема целюсь метко
    Не в бровь, а в глаз до едких слез!..
Из-под трахомы черных сплетен
    Взгляните в давний черновик,
Мой труд подскажет многолетний,
    Что я дозорный большевик!
Я знаю вас как волчью стаю...
    Мои друзья - где красный флаг
В руках рабочих полыхает,
    И крепнет большевитский шаг!..

    Старый селькор.

* Накануне

Рукопись из домашнего архива Кузьмичевых

В снегу лежит немой простор!..
    И на реке броня крепка...
    Бегущих струй издалека
Безсилен, слой - на лед напор...
Но, день о то дня ропот их
    В тюрьме становится грозней!..
    Набег один еще дружней, -
И, дико вспенется разлив...
Броню собьет зимы лихой!..
    Вздохнет свободная земля...
    В шелку - раскинтся поля...
И лес оденется листвой...

Е. Кузьмичев


* Начало весны

Рукопись
Из домашнего архива Кузьмичевых

В чаще кукушка не кукует...
    Во сне лепечет голый лес...
Но по опушкам на полянах
Тетеревья давно токуют
И в зорях утренних-румяных
    Весною дышит край небес...

Поля пестреют от проталин
    В овраге искрится хруталь...
Глядится солнышко с любовью
Между березовых прогалин
И, млея... тихо манит новью
    Среди снегов - синея даль...

Е. Кузьмичев
