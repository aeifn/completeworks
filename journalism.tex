\section{Раздел книги «1917 год в деревне»}

Московская губерния

Крестьянское движение в Московской губернии впервые отмечается в мае в двух уездах. В июле участвовали в движении все 13 уездов губернии. В октябре тоже. Различныеформы движения по своей доле во всем движении губернии располагаются в убывающем порядке приблизительно следующим образом: захваты лугов, захваты земли, дезорганизация частновладельческих хозяйств, противоправительственное движеие, движение селькохозяйственных рабочих.

Из воспоминаний крестьянина Е.К.Кузьмичёва Волоколамский уезд, Московской губ.*) В нашей местности «передовики» – крестьяне в начале революции, желая угодить всему населению, перед собранием решили отслужить молебен с попами и певчими, да не тут то было. Черносотенцы и на улицу не показались. А к вечеру передали от земского начальника бумажку, в которой тот сначала уговаривал не верить нелепым слухам о свержении царя, а потом грозил нас всех загнать туда, куда Макар телят не гонял, за что к строптивому пришлось немедленно послать новых милицейских исполнителей и посадить его на несколько дней безвыездно дома. Но таких остолопов помещиков было немного. Некоторые волоколамские помещики знали о приходе революции десятки лет назад. И землю через крестьянский банк постарались продать заранее (Д. П. Шипов). Другие до 1917 г. большую половину сдавали в аренду, так как посев льна то того вздувал цену, что помещики, без всякой заботы о хозяйстве, совсем пересели на шею мужика и куда хотели туда на нем и ездили. Поэтому в начале революции у нас не было резких аграрных волнений. Но мужики-арендаторы инстинктивно поняли, что владычеству господ пришел конец. Сначала они стали задерживать выплату арендных денег, а потом и совсем перестали о них заботиться. Крупней всех землевладельцев был С. Б. Мещерский, при селе Лотошине, прежде Тверской, а теперь Московской губернии, Волоколамского уезда. Его строевые леса протянулись на сотни верст. Его земля карту Кульпинской волости покрывала почти всю, только пятая доля оставалась под крестьянским наделом. Кроме своего культурного хозяйства, поддерживаемого субсидией, он сдавал покосы, пахот и выпасы под круговую обработку ближайшим селениям. На огромное пространство растянулась Мещерская вотчина и вся поголовно работала земельному королю, вплоть до революции. И когда сбагрили помещика-паука, то иначе и не хотели думать крестьяне, как следующим образом: «Помещик был наш, мы ему работали, и достояние, бывшее у него, нам одним должно достаться». Растаскивали запасный спирт, делили и уводили по домам скот. В десяти верстах крупное поместье Ярополец графа Чернышева-Кругликова. В последние годы имение Безобразова тоже протянулось далеко-далеко. До революции велось хозяйство , но большей частью земля сдавалась в аренду за деньги. Эйлеровское имение – в шести верстах от нашей деревни. Велось порядочное хозяйство. Как и в Лотошине, было налажено молочное доходное дело, но здесь не вырабатывали ни сыр ни масло. Однако имение никогда не обходилось без крестьянской трудовой помощи. В годы империалистической войны много работало военнопленных. Земля вокруг Белой Колпи, от нас в четырех верстах, еще за двадцать лет до революции князем А.В. Шаховским, кроме усадебной, вся сдавалась в аренду.

*Местность, описываемая автором, относилась в 1917 г. частично к Московской губ. (Кульпинская волость, Волоколамский уезда), частично к Тверской губернии (Лотошинская волость, Старицкого уезда). В данное время относится к Волоколамскому уезду, Московской губ.

Рядом, в селе Александровском, княгиня Шаховская никогда и не бывала в своем пустующем поместьи, но были сторож и управляющий, - сами наживали деньги с лесов и земли и большую долю прибавляли к княжескому капиталу. Помещиков можно было бы описывать нескончаемо. Вот какой был обильный урожай на них в нашей местности. От крестьянского труда всякому отделялась львиная доля, а самому мужику доставались в удел горб да грыжа. Зато ближайшим к поместьям крестьян теперь досталась лучшая и большая часть земли в сравнении с отдаленными селениями. Хотя и прошел бурно 1905 год, но к Февральской революции крестьяне Волоколамского уезда политически не были еще подготовлены. И если полезли с подпиской к эсеровской партии, то совершенно бессознательно. Из моих личных наблюдений — в книжку уездного комиссара ежедневно записывались десятками и больше. Но во все время до Октябрьской революции из записавшихся никто ни в чем себя не проявлял. Не проявили себя и те, что записывали крестьян… Правда, изредка на собраниях спорили с большевиками. Крестьяне совершено терялись в недоумении, не зная, кто друг. Шли больше по инстинкту. При выборах в Учредительное собрание появлялись агитаторы, но настолько слабые, что путали программы партий до умопомешательства. Большинство крестьян все-таки нашли своих и голосовали за 5-ый номер, большевистский. Помню — в то время 5-му номеру большую услугу оказали появившиеся в деревне кронштадтские матросы, рабочие и демобилизованные. Что надо было крестьянам от большевиков? Большевики стояли за немедленный захват земель от помещиков. «Леса, луга и недра ваши». Во время голосования, ближайшая к нам, самая спорная и, подчас, бестолковая деревня — и то вместо одного прислала четырех крестьян караулить урну с билетами 5-го номера, за который они открыто голосовали. Негде ложиться. В избе тесно. «В сенях простоим всю ночь, а от ящика не отойдем». Вот как горячо шли навстречу. Церковные земли в нашей местности повсюду были немедленно поделены крестьянами. Церковнослужителям было сказано прямо и честно: «Хотите землей пользоваться, - пользуйтесь вместе с нами. Люди все одинаковы. Мы с зарей встаем косить, и ваше семейство пусть поднимется. Мы — пахать, и вы — с нами. Так-то лучше, скорей поверите, отчего у мужика к погоде поясница ломит да грыжа покоя не дает». После того беднота бывшая стала наедаться хлебом. Захваты у кулаков происходили только по решениям земельных комиссий, тогда, когда собственники, несмотря на предписание земельного отдела вступить в общество, шли косить. Скошенное сено общими силами мужики сгребали, делили по едокам. Помещикам же сразу дали понять, что земля будет наша. К ним не шли на работу и не платили деньги за аренду. Далее, на собраниях, в присутствии доверенных помещичьих управляющих, решали оставить столько «барину», сколько он сможет обработать собственной семьей без найма. И, несмотря на присутствие представителей разных партий, такие собрания выносили единогласные резолюции. Иногда в земельные комисси поступали наивные жалобы помещиков на захват приволья для скота. Из-за неисполнеия общественных арендных договоров выезжала на место споров милиция. Но все решалось революционно в пользу большинства. Видя хотя и бескровный, но такой всеобщий захват их земли, помещики день ото дня становились трусливее. И наконец стали совсем исчезать с горизонта революции. С первой победной вестью пролетариата на автомобиле приехали в волость несколько рабочих, с ними наши земляки. Сказали, что земская единица уничтожится. Будут избраны советы на новых началах, и уехали, оставив нам новые нерешенные заботы. Следующее собрание было бурное. В земстве, хотя и слабая, но нашлась защита Керенского. Целый день цедили никчемные слова нерешительные люди, удерживая крестьян от смелой большевистской революции. Пишущему эти строки наконец пришлось громко отчеканить о слабых и чуждых нам действиях Керенского, о том, что если не признать красных героев Октября, то снова появятся только что изгнанные помещики, над спиной труженика снова засвистит арапник царских приспешников. Придется крестьянам исправлять непоправимое, расставлять по осиротелым пням срубленные головы строевого леса и строить барские «хоромы», веками пополнять утробу прожорливого капитала. «Если мы — беспартийные крестьяне, не большевики по записи, тем не менее вляемся самыми ярыми их товарищами на деле». Последние слова попали в цель. Послышались сначала отдельные восклицания: «Мертвого с погоста не носят», «Большевики к нам ближе других партий», «Это верно — правильно! Голосуем за большевиков!» - подхватило собрание, а к нему с ревом присоединилась толпа слушателей, стоящая на улице. А через неделю, с какой-то затаенной поспешностью, из земского склада за полцены и меньше кулакам продавалось все ценное и необходимое в крестьянском обиходе, как то: молотилки, веялки, косилки, железные плуги «Сакка», бороны «Зигзаг» И ПР. Как при сдаче осажденной крепости, земские работники ничего не хотели оставить ненавистным им большевикам… Но наш крестьянский инстинкт не ошибся, - мы действительно в большевиках нашли свою родную партию, всегда смело и дружно стоящую за угнетенных и обиженных злыми хищниками, родителями капитала.

Е. К. Кузьмичев. Крестьянин. Родился в 1867 г. Окончил начальную школу. Активный селькор. Беспартийный. Ативный участник крестьянского движения в 1917 году. Крестьянский поэт. В настоящее время занимается сельским хозяйством и литературной работой, а также является членом с-х секции Московского земельного отдела.

Адрес: деревня Васильевская, Раменской волости, Волоколамского у., Московской губ.
